\documentclass[9pt,twocolumn,twoside]{styles/osajnl}
\usepackage{fancyvrb}
\usepackage[colorinlistoftodos,prependcaption,textsize=normal]{todonotes}
\newcommand{\TODO}[2][]{\todo[color=red!10,inline,#1]{#2}}
\newcommand{\GE}{\TODO{Grammar}}
\newcommand{\SE}{\TODO{Spelling}}
\newcommand{\TE}{\TODO{Term}}
\newcommand{\CE}{\TODO{Citation}}
\journal{i524} 

\title{ Analysis of Pentaho}


\author[1,*]{Bhavesh Reddy Merugureddy}

\affil[1]{School of Informatics and Computing, Bloomington, IN 47408, U.S.A.}


\affil[*]{Corresponding authors: bmerugur@umail.iu.edu}

\dates{Paper 1, \today}

\ociscodes{Pentaho, Data Integration, Big Data, Community, ETL, MapReduce, SQL, Hadoop, OLAP}

% replace this with your url in github/gitlab
\doi{\url{https://github.com/bhavesh37/sp17-i524/blob/master/paper1/S17-IR-2018/report.pdf}}


\begin{abstract}
Pentaho is a leading business analytics and data integration tool that
provides a qualified open source-based platform to assist a variety of
big data deployments. It enables different organizations to utilize
their data which helps them in delivering their services efficiently
with minimum risk.Pentaho is often considered as an ideal application
which can be used by businesses that desire to get the most out of
their data and can also be used for embedded analytics. \newline
\end{abstract}

\setboolean{displaycopyright}{true}

\begin{document}

\maketitle

\TODO{This review document is provided for you to achieve your
  best. We have listed a number of obvious opportunities for
  improvement. When improving it, please keep this copy untouched and
  instead focus on improving report.tex. The review does not include
  all possible improvement suggestions and for each comment you may
  want to check if it applies elsewhere in the document.}

\TODO{Abstract: Sounds a bit too much like a press release. Words and
  phrases like "leading", "qualified", "minimum risk", as well as the
  whole last sentence, are subjective. Please revise.}

\TODO{The paper in general sounds too much like an advertisement or
  press release. Your goal should be to inform readers about 1) the
  context in which using Pentaho makes sense, and 2) what Pentaho can
  do and how it compares to other similar services. In many cases,
  you've used assertions or words that don't belong in a neutral
  comparison like this. Please see the comments below.}

\TODO{Please use more descriptive names for your references instead of
  "pent1", "pent2", etc.}

\TODO{Assessment: Revisions required. Please address the review
  comments by end of March.}

\section{Introduction}

Pentaho can be viewed as a business intelligence suite that provides
data mining, reporting, dashboarding and data integration
capabilities. Generally, organizations tend to obtain meaningful
relationships and useful information from the data present with
them. Pentaho addresses the obstacles that obstruct them from doing so
\cite{pent1}. \TODO{What are these obstacles? You can help
  understanding by providing more context here.} The platform includes
a wide range of tools that analyze, explore, visualize and predict
data easily \TODO{"easily" is subjective; please don't use} which
simplifies blending the data. The sole objective of pentaho is to
translate data into value. \TODO{This sentence reads like an
  advertisement. Translating data into value is such a general
  statement it can be applied to any of a large number of
  technologies. You need to focus on exactly what Pentaho does.} Being
an open and extensible source, pentaho provides big data tools to
extract, prepare and blend any data. \TODO{What do you mean by
  blending data? I am not familiar with this term.} Along with this,
the visualizations and analytics will help in changing the path that
the organizations follow to run their business.  \TODO{Again, this
  statement is too general and thus useless. Please, remove.} From
spark \TE and hadoop \TE \TODO{These are proper names, so need to be
  capitalized} to noSQL, pentaho \TE transforms big data into big
insights. \TODO{Another statement that is too general and needs to be
  removed.}

\section{Pentaho community}
\TE \TODO{Community should be capitalized since it's part of the
  name.}

Pentaho provides two different editions, community \TE edition and \TE
enterprise edition. As the name suggests, the enterprise edition comes
with more packages to provide addition \GE support. The community \TE
edition enables the developers or users to create complex solutions
for the problems pertaining to their business \cite{pent2}. The
pentaho \TE community has a group of intellectual \TE
\TODO{"intellectual" is subjective, please don't use} people and helps
the users in becoming a part of them and benefit \GE from the open
source contributions. These open source projects are helpful in
delivering reliable and faster products which are timely tested by the
community. The community includes all users like developers, testers
and managers. \TODO{The preceding two sentences are out of scope. You
  don't need to explain what an open source community is; just focus
  on Pentaho.} Generally, the community edition platform enables the
developers to sketch their design and develop a rough version of their
product after which they can upgrade to enterprise edition for final
production. \TODO{Here, it would be useful to state more explicitly
  what the difference between the Community and Enterprise versions
  are. Why does the Community edition allows users to sketch a
  solution, but not implement a working solution?}

Pentaho provides an interactive console to its users. With a few
clicks of the mouse, users are allowed to interact with new data
models and data. The platform hides the database connections and
underlying application server and provides access to various data
sources \cite{pent3}. It provides metadata management capabilities and
a dashboard to allow the administrators set security levels, monitor
servers and set user access. There are many server plugins and desktop
applications provided by pentaho \TE.

\subsection{Server applications}

\TODO{Why is this a subsection in the "Pentaho Community" section?
  Please, improve the organization of your sections.}

Business Intelligence platform is a basic service that provides
reports, displays dashboards, reports business rules and performs OLAP
analysis. The latest version \TODO{A paper like this should be
  independent from a specific version or time, so avoid talking about
  latest version} comes with RESTful services and re-written scheduler
along with a migration system. It generally runs in Apache java \TE
application server and can be embedded in any other java \TE
application server \cite{pent1}. Pentaho analysis service is another
server application that is written in java which primarily focuses on
online analytical processing. It aggregates data into a memory cache
by performing read operation from data sources like SQL. It comes with
the pentaho platform in both the editions. These are some of the
server applications provided by pentaho.

\TODO{It's not immediately clear that these are Pentaho
  services. Please, clarify this at the beginning of the paragraph or
  as you introduce the BI platform and Analysis service.}

\subsection{Desktop applications}

Pentaho data mining is a desktop application that searches for
patterns in data by performing knowledge analysis. All the techniques
of data mining \TODO{It's better to be more specific here. No system
  can be employed by a single system since they are constantly
  changing. It's better to say something like "Common data mining
  techniques like clustering, ..., are employed in ..."} such as
classification, clustering, regression and visualization are employed
by this application along with some machine learning algorithms. This
helps the users in predicting the trends in future. Pentaho metadata
editor is an application that is used as an abstraction layer from the
underlying data sources and helps the users in creating effective
\TODO{"effective" is subjective} business models which can be used by
other applications in creating reports for the analytics. There are
many more useful desktop applications. \TODO{Provide some examples.}
 
\subsection{Server plugins}

\TODO{Please, give more context. You immediately jump to introduce
  certain plugins without explaining why the plugin system exists.}
Some of the important server plugins are community data access and
data browser. Community data access is a pentaho \TE server plugin
that provides a common layer on the business analytics server for an
easy data access. It runs the server by providing a REST interface and
gets back the results in various forms such as xml, csv or
json. Community data browser is a plugin that helps R in performing
analytics on the data. It does the job of supplying queries to R by
using online analytical processing browser.

\section{Data Integration}

Extract, transform and load (ETL) are the basic operations that act as
a tool for transforming data from one database and placing in other
database. These processes can be carried out in pentaho \TE with the
help of a component called pentaho data integration \TE, which is also
referred as kettle \cite{pent4}. The most useful functions of pentaho
data integration include massive load of data into databases, data
cleansing, migrating data between applications and integrating several
applications. It is metadata oriented and can be used as a standalone
application. The ability of transforming data is so high that the data
can be manipulated with a very few limitations. \TODO{This sentence is
  subjective. Perhaps, provide some examples of things you can do with
  Pentaho that would be difficult to do with another tool.} Various
input and output formats such as datasheets and text files are
supported by pentaho data integration \TE.

The transformation process undergoes three steps, input step,
transformation step and output step. In the input step, data is
ingested \cite{pent5} \TODO{What does it mean for data to be ingested?
  Please, use a different term or provide more context.}. The data is
then processed within pentaho data integration and the transformed
data is given out in the output step. All these steps are carried out
in parallel. The throughput of transformation process is restricted to
speed of the step which is slowest. The slowest step is often referred
as bottleneck. To improve the performance of transformation process,
two steps are run in a loop which are, identification of the
bottleneck and continued improvement of bottleneck until it is no
longer a bottleneck. \TODO{Please, clarify how Pentaho helps in
  correcting these bottlenecks. This is not clear from this
  paragraph.}

Pentaho data integration has a set of components that contribute to
its functionalities. They are spoon, kitchen, pan and carte
\cite{pent6}. \TODO{When introducing new terms, it's good to italicize
  them or use some other formatting to show these are being introduced
  for the first time. In addition, you need to capitalize all these.}
Spoon can be considered as a desktop application that creates simple
and even complex extract, transform and load (ETL) jobs without making
the users write or read code. Spoon is the application that is used
for transformations and jobs with the help of editor. So, \TE
\TODO{"So" is too colloquial for a paper like this} it is the one that
is used in most of the cases such as editing, debugging or running a
transformation or a job. As the transformations are created in spoon
\TE, they can be executed with the help of a standalone command line
process called pan \TE. It is an engine that reads data, manipulates
it and loads into various data sources. Kitchen is another standalone
command line process that for executing jobs. It schedules different
jobs to run at regular intervals. Carte provides remote execution
capabilities and a medium for setting up a remote ETL server.

\section{Architecture}

\TODO{You should start with this section before you talk about desktop
  applications, server plugins and so on.}  Pentaho architecture can
be considered as a set of four components which are presentation
layer, business intelligence platform, data and application
integration and third party applications. Data can be provided to the
presentation layer by reporting, analysis or process management. This
data can then be accessed through a web service, portal or a browser
\cite{pent7}. The security and repository issues are dealt by the
business intelligence platform. Data integration and third party
applications are respectively, the integration layer and applications
with database from various sources.

The architecture also includes a set of predefined layers such as data
layer, server layer and client layer. \GE Data layer allows an application
to connect to a data source. Server layer serves as a middle layer and
several applications run on the server. Dashboards are provided to the
end users by deploying them on the server along with the required
reports. As mentioned above, a user console is provided that is used
for security and configuration purposes. Client layer is of two forms,
thin client and thick client. Thin client generally runs on a
server. Analyzer and dashboard editor can be considered as the
examples. Report designer and data integration come under thick client
which act as a standalone.

\section{Big Data Use cases}

Big data refers to humongous \TODO{"humongous" is somewhat colloquial,
  you can simply use "large"} volumes of data being taken from
multiple data sources and put into data stores. A use case can be
defined as a technology solution for business specific challenges. Big
data use cases help in understanding the problems that big data
addresses. \TODO{These two sentences are too general and can apply to
  any technology. Please focus on Pentaho.}

Cyber security analysis helps the end users such as data scientists
and security analysts in quickly detecting the threats. Cyber security
analytics allows the users to utilize most of the staff resources via
automation \cite{pent8}. It empowers \TODO{"empowers" is subjective,
  please don't use} the data scientists with predictive analytics with
the help of machine learning tools. It also provides the automation of
blending and reporting on a variety of data. Pentaho platform can be
utilized for data processing, data ingestion and delivery of threat
calls with minimal costs and complexity.

Pentaho optimizes data warehouse and speeds up the development and
deployment processes. It employs a simplified process for offloading
to Hadoop. The offloaded data is usually less frequent data. Hand
coding in MapReduce jobs and SQL can be avoided by the usage of visual
integration tools. It provides access to data sources ranging from
relational to operational to NoSQL technologies. Pentaho MapReduce
helps in achieving high performance in a cluster environment. It
provides a graphical and intuitive big data integration.

Another use case identified by pentaho is the streamlined data
refinery. Pentaho data integration processes and refines different
data sets by using Hadoop as its data processing platform. It provides
modelled, delivered and published data sets to the users for visual
analytics just by a mouse click. It can be seen as an integration
process that blends huge volumes of highly diversified data. It also
supplies tools for in-cluster simplified data processing and is
regarded as a highly practical approach.

Pentaho’s big data support extends the 360-degree view to internal and
external customer related data. Customer service teams are provided
with time-sensitive and blended streams of data. This helps in making
profitable \TODO{"profitable" is subjective} decisions. The presence
of an adaptive big data layer relieves several organizations from
evolving technologies. Customers are given access to customizable,
intuitive \TODO{"intuitive" is subjective} and interactive
dashboards. Data scientists are provided with predictive analytics and
data mining tools.

Monetizing the data is the final use case addressed by pentaho. It
allows the users to capitalize on big data with the help of powerful
data processing and embeddable data analytics \cite{pent9}. Pentaho’s
big data analytics platform empowers easy big data ingestion and
transformation as it works as a no-code data integration
environment. It is a flexible platform that supports security and
deployments specific to customers.

\TODO{This paragraph reads like a press release. Doesn't describe an
  actual use case. }
\section{Comparison}


Pentaho products compete with some big names \TODO{"big names" is
  colloquial, please avoid} in current field such as SAP, IBM and
oracle \TE. Pentaho provides open source solutions and is considered
to be much cheaper than the proprietary equivalents. Jaspersoft is an
established open source rival of pentaho. Though both pentaho and
jaspersoft offer similar features with similar costs, pentaho has got
wider online presence and more followers in social media
\cite{pent10}.

\section{Conclusion}

Pentaho is an open source based platform for diverse big data
deployments. It empowers \TE analytics in any environment by
delivering governed data \TODO{What is governed data?}. It has unified
data integration and analytics components which are comprehensive
\TODO{"comprehensive" is subjective} and embeddable. The primary aim
of pentaho is to enable organizations to find new revenue streams with
extraordinary \TE service at minimum risk \TE. It helps them in
harnessing the value from their data in order to make their operations
efficient and consistent.

\TODO{"extraordinary service", "minimum risk", "comprehensive
  components", etc. These all belong in a press release or an
  advertisement, but not a paper like this.}

\bibliography{references}
 

\newpage

\appendix


\end{document}
