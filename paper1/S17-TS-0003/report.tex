\documentclass[9pt,twocolumn,twoside]{styles/osajnl}
\usepackage{fancyvrb}
\journal{i524} 
%\usepackage[nomarkers,figuresonly]{endfloat}

\title{\centering%
Nagios (Example Paper for I524)}


\author[1]{Tony Liu}
\author[1]{Vibhatha Abeykoon}
\author[1]{Gregor von Laszewski}


\affil[1]{School of Informatics and Computing, Bloomington, IN 47408, U.S.A.}


\dates{status: early draft, paper-1, S17-TS-000?, S17-TS-0003, S17-TS-0001\today}

\ociscodes{Cloud, I524}

% replace this with your url in github/gitlab
\doi{\url{https://github.com/vibhatha/sp17-i524/blob/master/paper1/S17-TS-0003/report.pdf}}


\begin{abstract} Nagios is a system, network and infrastructure
  monitoring tool providing instant awareness of IT infrastructure.
  Nagios allows to monitor the infrastructure, alert the system admin,
  provide visualized reports, schedule downtime for maintenance, and
  plan upgrade in advance with trends and capacity diagrams. The
  design emphasizes highly on flexibility and scalability. We
  summarize details of Nagios and outline how it is useful as part
  ofthe services used in big data. \end{abstract}

\setboolean{displaycopyright}{true}

\begin{document}

\maketitle

\section{Introduction}

Nagios ~\cite{www-nagios, wiki-nagios} is a system, network and
infrastructure monitoring tool under open source license that provides
instant awareness of mission-critical IT infrastructure. Nagios allows
to monitor the infrastructure, alert the system admin, provide
visualized reports, schedule downtime for maintenance, and plan
upgrade in advance with trends and capacity diagrams. Its design
emphasizes highly on flexibility and scalability. To provide such
flexibility, Nagios is composed of different modules. Through the
modular design Nagios cab adapts easily to systems and networks
monitoring needed by different users. It also allows integration of new
components that allow adaptation towards service monitoring needs that
have not yet been distributed with Nagios, allowing extensibility.

\section{Architecture}

Nagios \cite{nagios-paper-2012} flexible modular architecture allows
users to customize modules to monitor the network and
infrastructure. We show the elementary Nagios archirtecture in
Figure~\ref{fig:Nagios-architecture} ~\cite{nagios-book}.

\begin{figure}[htb]
\centering
\includegraphics[width=\columnwidth]{images/nagios-architecture}
\caption{Nagios Architecture~\cite{nagios-book}.}
\label{fig:Nagios-architecture}
\end{figure}

Nagios architecture consists of two main databases. They are MySQL and
Postgre SQL.  Nagios XI is directly connected with MySQL and Postgre
SQL. There are two main components connected with MySQL, NDOUtils and
NagioSQL are these two components. The databases keeps the parameters
and configurations needed for the Nagios Core configuration from the
XI interface \cite{archi-nagios}. The NagiosSQL refers to the
configuration files to get configuration parameters. NagioSQL forms an
advance configuration section of the XI interface to enable the
functionality.Nagios local installation is also connected with three
main components. They are Local Checks, NRDP/NSCA and NRPE components.

Nagios has a number of components such as Nagios core, file system,
Nagios daemon, services, plugins, event handlers, performance processors
and notification commands.

The main component is {\em Nagios core}, which is a scheduler daemon
detecting network devices and services regularly. The core can alert
the administer through various notification methods like email,
message and web interface about system and event changes to alert of
issues and system states. Nagios plugin featues allows to register
events that can be triggered to execute actions actions and print
status updates.

It has two types,check and notification. Both are used by Nagios
core. The check plugin is used to check and monitor devices and
services. The notification plugin is used to send out alerts if the
check plugin detected any status change. Beside these two, the users
themselves can develop their own custom plugins.

Nagios uses {\em modules} that call through the Nagios API Nagios
Events that are enacted upon by the Nagios Broker. The user can
develop a modules with customized functionality and embed them within
Nagios core. Whenever an event triggers the module, the module
will be called to execute. The benefit of Nagios module is that the
user can access all necessary information within the core process such
as the Nagios status and check results.

Nagios' configuration file is text-based. It supports the
sophistication of Nagios al;lowing to monitor large
infrastructure. Also, the user must understand the configurable
options. Management of the configuration has to be condusted carefully
to avoid simple mistakes such as introduced by spelling
errors. However, sample configurations and tools to generate
configuration files by applying such configuration templates are alos
supported by a user-friendly web interface.

Nagios also contains a Web interface. It is developed using
CGI technology in C programming language. Although it is not 
compatible with current popular web technologies like CSS, AJAX
and JQuery, it provides sufficient scalability for larger infrastructures.


\subsection{Comprehensive Monitoring}

Using Nagios there are many structures that can be monitored, such as
operating systems, networking protocols, system metrics and
infrastructure components. There are many APIs to handle the easy
monitoring of in-house applications, services and systems. In
considering the network monitoring, basic use is the pinging and
Nagios uses it in a complex way and configurations are complex as it
is used to monitor many of the networking infrastructures. The feature
is post query support is also there in Nagios, this is very important
to monitor TCP or UDP connections with a particular host. The
capability of pinging and keeping track on multiple ports is also
enabled in this framework, so it is very easy to monitor a huge
network in a particular organization. When it comes to service
management, there are group service configurations that can be added
to the system by Nagios, this provides a variety of options to a team
to manage systems in a flexible way. There are a variety of
centralized views provided. For instance General view provides a
central dashboard which holds the keys to all other kinds of services
and tools like monitoring performance, network outage, etc.  Service
details view provides the last check details, service type, host and
many other information related to services. This is a vital tool, when
it comes to managing thousands of services from a single
organization. Host details and host groups can also be viewed by the
tools in Nagios, this is very important to provide a more structured
service when an organization have a number of clients, so the group
management is a vital task when it comes to isolating services in
cases of maintenance and development. Collapsed tree status map and
marked up circular status map shows the connection between different
hosts and routers in a network. This is an important tool to identify
the service distribution and to identify breakdowns. By these views,
the failures can be notified by a physical and geographical location,
and this is a vital factor for a system management with less downtime
\cite{info-nagios}.

\subsection{Problem Remediation}

Systems running on live environments generally associated with
expected and unexpected problems. Tracking these problems is a
challenging task.  Nagios has the capability of providing such events
as alerts in an automatic way.  Nagios consists of fast recoverying
capability with the support of automated tools. These tools has the
capability of providing services with less downtime. 

\subsection{Visibility and Awareness}

Visibility and awareness deal with a couple of factors. Notifications,
real time updates, reporting, centralized dashboard are some of these
main features. Nagios consists of a centralized dashboard which
provides the access to all these information. The centralized
architecture supports system administrators to deal with less problems
in reaching vivid services in order to access information. In making
decisions, it is vital to have access to all information in a quick
manner.  Nagios provides these features with the centralized
dashboard.

\subsection{Reporting}

In referring to records and statistics, Nagios provide reporting tools
to keep track on the outputs from the dashboards and central
data collection systems, so that the data can be analyzed at the
present moment and also the ability to check the data in the history.
When it comes to tools like crystal reporting, casper
reporting, etc, Nagios provide support to third party tools so the
users can use the familiar reporting tools.

\subsection{Open-Source}

Nagios is an open source tool and the license for documentation and software
is under different licenses. Nagios open software license \cite{op-sw-lic-nagios}, Nagios software
license \cite{sw-lic-nagios} and GPL are the licenses own by Nagios. 

\subsection{Multi-Tenant Capabilities}

Concurrent access for multiple users is a vital factor as far as a
scaled business is considered. The necessity of multiple user login,
accessing data, sharing data and updating data is a vital task in such
an organization. Nagios consists of a web interface which consists of
all these features. The importance of the web interface is that, the
installation cost and maintenance cost is less. Once it is deployed to
a server, every user in that particular network containing the server
has access to the web interface by means of web browser. As mentioned
earlier, Nagios dashboard comes with a web interface enabling easy
access to all users.

In referring to these factors, it is clear how Nagios can be used to
implement a solid foundation to maintain the information technology
based systems in an organization.

\section{Nagios for Big Data}

Big data is another word which refers to data sets generated by
communication channels, networks, security details, etc. Nagios is a
monitoring tool with major supporting tools to enhance network
performance, failure detection, performance evaluation, etc. In
referring to web services which access bulk data in form of JSON
objects or SOAP format, network traffic occurred by these actions are
very high. In such instances, Nagios can be used to monitor the
network in order to provide better performance by analyzing the
current status and providing suggestions on using the bandwith
widely. Furthermore, the monitoring tools and reporting tools can
provide a wide idea on the fact how data is being collected in a
particular network. The rate of change of free capacity of local
storages in servers is a vital factor in case of providing storage
facility with less data damage and no data loss. Nagios has the
capability of providing an analytic understanding on the nature of big
data distribution and management in a particular network.

\subsection{Big Data Requirements for Monitoring}

In monitoring of clouds and clusters, Nagios can be used as a powerful
tool. The clouds and clusters always depend on the network. In such
senarios, network traffic monitoring, failure detection, network
bottle-neck detection and similar tasks can be done using Nagios.
And also the status monitoring such as memory, network bandwith,
cpu performance, gpu performance and similar activities can be
tracked using Nagios. 

Servers, disks and file systems are vital factors in big data management.
The server performances and storage status is a vital factor for an
organization providing cloud based software solutions. So these file system
management becomes a very important task in order to provide less downtime
services. Nagios has monitoring capability and visualizing capability,
these dashboards can be used to provide information to the users in an easy
manner. Real time performance update, storage status, server failures can be
detected using Nagios. All these factors together form a solid background
to provide cloud based software solutions \cite{bigdata-nagios}.  

   
\section{Nagios for Big Data Application}

Big data applications are the applications handling huge amount of
data and thousands of users through out a network. In such a process,
there are many factors that has to be considered. Performance is
one of the main concerns in an organization. In order to achieve better
performance, speed of the network, easy access to data, user management,
security and many more factors contribute. Nagios has the capability of
monitoring most of these factors and it provides updates, reports and
centralized dashboard to system administrators. Big data always
deal with very complex networks, so data flow must be smooth and fast.
Nagios provides the network monitoring tools to understand the nature
of a network in a given time. With the real time updates, the bottle-necks
and network related problems can be traced quickly. Multi-user management
is also a value added feature in Nagios, with big data applications, many
users access same bulk of data in same time. Network monitoring tools and
other performance tools in Nagios provide a wide understanding about user
actions and network traffic distribution. Nagios provide a better infrastructure
to this kind of applications. 

\section{Conclusion}

Nagios provides a wide understanding to the network management, reporting,
centralized dashboard and many more other features enabling a solid
foundation for a scalable organization. It offers a variety of
reports and vivid dashboards under the main panel in order to help
system administrators to make decisions. Nagios can be identified as a long
term solution for big data related systems to manage both users and resources.
Nagios shape up the organizational hierarchy in order to provide flexible
and smooth services to the customers. NO MENTIONING OF BIG DATA

% Bibliography

\bibliography{references}
 
%\newpage


\end{document}
