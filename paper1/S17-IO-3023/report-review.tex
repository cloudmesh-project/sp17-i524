\documentclass[9pt,twocolumn,twoside]{styles/osajnl}
\usepackage{fancyvrb}
\usepackage[colorinlistoftodos,prependcaption,textsize=normal]{todonotes}
\newcommand{\TODO}[2][]{\todo[color=red!10,inline,#1]{#2}}
\newcommand{\GE}{\TODO{Grammar}}
\newcommand{\SE}{\TODO{Spelling}}
\newcommand{\TE}{\TODO{Term}}
\newcommand{\CE}{\TODO{Citation}}
\journal{i524} 

\title{H-Store}

\author[1,*,+]{Karthick Venkatesan}


\affil[1]{School of Informatics and Computing, Bloomington, IN 47408, U.S.A.}


\affil[*]{Corresponding authors: vkarthickprabu@gmail.com}

\affil[+]{HID - S17-IO-3023}

\dates{paper1, \today}

\ociscodes{H-Store,OLTP, I524}

% replace this with your url in github/gitlab
\doi{\url{https://github.com/karvenka/sp17-i524/tree/master/paper1/S17-IO-3023/report.pdf}}


\begin{abstract}
This paper provides information on the H-Store database
technology.H-Store is a complete redesign of the traditional RDMS
database.The paper provides details on the problems that it solves and
its application to the Fast Data problem in Data Processing.  It also
provides details on the architecture and references details on
research done to compare the database to current market OLTP databases
\end{abstract}

\setboolean{displaycopyright}{true}

\begin{document}

\maketitle

\TODO{This review document is provided for you to achieve your
  best. We have listed a number of obvious opportunities for
  improvement. When improving it, please keep this copy untouched and
  instead focus on improving report.tex. The review does not include
  all possible improvement suggestions and for each comment you may
  want to check if it applies elsewhere in the document.}

\TODO{Please format lines your LaTeX source to be 80 characters long
  to make reading and review easier.}

\TODO{Abstract: The abstract needs to focus on H-Store and what is
  most important about it. Currently, only your second sentence does
  that. The first sentence is unnecessary and should be
  removed. Always try to focus on the technology rather than what the
  paper does. Avoid "This paper..." You can leave a last sentence that
  clarifies what the paper will go over, but before that, you need to
  give a little more information about H-Store. The abstract is the
  part of the paper that the most readers will read and will used it
  to decide whether to read anything else. Right now, there isn't a
  lot of information in it to go on.}

\TODO{The "Use Cases" section needs significant revisions since it is
  very general. The "Architecture" section can be improved with better
  organization and motivation. Some issues with references to figures
  and other resources. See below for more detailed comments.}

\TODO{Assessment: Revisions required. Please address the review
  comments by end of March.}

\section{Introduction}

H-Store is an experimental database management system (DBMS) designed
for online transaction processing applications that was developed by a
team of researchers at Brown University, Carnegie Mellon University,
the Massachusetts Institute of Technology, and Yale University.

Traditional SQL databases offer rich transaction capabilities, ad-hoc
query and reporting capability, and vast amounts of standards-based
tooling. Where they fall short is in the ability to scale out to meet
the needs of modern, high-performance applications.NoSQL is a solution
to issues of scale that emerged as organizations began dealing with
immense volumes, velocity, and variety of big data from a multitude of
sources. NoSQL solutions offered availability, flexibility and
scale. However, they came with difficult tradeoffs - sacrifice of data
consistency and transactional (ACID) guarantees \TODO{Please include
  what ACID stands for.}, loss of ad-hoc query capability and
increased application complexity.By removing just the parts of SQL
that hampered scalability and performance, it was able to increase
both while maintaining key advantages of SQL systems.NewSQL
technologies offer the best of both worlds: the scale of NoSQL with
the ACID guarantees, strong consistency, minimized application
complexity and interactivity of SQLThe \SE H-Store DBMS belongs to
this class of parallel database management systems, called NewSQL.

\TODO{Please include a space after each period, comma and similar
  punctuation.}

As noted in \cite{stonebraker2007} \TODO{Avoid "As noted in ...",
  "According to ..." and similar structures. You can just say "Current
  RBMS systems were... " and provide the reference at the end of the
  statement that refers to it.} the current RDBMS systems were
architected more than 25 years ago, when hardware characteristics were
much different than today.  Today \GE processors are thousands of
times faster and memories are thousands of times larger. Disk volumes
have increased enormously, making it possible to keep essentially
everything, if one chooses to in memory. \TODO{This is not precise. By
  "memory" do you mean disk? If so, this is the incorrect term. Memory
  usually implies RAM. Regardless, whether you mean RAM or disk, it is
  not true that you can "keep essentially everything" in either.} The
current day DBMS include the following architectural features which
are not fully utilising the technological advancements in the last 25
years

\begin{itemize}
\renewcommand{\labelitemi}{\scriptsize$\bullet$} 
\item Disk oriented storage and indexing structures
\item Multithreading to hide latency
\item Locking-based concurrency control mechanisms
\item Log-based recovery
\end{itemize}

H-Store is a complete redesign of the traditional RDMS database and
experimetal results show that the OLTP processing on a H-Store
database is faster by a factor of 82 \cite{stonebraker2007}.

\section{Architecture}

\begin{figure}[http]
\centering
\fbox{\includegraphics[width=\linewidth]{hstore}}
\caption{H-Store Architecture} \cite{www-H-StoreArch}
\TODO{You need to actually refer to this figure in the
  text. Otherwise, why is it here?}
\label{fig:false-color}
\end{figure}

H-Store is a parallel, row-storage relational DBMS that runs on a
cluster of shared-nothing, main memory executor nodes.

\TODO{Using keywords like "row-storage", "shared-nothing", "executor
  nodes" without explaining them properly is only confusing. Please
  either expand this or remove altogether.}

A single H-Store instance is a cluster of two or more computational
nodes deployed within the same domain. A node is a single physical
computer system that hosts one or more execution sites and one
transaction coordinator. A site is the logical operational entity in
the system; it is a single-threaded execution engine that manages some
portion of the database and is responsible for executing transactions
on behalf of its transaction coordinator. The transaction coordinator
is responsible for ensuring the serializability of transactions along
with the coordinators located at other nodes.A typical H-Store node
contains one or more multi-core CPUs, each with multiple hardware
threads per core. As such, multiple sites are assigned to nodes
independently and do not share any data structures or memory.

Every table in H-Store is horizontally divided into multiple shards
that each consist of a disjoint sub-section of the entire table. The
boundaries of a table’s fragments are based on the selection of a
column for that table; each tuple is assigned to a fragment based on
the hash values of this attribute. \TODO{Not clear what attribute you
  are referring to.} H-Store also supports partitioning tables on
multiple columns. Related fragments from multiple tables are combined
together into a partition that is distinct from all other
partitions. Each partition is assigned to exactly one site.

All tuples in H-Store are stored in main memory on each node; the
system never needs to access a disk in order to execute a query.It
replicates partitions to ensure both data durability and availability
in the event of a node failure. Data replication in H-Store occurs in
two ways: (1) replicating partitions on multiple nodes and (2)
replicating an entire table in all partitions. For the former, H-Store
adobts the k-safety concept, where $k$ is defined by the administrator
as the number of node failures a database can tolerate before it is
deemed unavailable.

Client applications make calls to the H-Store system to execute
pre-defined stored procedures. Each procedure is identified by a
unique name and consists of user-written Java control code intermixed
with parameterized SQL commands. The input parameters to the stored
procedure can be scalar or array values, and queries can be executed
multiple times.

Each instance of a stored procedure executing in response to a client
request is called a transaction. Similarly as with tables, stored
procedures in H-Store are assigned a partitioning attribute of one or
more input parameters. When a new request arrives at a node, the
transaction coordinator hashes the value of the procedure’s
partitioning parameter and routes the transaction request to the site
with the same id as the hash. Once the request arrives at the proper
site, the system invokes the procedure’s Java control code, which will
use the H-Store API to queue one or more queries in a batch. The
control code invokes another command in H-Store to block the
transaction while the execution engine executes the batched queries.

All of operations within a transaction are atomic across all the
partitions involved in the transaction. When a transaction executes,
it has exclusive access to the data at the partitions that it locks,
therefore all of its operations execute on a consistent view of the
database and are isolated from all other transactions (since no other
transactions execute concurrently at those partitions). Finally, with
snapshots and command logging, the state of the database is completely
recoverable and all changes made by transactions are durable
\cite{kallman2008} \cite{www-H-StoreArch}.

\TODO{It is not clear why you are citing both a paper and a website in
  the same place. It seems like you are trying to refer to these two
  resources for the whole section, which is not correct. You need to
  add the citation next to the first sentence in your paper that
  references it. Then if other important statements use the same
  source, you should add the reference there as well. You shouldn't
  have blanket citations for a whole section of the paper.}

\TODO{This section is a little disjointed. It could use subsections
  and some motivation about why you are talking about these parts of
  the architecture in particular.}

\section{Documentation}

\begin{itemize}
\renewcommand{\labelitemi}{\scriptsize$\bullet$} 
\item Detailed documentation on H-Store  Deployment , Configuration,Debugging and Development is available at  \cite{www-H-Store}
\item H-Store DBMS is open source and complete source code is available in the  \cite{github-H-Store} github repository. 
\end{itemize}

\section{Licensing and Commercial Use}

\TODO{The text and the section title do not match. There is nothing
  here about licensing. In addition, focusing so much on VoltDB seems
  out-of-scope for the paper. Is VoltDB the primary implementation of
  H-Store that you decided to focus so much on it? Are there other
  implementations?}

A commercial implementation of H-Store is VoltDB.VoltDB was developed
for production environments, and thus it is focused on
high-performance throughput for single-partition transactions and
provides robust handling of failures that are obviously needed for a
main memory system. H-Store does not have VoltDB’s tools for managing
and maintaining clusters. But because H-Store is a research database
(thus does not provide many of the safety guarantees that VoltDB
does), this allows H-Store to support additional optimizations, such
as speculative execution and arbitrary multi-partition
transactions.For example, in VoltDB every transaction is either
single-partition or all-partition. That is, any transaction that needs
to touch multiple partitions will cause the VoltDB’s transaction
coordinator to lock all partitions in the cluster, even if the
transaction only needs to touch data at two partitions. However
H-Store has a different internal transaction coordination subsystem
. For each query batch submitted by a transaction, the heavily
optimized H-Store Batch Planner determines what partitions each query
needs to access and only dispatches requests to those
partitions.H-Store uses VoltDB’s VoltProcedure API, so any stored
procedures written for VoltDB should still work in H-Store
\cite{www-H-StoreFaq}.

\section{Use Case}

Big Data is data at rest. Big Data describes data’s volume – petabytes to exabytes - and variety: structured, semi-structured and unstructured data that has the potential to be analyzed for information. Big Data systems facilitate the exploration and analysis of stored, large data sets.
Big data is often created by data that is generated at incredible speeds, such as click-stream data, financial ticker data, log aggregation, or sensor data. Often these events occur thousands to tens of thousands of times per second. The benefits of big data are lost if fresh, fast-moving data from real time sources is dumped into HDFS, an analytic RDBMS, or even flat files, because the ability to act or alert right now, as things are happening, is lost. 

\TODO{This paragraph is out of scope. You don't need to give a
  description of Big Data in the "Use Case" section of a paper on
  H-Store.}

So this data stream which is the source for big data is called Fast Data.For Fast Data we're not measuring volume in the typical gigabytes, terabytes, and petabytes . For Fast Data we are measuring volume in terms of time: the number of megabytes per second, gigabytes per hour, or terabytes per day. So the key difference between Big and Fast Data is Volumne vs Velocity.So to build systems and applications to take advantage of Fast Data  to make real-time, per-event decisions that have direct, immediate impact on business interactions and observations we need a DBMS which  offers high speed transactional ACID performance and the ability to process thousands to millions of discrete incoming events per second is ideal for processing such type of data. Fast data systems operationalize the learning and insights that companies derive from big data.NewSQL systems such as H-Store and VoltDB are ideal for the Fast Data use case \cite{www-FastData}.

\TODO{As before, this is largely out of scope. You can have a sentence
  or two about Big Data and Fast Data at the beginning of the section,
  then go into the use cases. These paragraphs break up the flow of
  the paper and are out of scope.}

Below are four common uses \TE cases for Fast Data 
\begin{itemize}
\renewcommand{\labelitemi}{\scriptsize$\bullet$} 
\item Real-time recommendation engines or hyper-personalization
  applications that can detect and act on individual customer needs in
  real-time.For example real-time analysis of subscriber data based on
  event triggers such as the end of a call, use of the mobile device
  in a particular location, or a user hitting a data usage
  threshold. These insights can then used to initiate real-time
  campaigns and communications, delivered while the customer’s device
  is still in hand. Timely and relevant offers, improved service
  response, and proactive updates on billing/usage result in a
  superior customer experience.
\item Real-time, "down to the last dollar" resource management
  applications, e.g., bid and order management.Managing advertising
  transactions at scale requires the ability to process transactions
  fast enough to align advertising purchases with customer
  budgets. H-Store high speed processing capabilities has the ability
  to meet the real-time demands of mobile advertising customers while
  handling a growing number of transactions – all while keeping client
  ad budget and actual spend aligned. \TODO{This is subjective and is
    more suitable to a press release than a paper on H-Store.}
\item Per-event analytics with automated decision making to enforce a
  policy or authorization level, e.g., credit card fraud, managing API
  calls and authorizations in real time.For example we can apply bank
  business rules to a stream of card swipes, hundreds to thousands of
  times per second in real time. Using stored procedures to manage the
  analytic logic, paired with state held in the DBMS, banks can
  increase their ability to detect fraudulent cards on the first
  swipe. As blacklists and rules change, these can be uploaded into
  DBMS , where the new rules and stored procedures immediately affect
  card processing decisions.
\item Sensor data management in Internet of Things (IoT)
  applications.For example high-velocity data can be used by utilities
  to generate insights on a per-event basis. Leveraging the smart
  energy data, it can alert utilities if their usage is trending
  toward exceeding their forecasted energy budget.Additionally,
  reporting meters can be automatically compared to identify meters
  that did not provide a current status. If a meter misses a defined
  number of consecutive reporting intervals, a technician can be
  automatically assigned to fix or replace it.
\end{itemize}

\TODO{You need to discuss specific H-Store use cases. The above use
  cases could apply to any number of data storage solutions. You
  should try to find cases where H-Store was deployed and what the
  advantages to using traditional DBMSs or NoSQL systems were. The
  whole "Use Cases" section needs to be revised.}

\section{Conclusion}
H-Store and its commercial implementation VoltDB are main memory DBMS
designed for applications running on a cluster of nodes needing high
transaction throughput. They are ideally suited for the Fast Data use
case of Big Data. As an example of a ”NewSQL” DBMS, we can expect them
and other similar main memory Data bases to gradually take over the
Modern Transaction Processing market from legacy disk-oriented vendors
because of superior performance. \TODO{This is subjective and better
  suited for an advertisement or a press release than a scientific
  paper.}

\section*{Acknowledgements}

The authors thank Prof. Gregor von Laszewski for his technical guidance.
% Bibliography

\bibliography{references}


\end{document}
