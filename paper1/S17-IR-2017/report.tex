\documentclass[9pt,twocolumn,twoside]{styles/osajnl}
\usepackage{fancyvrb}
\journal{i524} 

\title{An Overview of OpenNebula Project and its Applications}

%\author[1,2,3]{John Smith}
%\author[2]{Alice Smith}
%\author[1]{Bruce Wayne}
\author[1]{Veera Marni}

\affil[1]{School of Informatics and Computing, Bloomington, IN 47408, U.S.A.}
%\affil[2]{School of Science, University of Technology, 2000 J St. NW, Washington DC, 20036}
%\affil[3]{School of Optics, University of Technology, 2000 J St. NW, Washington DC, 20036}

\affil[*]{Corresponding authors: vmarni@umail.iu.edu}

%\dates{project-000, \today}
\dates{\today}

\ociscodes{opennebula, open cloud, cloud computing}

% replace this with your url in github/gitlab
\doi{\url{https://github.com/narayana1043/sp17-i524/blob/master/paper1/S17-IR-2017/report.pdf}}


\begin{abstract}
OpenNebula orchestrates storage, network, virtualization, combining 
both data center resources and remote cloud resources, according to  
allocation policies. This paper provides insights into how OpenNebula 
can provide the right cloud services for unique needs of each 
organization. \newline
\end{abstract}

\setboolean{displaycopyright}{true}

\begin{document}

\maketitle

\section{Introduction} 

{OpenNebula}\cite{www-wiki-opennebula}is an open cloud platform 
using which each organization setup the right cloud for its 
organizational needs. This quite natural like it had happened with the 
databases and web-servers. {As one cloud could not solve all the 
needs of various work environments OpenNebula helps in setting up and 
deploying cloud platform based on needs}\cite{www-about-opennebula}. 
It is one of the solutions for the management of virtualized data centers
with built-in features available for deployments in hybrid clouds.

Its interoperability makes cloud progress in developments of new 
methodologies to take advantage of the IT assets in turn saving 
investments and completely avoiding lock-in costs for the project. 
Its platform manages a data center's virtual infrastructure to build 
private, public and hybrid implementations of infrastructure as a 
service. Private cloud hosting, on the other hand, by definition is a single-tenant 
environment where the hardware, storage and network are dedicated to a 
single client or company. 

The public cloud is defined as a multi-tenant 
environment, where consumers buy cloud services in a cloud computing environment 
that is shared with a number of other clients or tenants. Private cloud hosting, 
on the other hand, by definition is a single-tenant environment where the 
hardware, storage and network are dedicated to a single client or company. Hybrid 
cloud is mixture of both public and private clouds where one uses both public 
and private clouds.OpenNebula is a cloud computing platform for managing heterogeneous
distributed data center infrastructures.

\section{Architecture of OpenNebula}

The OpenNebula platform manages a data center's virtual infrastructure 
to build private, public and hybrid implementations of infrastructure as 
a service. It provides features at two main layers of Data Center 
Virtualization and Cloud Infrastructure.

\begin{figure}[htbp]
	\centering
	\fbox{\includegraphics[width=\linewidth]{images/functionality.png}}
	\caption{{The OpenNebula Engine for Data Center Virtualization and 
		Cloud Solutions}\cite{www-opennebula-image}.}
	\label{fig:false-color}
\end{figure}

\subsection{Data Center Virtualization Management}

Many users use OpenNebula to manage {data center 
virtualization}\cite{www-dcv-opennebula}, 
consolidate servers, and integrate existing IT assets for computing, 
storage, and networking. In this deployment model, OpenNebula 
directly integrates with virtual machine monitor(VMM) or hyper-visor and
 has complete control over virtual and physical resources, providing advanced features 
for capacity management, resource optimization, high availability and 
business continuity. 

\subsection{Cloud Management}

OpenNebula also provides a multi-tenancy and cloud-like provisioning 
layer on top of an existing infrastructure management solution where a single instance
of a software application servers multiple customers. Each coustmer 
is called a tenant and each tenant may be given ability to customize 
some parts of the application. It helps in provisioning, elasticity and 
multi-tenancy cloud features like virtual data centers provisioning, 
data center federation or hybrid cloud computing to connect in-house 
infrastructure is managed by already familiar tools for 
infrastructure management and operation. 

\section{Integration, API's and Language binding}

OpenNebula has been designed to be adapted to most 
infrastructures and be extended with new components. Its 
interfaces hide most complexities of a cloud by and are suited to meet 
end user needs.

\subsection{System Interfaces}
It is a modular system that can implement a variety of cloud 
architectures and can interface with multiple data center services. 
It interfaces can be classified into 2 categories

\begin{enumerate}
	\item end-user cloud interfaces
	\item system interfaces
\end{enumerate}

Cloud interfaces are primarily used to develop tools to the end-user, 
and they provide a high level abstraction of the functionality 
provided by the cloud. They are designed to manage virtual machines, 
networks and images through a simple and easy-to-use REST API. OpenNebula 
features a EC2 interface, implementing the functionality offered by the 
Amazon's EC2 API, mainly those related to virtual machine management. 
In this way, you can use any EC2 Query tools to access your OpenNebula 
Cloud.

{System interfaces}\cite{www-opennebula-systeminterfaces} expose the 
full functionality of OpenNebula and are 
mainly used to adapt and tune the behavior of OpenNebula to the 
target infrastructure. The XML-RPC interface is the primary interface 
for OpenNebula, exposing all the functionality to interface the 
OpenNebula daemon. The OpenNebula cloud API provides a simplified and 
convenient way to interface with the OpenNebula core XMLRPC API. 
OpenNebula also includes 2 language bindings for OCA: Ruby and JAVA. 
The OpenNebula OneFlow API is a RESTful service to create, control 
and monitor service to create, control and monitor services composed 
of interconnected VMs with deployment dependencies between them.

\subsection{Infrastructure Integration}

The interactions between OpenNebula and the {Cloud 
infrastructure}\cite{www-opennebula-infraintegration} are 
performed by specific drivers. Each one addresses a particular area:

\textbf{Storage} The OpenNebula core abstracts storage 
operations that are implemented by specific programs that can be 
replaced or modified to interface special storage back-ends and file 
systems.

\textbf{Virtualization} The interaction with the hypervisors are also 
implemented with custom programs to boot, stop or migrate a virtual 
machine. This allows you to specialize each VM operation so to 
perform custom operations.

\textbf{Monitoring} Monitoring information is also gathered by 
external probes. You can add additional probes to include custom 
monitoring metrics that can later be used to allocate virtual 
machines or for accounting purposes.

\textbf{Authorization} OpenNebula can be also configured to use an 
external program to authorize and authenticate user requests. In this 
way, you can implement any access policy to Cloud resources.

\textbf{Networking} The hypervisor is also prepared with the network 
configuration for each Virtual Machine.

\section{Ecosystem}

The {OpenNebula Ecosystem}\cite{www-opennebula-ecosystem} is formed 
by external tools and extensions 
that complement the functionality provided by the OpenNebula Cloud 
Management Platform. In addition, the Ecosystems built around the 
cloud interfaces implemented by OpenNebula, Amazon AWS and OGC OCCI, 
can also be leveraged. 

\section{Use Cases of OpenNebula}

\subsection{For the Infrastructure Manager}

OpenNebula responds to infrastructure needs for services with 
dynamic resizing of the physical infrastructure by adding new hosts 
and dynamic cluster partitioning to meet capacity requirement of 
services. It has centralized management of all the virtual and 
physical distributed infrastructure. It can improve the utilization 
of existing resources in the data center and infrastructure sharing 
between different departments managing their own production clusters, 
so removing application silos. It improves operational saving with 
server consolidation to a reduced number of physical systems, so 
reducing space, administration efforts, power and cooling 
requirements. It has also reduced infrastructure expenses with the 
combination of locate and remote cloud resources so eliminating the 
over purchase of systems to meet peak demands.

\subsection{For the Infrastructure user}

It is built for fast delivery and scalability of services to meet 
dynamic demands of service end-users. It supports heterogeneous 
execution environment with multiple, even conflicting, software 
requirements on the same shared infrastructure. It also provides full 
control of the life cycle of virtualized services management.

\subsection{For System Integrators}

It fits into any existing data center due to its open, flexible and 
extensible interfaces, architecture and components. It can build any 
type of cloud deployment. It is open sourced under Apache license and 
has seamless integration with any product and service in the 
virtualization/cloud ecosystem and management tool in the data 
center.

\subsection{Usage in Industry}
{BIT is a business to business internet service provider in the Netherlands 
specialized in colocation and managed hosting. It has tested both OpenStack and
OpenNebula in a lab environment where they found OpenNebula served them 
better than Openstack}\cite{www-opennebula-bit}. {For more look into user testimonials on 
OpenNebula webpage}\cite{www-opennebula-usertests}.

\section{Key Features and Components OpenNebula}

\subsection{Features}
There are several key {features of 
OpenNebula}\cite{www-features-opennebula} for the comprehensive 
management of virtualized data centers to enable private, public and 
hybrid clouds. Some of these key features are interfaces for cloud 
consumers, service management and catalog, interfaces for 
administrators and advanced users, appliance market place, 
chargeback, capacity and performance management, high availability 
and business continuity, virtual infrastructure management and 
orchestration, external cloud connector, platform independent, 
security, integration with third-party tools, fully open-sourced, 
automatic upgrade process, quality assurance and community support.

\subsection{Components}

OpenNebula has several {advanced components}\cite{www-components} 
that can be easily 
integrated and deployed. Some of the important components that are 
extensively used are listed below:

\begin{enumerate}
	
	\item Multi-VM Applications and Auto-scaling
	\item Host and VM Availability
	\item Data Center Federation
	\item Cloud Bursting
	\item Application Insight
	\item Public Cloud
	\item MarketPlace
	
\end{enumerate}

\section{Disadvantages}

There are also some drawbacks that needs attention before one choose to 
use OpenNebula. Some of them are Greater dependency on service providers, 
Risk of being locked into proprietary or vendor-recommended systems, 
Potential privacy and security risks of putting valuable data on someone else's system. 
Another important problem what happens if the supplier suddenly stops 
services. Even with this disadvantages the technology is still used 
greatly in various industries and many more are looking forward to 
move into cloud.

\section{Educational Material}

A great place to start is by reading the {OpenNebula 
documentation}\cite{www-opennebula-documentation}. 
{Future Systems has a tutorial on their 
webpage}\cite{www-opennebula-tutorial} to get started with 
OpenNebula. {Handbook of Research on High Performance and Cloud 
Computing in Scientific Research and Education}\cite{book-hpc} by 
Marijana Despotovic-Zrakic is a book to get started from the 
basics of cloud computing and understanding the working of OpenNebula.

\section{Conculsion}

The OpenNebula's architecture and its integration to adapt to most 
infrastructures make OpenNebula's ability to handle public, private and hybrid
clouds with ease. OpenNebula has reduced  efforts for handling cloud services 
 by reducing costs of infrastructure, increasing the utilization of available resources, 
increase the computing power by integrating different machines and simplifying the 
development and deployment in industry. 

\bibliography{references}

\end{document}
