\documentclass[9pt,twocolumn,twoside]{styles/osajnl}
\usepackage{fancyvrb}
\usepackage[colorinlistoftodos,prependcaption,textsize=normal]{todonotes}
\newcommand{\TODO}[2][]{\todo[color=red!10,inline,#1]{#2}}
\newcommand{\GE}{\TODO{Grammar}}
\newcommand{\SE}{\TODO{Spelling}}
\newcommand{\TE}{\TODO{Term}}
\newcommand{\CE}{\TODO{Citation}}
\journal{i524} 

\title{An Overview of OpenNebula Project and its Applications}

%\author[1,2,3]{John Smith}
%\author[2]{Alice Smith}
%\author[1]{Bruce Wayne}
\author[1]{Veera Marni}

\affil[1]{School of Informatics and Computing, Bloomington, IN 47408, U.S.A.}
%\affil[2]{School of Science, University of Technology, 2000 J St. NW, Washington DC, 20036}
%\affil[3]{School of Optics, University of Technology, 2000 J St. NW, Washington DC, 20036}

\affil[*]{Corresponding authors: vmarni@umail.iu.edu}

%\dates{project-000, \today}
\dates{\today}

\ociscodes{opennebula, open cloud, cloud computing}

% replace this with your url in github/gitlab
\doi{\url{https://github.com/narayana1043/sp17-i524/blob/master/paper1/S17-IR-2017/report.pdf}}


\begin{abstract}
This paper provides insight into how OpenNebula can provide the right 
cloud services for unique needs of each organization by managing data 
center's virtual infrastructure to build private, public and hybrid 
implementations of infrastructure as a service.

\end{abstract}

\setboolean{displaycopyright}{true}

\begin{document}

\maketitle

\TODO{This review document is provided for you to achieve your
  best. We have listed a number of obvious opportunities for
  improvement. When improving it, please keep this copy untouched and
  instead focus on improving report.tex. The review does not include
  all possible improvement suggestions and for each comment you may
  want to check if it applies elsewhere in the document.}

\TODO{Abstract: Your abstract should convey what OpenNebula is and the
  most important things about it. It's the part of the paper that most
  people will read and will use to decide whether to read the whole
  paper. Currently, your abstract doesn't achieve its main purpose,
  and can be replaced with a single statement: "OpenNebula provides
  services for managing data centers' virtual infrastructure." The
  rest is either unnecessary -- "This paper provides insight...",
  "unique needs of each organization" -- or unclear -- "private,
  public and hybrid implementations..."; how would someone not
  familiar with OpenNebula or its problem domain know what these terms
  mean? Please, redo the abstract by focusing on what OpenNebula is,
  and what is most important about it. Try to keep sentences short to
  make it more clear; right now your abstract is a single sentence
  that is too long.}

\TODO{Abstract: "center's" should be "centers'"}

\TODO{In general: The paper needs to be revised. Many section consist
  of assertions or subjective valuations that haven't been supported
  by other parts of the paper or by references. This makes the paper
  sound more like a press release than a scientific paper. You need to
  focus on informing what OpenNebula is used for, and provide the
  necessary context as well. Please, see the comments below. Please
  address the review comments by end of March.}

\TODO{Please, don't just comment out parts of the paper that are no
  longer needed, such as the list of dummy authors and affiliations
  from the template. These don't need to be preserved for any purpose
  and need to be deleted from the paper.}

\section{Introduction} 

{OpenNebula}\cite{www-wiki-opennebula}is a \GE open cloud platform 
using which each organization setup the right cloud for its 
organizational needs. This quite natural like it happened with the 
databases and web-servers \GE. {As one cloud could not solve all the 
needs of various work environments OpenNebula helps in setting up and 
deploying cloud platform based on needs}\cite{www-about-opennebula}. 
It is a simple feature rich and flexible solution for the management 
of virtualized data centers.

\TODO{"simple", "feature-rich" and "flexible" are subjective, and are
  more appropriate for a press release, advertisement, or a company
  website than a neutral paper.}

It's \GE \TODO{Please, know the difference between "it's" and "its"}
interoperability makes cloud progress in developments of new
methodologies to take advantage of the IT assets in turn saving
investments and completely avoiding lock-in costs for the project.
It's \GE platform manages a data center's virtual infrastructure to
build private, public and hybrid implementations of infrastructure as
a service. It is a trunkey \SE enterprise-ready solution that includes
all features need \GE to provide private and public cloud services. It
is a open cloud architecture which is simple, open, reliable and
flexible.

\TODO{Again, this paragraph sounds like a press release or an
  advertisement. "turn-key", "enterprise-ready", includes "all
  features needed to provide...", "simple", "reliable", etc. These are
  all content-less, or assertions that haven't been supported in the
  paper. This paragraph needs to be removed or rewritten.}

The paper is organized as follows. First OpenNebula's Architecture is 
discussed followed by Integration, API's and Language Binding. It is 
then followed by OpenNebula ecosystem and licensing. Finally, use 
cases are discussed with respect to its intended users which followed 
by key features and educational resources. It is then concluded by 
high lighting the major drawbacks that need attention in moving 
forward.

\TODO{This paragraph is unnecessary, especially for a short paper.}

\section{Architecture of OpenNebula}

OpenNebula is a cloud computing platform for managing heterogeneous
distributed data center infrastructures. \TODO{This sentence belongs
  in the introduction.} The OpenNebula platform manages a data
center's virtual infrastructure to build private, public and hybrid
implementations of infrastructure as a service. \TODO{Please, provide
  context. What are private, public and hybrid implementations, and
  why are they desirable?} It provides features at two main layers of
Data Center Virtualization and Cloud Infrastructure.

\begin{figure}[htbp]
	\centering
	\fbox{\includegraphics[width=\linewidth]{images/functionality.png}}
	\caption{The OpenNebula Engine for Data Center Virtualization
          and Cloud Solutions. \CE \TODO{If you didn't create the
            diagram yourself, you need to cite where it is coming
            from.}}
	\label{fig:false-color}
\end{figure}

\subsection{Data Center Virtualization Management}

Many users use OpenNebula to manage {data center
  virtualization}\cite{www-dcv-opennebula}, consolidate servers, and
integrate existing IT assets for computing, storage, and
networking. In this deployment model, OpenNebula directly integrates
with hyper-visors \TODO{What are hyper-visors?} and has complete
control over virtual and physical resources, providing advanced
features foe \SE capacity management, resource optimization, high
availability and business continuity.

\subsection{Cloud Management}

OpenNebula also provides a multi-tenant, cloud-like provisioning layer
on top of an existing infrastructure management solution (like VMware
vCenter). It helps in provisioning, elasticity and multi-tenancy
\TODO{Again, please provide context. What are "provisioning",
  "elasticity" and "multi-tenancy"? Your goal is to make someone
  understand what OpenNebula is and why it's useful. All these
  buzzwords need to be explained or not used at all.} cloud features
like virtual data centers provisioning, data center federation or
hybrid cloud computing to connect in-house infrastructure is managed
by already familiar tools for infrastructure management and operation.

\section{Integration, API's and Language binding}

\TODO{You need to introduce what you are trying to achieve in this
  section. Otherwise you have some subsection that don't really flow
  together very much, and each could stand independently. What is the
  connection between them?}

\subsection{System Interfaces}
OpenNebula has been designed to be easily adapted to any
infrastructure and easily be extended with new components. The result
is a modular system that can implement a variety of cloud
architectures and can interface with multiple data center
services. \TODO{What are some examples where this has been done
  successfully? These are assertions that need to be supported by
  evidence.}  It interfaces can be classified into e categories \GE

\begin{enumerate}
	\item end-user cloud interfaces
	\item system interfaces
\end{enumerate}

Cloud interfaces are primarily used to develop tools to the end-user, 
and they provide a high level abstraction of the functionality 
provided by the cloud. They are designed to manage virtual machines, 
networks and images through a simple and easy-to-use REST API. The 
clod interfaces hide most of the complexity of a cloud and specially 
suited for end-users. OpenNebula features a EC2 interface, 
implementing the functionality offered by the Amazon's EC2 API, 
mainly those related to virtual machine management. In this way, you 
can use any EC2 Query tools to access your OpenNebula Cloud.

\TODO{This is a good paragraph; you should try to make the rest of the
  paper take a similar approach.}

{System interfaces}\cite{www-opennebula-systeminterfaces} expose the 
full functionality of OpenNebula and are 
mainly used to adapt and tune the behavior of OpenNebula to the 
target infrastructure. The XML-RPC interface is the primary interface 
for OpenNebula, exposing all the functionality to interface the 
OpenNebula daemon. The OpenNebula cloud API provides a simplified and 
convenient way to interface with the OpenNebula core XMLRPC API. 
OpenNebula also includes 2 language bindings for OCA: Ruby and JAVA. 
The OpenNebula OneFlow API is a RESTful service to create, control 
and monitor service to create, control and monitor services composed 
of interconnected VMs with deployment dependencies between them.

\subsection{Infrastructure Integration}

The interactions between OpenNebula and the {Cloud 
infrastructure}\cite{www-opennebula-infraintegration} are 
performed by specific drivers. Each one addresses a particular area:

\textbf{Storage} The OpenNebula core \TE issue abstracts storage 
operations that are implemented by specific programs that can be 
replaced or modified to interface special storage back-ends and file 
systems.

\textbf{Virtualization} The interaction with the hypervisors are also 
implemented with custom programs to boot, stop or migrate a virtual 
machine. This allows you to specialize each VM operation so to 
perform custom operations.

\textbf{Monitoring} Monitoring information is also gathered by 
external probes. You can add additional probes to include custom 
monitoring metrics that can later be used to allocate virtual 
machines or for accounting purposes.

\textbf{Authorization} OpenNebula can be also configured to use an 
external program to authorize and authenticate user requests. In this 
way, you can implement any access policy to Cloud resources.

\textbf{Networking} The hypervisor is also prepared with the network 
configuration for each Virtual Machine.

\section{Ecosystem}

The {OpenNebula Ecosystem}\cite{www-opennebula-ecosystem} is formed by
external tools and extensions that complement the functionality
provided by the OpenNebula Cloud Management Platform. In addition, the
ecosystems \TODO{needs to be capitalized} built around the cloud
interfaces implemented by OpenNebula, Amazon AWS and OGC OCCI, can
also be leveraged.

\section{Use Cases of OpenNebula}

\subsection{For the Infrastructure Manager}

OpenNebula responds quickly \TODO{"quickly" is subjective unless there
  is something to back it up} to infrastructure needs for services
with dynamic resizing of the physical infrastructure by adding new
hosts and dynamic cluster partitioning to meet capacity requirement of
services. It has centralized management of all the virtual and
physical distributed infrastructure. It can improve the utilization of
existing resources in the data center and infrastructure sharing
between different departments managing their own production clusters,
so removing application silos. It improves operational saving with
server consolidation to a reduced number of physical systems, so
reducing space, administration efforts, power and cooling
requirements. It has also reduced infrastructure expenses with the
combination of locate and remote cloud resources so eliminating the
over purchase of systems to meet peak demands.

\TODO{Most of the preceding paragraph are assertions that need to be
  back up by references. Otherwise, this sounds like an
  advertisement.}

\subsection{For the Infrastructure user}

It is built for fast delivery and scalability of services to meet 
dynamic demands of service end-users. It supports heterogeneous 
execution environment with multiple, even conflicting, software 
requirements on the same shared infrastructure. It also provides full 
control of the life cycle of virtualized services management.

\subsection{For System Integrators}

It fits into any existing data center due to its open, flexible and 
extensible interfaces, architecture and components. It can build any 
type of cloud deployment. It is open sourced under Apache license and 
has seamless integration with any product and service in the 
virtualization/cloud ecosystem and management tool in the data 
center.

\section{Key Features and Components OpenNebula}

\subsection{Features}
There are several key {features of 
OpenNebula}\cite{www-features-opennebula} for the comprehensive 
management of virtualized data centers to enable private, public and 
hybrid clouds. Some of these key features are interfaces for cloud 
consumers, service management and catalog, interfaces for 
administrators and advanced users, appliance market place, 
chargeback, capacity and performance management, high availability 
and business continuity, virtual infrastructure management and 
orchestration, external cloud connector, platform independent, 
security, integration with third-party tools, fully open-sourced, 
automatic upgrade process, quality assurance and community support.

\subsection{Components}

OpenNebula has several {advanced components}\cite{www-components} 
that can be easily 
integrated and deployed. Some of the important components that are 
extensively used are listed below:

\begin{enumerate}
	
	\item Multi-VM Applications and Auto-scaling
	\item Host and VM Availability
	\item Data Center Federation
	\item Cloud Bursting
	\item Application Insight
	\item Public Cloud
	\item MarketPlace
	
\end{enumerate}

\section{Educational Material}

A great place to start is by reading the {OpenNebula
  documentation}\cite{www-opennebula-documentation}.  {Future Systems
  has a tutorial on there \TE webpage}\cite{www-opennebula-tutorial}
to get started with OpenNebula. {Handbook of Research on High
  Performance and Cloud Computing in Scientific Research and
  Education}\cite{book-hpc} by Marijana Despotovic-Zrakic is great \TE
\TODO{"great" is subjective} book to get started from the basics of
cloud computing and understanding the working of OpenNebula.

\section{Conculsion}

OpenNebula is made impact \GE \TODO{How has it made impact? Needs
  reference.} in the way cloud services are offered before it was
introduced by reducing costs of infrastructure, increasing the
utilization of available resources, increase the computing power by
integrating different machines and simplifying the development and
deployment in industry. However, there are some disadvantages that
needs attention. \TODO{The Conclusion should provide a summary of what
  has been covered earlier in the paper. This is the first time you've
  mentioned any disadvantages. Perhaps, have a separate section on
  this.} Some of them are Greater dependency on service providers,
Risk of being locked into proprietary or vendor-recommended systems,
Potential privacy and security risks of putting valuable data on
someone else's system.  Another important problem what happens \GE if
the supplier suddenly stops services. Even with this disadvantages the
technology is still used greatly in various industries and many more
are looking forward to move into cloud.

\bibliography{references}

\end{document}
