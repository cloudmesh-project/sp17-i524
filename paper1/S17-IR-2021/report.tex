\documentclass[9pt,twocolumn,twoside]{../../styles/osajnl}
\usepackage{fancyvrb}
\journal{i524} 

\title{Docker Machine and Swarm}

\author[1]{Shree Govind Mishra}

\affil[1]{School of Informatics and Computing, Bloomington, IN 47408, U.S.A.}

\affil[*]{Corresponding authors:shremish@indiana.edu}
\dates{project-000, \today}

\ociscodes{Cloud, I524, Big Data, Virtualisation, Docker, Docker
  Swarm, Docker Machine , Virtual Machines, VMs, Containers,LInux
  Containers, lightweight containers,LXC}

% replace this with your url in github/gitlab
\doi{\url{https://github.com/Govind273/sp17-i524/blob/master/paper1/S17-IR-2021/report.pdf}}

\begin{abstract}
  Docker is a container-based technology which helps in the packaging
  and shipping of applications quickly and across networks.The usage
  of Docker is improving in the Agile Software workforce and among the
  DevOps because of its attributes of Continous Integration and
  testing. To create a cluster of Docker Engines into a single docker
  virtual engine is used the Docker-Swarm which is tested for more than
  50,000 containers.
\end{abstract}

\setboolean{displaycopyright}{true}

\begin{document}

\maketitle


\section{Introduction}
Docker is an open-source container-based technology. It is an
extension of Linux Containers (LXC): which is a unique kind of
lightweight, application-centric virtualization that reduces overhead
and makes it easier to deploy software on
servers\cite{www-docker-2}. A container allows a developer to package
up an application and all its part including its code, stack it runs
on, dependencies it is associated with, system tools and everything
the application requires running  within an isolated environment. This
makes it easier for programmers and developers to run more apps on the
same server and also to package and ship the apps very frequently.
Docker has been able to popularize the container approach in part
because it has improved the security and simplicity of container
environments. Plus, interoperability is enhanced by its association
with major companies – such as Google, Canonical, and Red Hat – on its
open source element libcontainer.

\section{How Docker differs from VMs}

The virtualization of application can be obtained with Hypervisors or
Virtual Machine Manager(VMM) which makes it easy for applications to
run in isolation with one another while sharing the same
underlying hardware architecture. VM hypervisors such as Hyper-V, KVM,
and Xen are all based on emulating virtual hardware which means they
are expensive in terms of system requirements as each VM requires the
underlying VMM, Memory, and RAM. Instead of virtualizing hardware,
containers rest on top of a single Linux instances. The only
requirement is to keep the virtual machine image small in containers.
Thus, the smaller the image is the less is needed to be stored and the
lesser information is needed to be sent around the network, which
makes them lightweight in comparison to the Virtual Machines. Thus a
large number of containers can be hosted on the same host OS in
comparison to the Virtual Machines. A full virtualized system usually
takes a time to start whereas the Docker container does take even less
than seconds to upstart and running with a lower overhead than the
VMs\cite{www-stackoverflow-docker}.  Containers are potentially much
more efficient than VMs because they are able to share a single kernel
and share application libraries.


\section{Uses Of Docker}

\subsection{Docker for DevOps}
Another major use of Docker is its use in the DevOps
community.``DevOps is the practice of operations and development
engineers participating together in the entire service lifecycle, from
design through the development process to production
support\cite{www-devops}''. Docker is not there to replace other
configuration management tools and instead can be incorporated with
other configuration management tools like Chef, Puppet, Salt or
ansible. The other major benefit of using Docker, Dockerfiles, the
registry and the whole Docker ecosystem is that the development teams
don't have to learn domain specific language, tools or technologies
which are suitable to a specific platform. Docker will
not be domain specific as it can be integrated with other
configuration tools and software as many times Docker can be
made to work with the other configuration management tools. Docker can
also the eliminate the need for a development team to have the same
versions of everything installed on their local
machine\cite{www-docker-1}. While working on to incorporate Docker to
their development workflow, Spotify created Helios, an application that
manages Docker deployments across multiple servers and that alerts the
development teams when the server isn't running the correct version of
a container.

\subsection{Docker as Virtualized Sandbox}
Docker allows systems administrators and developers to build
applications that can run on any Linux distribution or hardware in a
virtualized sandbox without making custom builds for all the different
environments\cite{www-docker-1}. It is easy to deploy Docker
containers in a cloud scenario. Which can easily integrate it with
typical DevOps environments seamlessly (Ansible, Puppet, etc.) or use
it as a standalone.

\subsection{Docker for continuous integration}
Docker can be used as git for continuous integration. Docker is
similar as changes in the system as changes can be tracked just like
changes in the git.  These collaboration features (docker push and
docker pull) are one of the features which are important.The Docker
pull/push are features which help the DevOps teams to easily
collaborate and in quickly building infrastructures together.The
application teams can thus share app containers with operation teams
and they can share MySQL, PosgreSQL, and Redis servers with
application teams\cite{www-docker-2}.


\section{Docker MACHINE}

Docker Machine is a tool that lets you install Docker Engine on
virtual hosts, and manage the hosts with docker-machine
commands\cite{www-docker-machine}. You can use Machine to create
Docker hosts on your local Mac or Windows Machine, on your company
network, in your data center, or on cloud providers like AWS or
Digital Ocean. Docker Machine is the only way to run Docker Engine on
Mac or Windows previous to Dockerv1.12 and starting with the
Dockerv1.12 we have Docker for Mac and Docker for Windows. Thus we can
create a cluster of Docker hosts which is called a Swarm using Docker
Machine.

\section{Docker SWARM}
Docker Swarm provides native clustering capabilities to turn a group
of Docker engines into a single, virtual Docker
Engine\cite{www-docker-swarm}. These helps to scale up the
applications as if these are all running on a single, huge machine. It
does so by providing a standard Docker API where if any tool which
communicates with the Docker, the daemon can use Docker Swarm to
transparently scale to multiple hosts: Dokku, Docker Compose, Krane,
Flynn, Deis, DockerUI, Shipyard, Drone, Jenkins and, of course, the
Docker client itself.  Docker Networking, Volume, and plugins can also
be used through their respective Docker commands via Swarm. Swarm has
been tested and is production ready to scale up to one thousand
(1,000) nodes and fifty thousand (50,000) containers with no
performance degradation in spinning up incremental containers onto the
node cluster Swarm also comes with a built-in scheduler, but you can
easily plugin the Mesos or Kubernetes backend while still using the
Docker client for a consistent developer experience. To find nodes in
the cluster, Docker Swarm can use either a hosted discovery service,
static file, etcd, consul and zookeeper depending on what is best
suited for the environment\cite{www-docker-machine}


\section{Ecosystem Support to Docker}
Finally, it is easy to deploy Docker containers in a cloud.
\cite{www-docker-1}. It can be integrated in a typical DevOps
environments seamlessly (Ansible, Puppet, etc.) or use it as a
standalone. You can do local development within a system that is
identical to a live server, deploy various development environments
from your host that each uses their own software, OS, and settings.
easily run tests on various servers and create an identical set of
configurations, so that collaborative work isn’t ever hindered by
parameters of the local host.Ecosystem support for Docker is improving
with time.
\begin{itemize}
\item[$\bullet$]Operating Systems support to Docker: Is compatible
  with virtually any distribution with a 2.6.32 + kernel RedHat Docker
  collaboration to work with across fedora and other(2.6.32)+.

\item[$\bullet$]It is compatible with Private PaaS(Platform as a
  Service) technologies: Openshift, Solum(Rackspace and Openstack).

\item[$\bullet$]Public PaaS technologies like Voxoz,
   Cocaine(Yandex), Baidu, etc.

\item[$\bullet$]Ecosystem Support for IaaS is present via Rackspace,
  Digital Ocean, AWS(Amazon Web Services), AMI, etc.

\item[$\bullet$]Orchestration tools Support and Integration with:
  Chef, Puppet, Jerkins, Travis, Ansible, etc.

\item[$\bullet$]In Openstack Docker integration into
  NOVA(compatibility with Glance, Horizon)are also present.
\end{itemize}
\section{Shortfalls of Docker}

Though dockers are legacy virtualization techniques, they are not the
goto solution for all kinds of virtualization and cannot be considered
as a replacement of the VMs.
\begin{itemize}
\item[$\bullet$]VMs are self-constrained with as they have a unique
  operating system(OS), drivers and application components whereas the
  containers are dependent as containers under Docker cannot run on
  Windows server.

\item[$\bullet$]VMs provide high level of isolation as the system's
  underlying hardware resources are all virtualized and thus any bugs
  or viruses could not affect the other VM in the virtualized
  environment.

\item[$\bullet$]The kinds of tools and technologies required to manage
  the containers are still lacking in the industry and thus only a few
  management tools from companies like Google and Docker such as
  Kubernetes and Swarm respectively are avalible which are not
  appropriate for an open-source product.
\end{itemize}
\section{Future with Docker}

Docker Inc has set a clear path to the development of core
capabilities (libcontainer), cross service management (libswarm) and
messaging between containers (libchan). Docker's key open source
project is libcontainer. Libcontainers enables the containers to work
with the Linux namespace, control groups, capabilities, AppArmor
security profiles, network interfaces and firewalling rules in a
consistent and predictable way.  Libcontainers is receiving help from
Google, RedHat, and Parallels to build the program as they will work
with docker as core maintainers of the
code\cite{www-docker-1}. Libconatiner which is written natively in
Google's Go is also being ported to other languages. Parallels' libct,
which includes libcontainer's functionality, has native C/C++ and
Python bindings. Microsoft is bringing Docker to their Azure platform,
a development which would integrate the Microsoft products with
Linux applications.



\section{Conclusion}

According to a survey by Docker community it is observed that there
are more than 1200 Docker Contributors, 10,000 Dockerized Applications
at index.docker.io, 300 Million Downloads, and 32,0000 Docker-related
Projects\cite{www-docker-3}. At Docker Ecosystem there is a potential
for some advanced deployment tools that combine containers,
configuration management, continuous integration, continuous delivery,
and service orchestration in the coming days.


\bibliography{references}

\end{document}

