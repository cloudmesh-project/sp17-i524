\documentclass[9pt,twocolumn,twoside]{styles/osajnl}
\usepackage{fancyvrb}
\usepackage[colorinlistoftodos,prependcaption,textsize=normal]{todonotes}
\newcommand{\TODO}[2][]{\todo[color=red!10,inline,#1]{#2}}
\newcommand{\GE}{\TODO{Grammar}}
\newcommand{\SE}{\TODO{Spelling}}
\newcommand{\TE}{\TODO{Term}}
\newcommand{\CE}{\TODO{Citation}}

\journal{i524} 

\title{RabbitMQ}

\author[1,*]{Abhishek Gupta}

\affil[1]{School of Informatics and Computing, Bloomington, IN 47408, U.S.A.}

\affil[*]{Corresponding authors: abhigupt@iu.edu}

\dates{project-001, \today}

\ociscodes{Cloud, I524}

% replace this with your url in github/gitlab
\doi{\url{https://github.com/cloudmesh/sp17-i524/blob/master/paper1/S17-IO-3005/report.pdf}}

\begin{abstract}
RabbitMQ provides simple yet powerful messaging platform which allows
applications pass messages in a reliable and fault tolerant
way. RabbitMQ implements Advanced Message Queuing Protocol (AMQP) and
is written in Erlang programming language. It runs on all major
operating systems and supports SDK in all major programming
languages\cite{www-rabbitmq-develop} including objective-C, swift and
node.js. When we look for messaging, we look for certain features:
asynchronous messaging, large scale, reliability, clustering,
multi-protocol, highly available, fault
tolerant. RabbitMQ\cite{www-rabbitmq-pivotal} fulfills these
requirements and provides a distributed, persistent, highly available,
fault tolerant messaging system which can scale as data grows.
\newline
\end{abstract}

\setboolean{displaycopyright}{true}

\begin{document}

\maketitle

\TODO{This review document is provided for you to achieve your
  best. We have listed a number of obvious opportunities for
  improvement. When improving it, please keep this copy untouched and
  instead focus on improving report.tex. The review does not include
  all possible improvement suggestions and for each comment you may
  want to check if it applies elsewhere in the document.}

\TODO{Please format your lines to be 80 characters long. This helps
  with reading the paper.}

\TODO{Abstract: "simple yet powerful" is subjective, so leave it
  out. Everything else looks good.}

\section{Introduction}

RabbitMQ based\cite{www-rabbitmq-pivotal} \CE \TODO{Citation should
  come right after RabbitMQ} off AMQP (Advanced Message Queuing
Protocol\cite{vinoski2006advanced}) \GE Which defines how client
applications connect to message brokers. Message brokers receive
messages from producers and make it available to consumers. The
producer posts messages to exchanges and broker agent reads these
messages from exchange and writes to queues. \GE Consumers can further
read messages from the queue. It also supports a notion of
acknowledgements where consumers can post an acknowledgement back to
broker and which in turn can post messages back to the producer.

\TODO{Some grammar issues in the preceding paragraph need to be
  addressed. In addition, it would be good to start by explaining what
  problems this producer/broker/consumer scheme is better suited to,
  compared to standard producer/consumer message passing schemes
  someone with a CS background might be familiar with.}

\TODO{Overall, a good overview of RabbitMQ, but there are some places
  where the paper can be more clear. In addition, there are minor
  language issues like not using singular/plural consistently, not
  using always articles ("a", "the"), and not capitalizing where
  appropriate. Finally, you need to review where in the text to best
  place references to your your sources. See below for a more detailed
  breakdown.}

\TODO{Assessment: Revisions required. Please address the review
  comments by end of March.}

Looking back at the history\cite{videla2012rabbitmq} of development of
rabbitMQ, \TE \TODO{Capitalize names properly.} it started with IBM
MQseries in 1993, 1997 Microsoft MQ, 2001 java messaging
service\TE. \TODO{How do these other techs relate to RabbitMQ?} It was
in 2003 where JPMorgan created the first version AMQP \GE which became
the base for RabbitMQ technologies formed in 2006.

\begin{figure}[htbp]
\centering
\fbox{\includegraphics[scale=0.2]{images/mq_timeline}}
\caption{Timeline for evolution of RabbitMQ}\cite{videla2012rabbitmq}
\CE \TODO{The citation needs to be inside the caption.}
\label{fig:false-color}
\end{figure}


\section{Architecture}

\cite{videla2012rabbitmq}At high level \GE \CE \TODO{A reference needs
  to go after the statement that refers to it, not before.}, producer
publishes the message to the broker. Consumer consumes or subscribes
the messages from the broker. Broker is the middle man who has the
knowledge of exchanges and queues. It maintains the bindings between
exchanges and queues. The consumers read the messages from the
queue. Following \GE table shows different exchanges supported by
RabbitMQ and corresponding default
exchanges\cite{www-rabbitmq-pivotal} \CE \TODO{Please, leave a space
  between the reference and the word it follows}.

\begin{center}
 \begin{tabular}{||c c||} 
 \hline
 Name & Default pre-declared names \\ [0.5ex] 
 \hline\hline
 Direct exchange & (Empty string) and amq.direct  \\ 
 \hline
 Fanout exchange & amq.fanout  \\
 \hline
 Topic exchange & amq.topic  \\
 \hline
 Headers exchange & amq.match (and amq.headers in RabbitMQ) \\
 \hline
\end{tabular}
\end{center}

\section{Type of exchanges}
\label{sec:examples}

The sections below explains various types of exchanges supported by
RabbitMQ.

\TODO{So far, you haven't explained what exchanges are, so it's
  difficult to follow. }

\subsection{Default Exchange}

It is created by the broker and all queues are bound to default
exchange unless specified separately. For example we have a queue
called demoqueue, all messages assigned to demoqueue will be routed by
default exchange.

\subsection{Direct exchange}

It is used to route messages to queue with a given routing key. For
example a message queue has routing key K. A message with same routing
key K will be delivered to same message queue.

\subsection{Fanout exchange}

Delivers messages to all queues that are bound to a given exchange
ignoring the routing key. For example if there are 5 queues bound to a
\GE exchange E. Messages to exchange E are delivered to all 5 queues.

\subsection{Topic exchange}

It delivers the messages to a given queue, not just based on the
routing key but a pattern specified in the message called topic. It
can be useful in scenario \GE when consumer/application decides to
which topic(s) it is interested in. Further, it can subscribe to those
topics(or messages).

\subsection{Header exchange}

This exchange ignores the routing key but uses the header parameter to
decide which messages goes to which queue. A message is matched when a
value in header \GE matches with the one specified in the queue
binding.  Exchanges have other important attributes apart from routing
key and exchange type \TODO{Punctutation} for example name,
durability, auto-delete, arguments etc. Durability allow \GE messages
to persist on disk in case of broker restarts. Auto-delete deletes the
exchange, once all queue \GE have finished using the exchange.

\begin{figure}[htbp]
\centering
\fbox{\includegraphics[scale=0.3]{images/exchanges-topic-fanout-direct}}
\caption{Different type of exchanges supported by
  RabbitMQ}\cite{www-rabbitmq-Johan} \CE
\label{fig:false-color}
\end{figure}

\section{Bindings}

Bindings are the rules that define how exchanges will route messages
to the queue. To route all messages from exchange E to queue Q, Q has
to be bound to Exchange E. Bindings use routing key as one of the
criteria to route messages to a queue. However, routing key is
optional and not always applicable. \GE

\section{Consumers}

Messages from message queue are eventually used or consumed by
consumers. Consumers can use push or pull mechanism to consume these
messages. Push API have messages delivered to the consumer whereas a
Pull API is used to fetch messages from the queue.

\section{Message Acknowledgement}

It has a built-in mechanism to send and receive
acknowledgements. \TODO{It's not clear what "it" refers to here.}
Producer can send messages and wait to acknowledgement in the response
queue from consumer. \GE Consumer can receive messages and post
acknowledgement to response queue. Here request queue and request
exchange \GE can be used to send and receive messages. Whereas \GE
\TODO{Don't start a sentence with "whereas."} response queue and
response exchange can be used to send and receive
acknowledgements. This mechanism makes RabbitMQ robust in case of
failures.

\section{Clustering}

\cite{videla2012rabbitmq}To \CE achieve high availability and making
sure producers and consumers send and receive data without knowing
about node failures, clustering was introduced. RabbitMQ follows
OTP(open telecom platform) framework provided by erlang to achieve
high availability. RabbitMQ by default doesn't replicate the content
of queues i.e. all queues are stored on one node in the cluster. To
achieve clustering rabbitmq \TE keeps track of metadata for queue,
exchange, binding and vhost. In case of cluster, \TODO{Not clear. What
  do you mean "in case of cluster?"} it only stores all information
about the queue like metadata, state and contents on one node rather
than all nodes in the cluster. However, it stores metadata and pointer
to actual data on each node in the cluster. This is to optimize
storage space and performance.

\begin{figure}[htbp]
\centering
\fbox{\includegraphics[width=\linewidth]{images/clustering-queue-metadata}}
\caption{Shows queue metadata for a cluster vs single
  node}\cite{videla2012rabbitmq} \CE
\label{fig:false-color}
\end{figure}

On the other hand exchanges are just bindings to the queues. So when
you send a message to rabbitmq \TE exchange, the channel checks the
routing key of the message and compares it against the queue
bindings. Further it sends the message to appropriate queue. Since,
\GE exchanges are just lookup tables for queue bindings, they are
replicated across all nodes on the cluster.

\TODO{It's better to explain exchanges like you have in this paragraph
  earlier in the paper, since they've been discussed quite a bit
  already.}

\section{Management}
RebbitMQ \TE provides all management using:
\begin{itemize}
	\item  web ui \TODO{Needs to be capitalized: Web UI}
	\item REST interface
	\item Rabbitmqctl command line utility
\end{itemize}

Web UI can be used by administrators to create user \GE, monitor
queues and exchanges, view statistics, add configurations etc. Similar
functionality can be achieved by cli \TE utility rabbitmqctl and REST
API. Rabbitmqctl can be used to automate rabbitmq deployments and
management. IT \SE can also be used to write automated tests. REST Api can \GE
used for integrating with 3rd part \TE UI and plugins. Using REST api \TE you
can monitor the number of connections, download or upload a
configuration, list the nodes in the cluster, create or delete
rabbitmq users, view or create virtual host, set permission for a user
etc. You can also list all current APIs using following url:
http://localhost:55672/api with some api documentation and
explanations.

\TODO{Don't list the URL. This is a small technical detail that is not
  in scope for a paper like this, and it is probably subject to
  configuration. What if I changed the port during configuration?}

\begin{figure}[htbp]
\centering
\fbox{\includegraphics[width=\linewidth]{images/rabbitmq-monitoring.png}}
\caption{RabbitMQ management UI showing messages queued and message
  data rates}\cite{www-rabbitmq-Johan} \CE
\label{fig:false-color}
\end{figure}

\section{Licensing}
RabbitMQ is licensed under \href{https://www.rabbitmq.com/mpl.html}
{Mozilla Public License(MPL) and GPL v2}. \cite{www-rabbitmq-pivotal}

\TODO{You should briefly state what this means. What kind of usage
  does MPL allow?}

\section{Use Cases}

RabbitMQ messaging can be useful in applications which require
asynchronous messaging for example an application initiates a task by
posting a message to RabbitMQ, it doesn't have to wait for the task to
get completed. Rather it can periodically check the status of
task. \GE It can be useful to scale up as the data volume grows by
adding additional nodes to RabbitMQ cluster. With distributed
applications where applications run as micro-services in a container
or virtual machine, \GE RabbitMQ is useful to communicate and share
data between different services. RabbitMQ can be managed separately
with its management UI, CLI tool and REST api \TE interface. This
decouples the messaging layer from application \GE and makes the
overall design very \TODO{Avoid adverbs like "very" in a scientific
  paper like this.}  robust.

\section{Conclusion}
RabbitMQ is an open source platform and provides a robust messaging
platform for applications. It provides simple manageability decoupling
with the application. RabbitMQ can scale well when application demand
increases and can handle more data by adding more nodes to the
cluster. Based on test \GE \cite{jones2011rabbitmq}conducted on
RabbitMQ with set of 4K(4096) and 16K(16384) messages, the performance
of RabbitMQ on multi node cluster decreases in comparison with single
node cluster to reach a threshold and then becomes stable. This
decrease in performance can be primarily accounted due to replication
between nodes in the cluster.These tests were conducted on combination
of single publisher and single subscriber, multiple publishers and
single subscriber, multiple publishers and multiple subscribers but
there is no concrete conclusion to these test and numbers. \TODO{Why
  not? Where are the tests lacking, and what other tests would be
  helpful in better understanding RabbitMQ's performance?}

\section*{Acknowledgements}
Special thanks to Professor Gregor von Laszewski, Dimitar Nikolov and
all associate instructors for all help and guidance related to latex
and bibtex, scripts for building the project, quick and timely
resolution to any technical issues faced. The paper is written during
the course {I524: Big Data and Open Source Software Projects, Spring
  2017} at Indiana University Bloomington.

 
\medskip

% Bibliography
% \bibliographystyle{unsrt}
\bibliography{references}

\end{document}
