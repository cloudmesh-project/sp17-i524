\documentclass[9pt,twocolumn,twoside]{../../styles/osajnl}
\usepackage{fancyvrb}
\usepackage{url}
\journal{i524} 

\title{Docker Container}

\author[1,*]{Vishwanath Kodre}

\affil[1]{School of Informatics and Computing, Bloomington, IN 47408, U.S.A.}

\affil[*]{Corresponding authors: vkodre@iu.edu}

\dates{paper1, \today}

\ociscodes{Docker Container, Docker Machine, Docker Swarm, I524}

% replace this with your url in github/gitlab
\doi{\url{https://github.com/cloudmesh/sp17-i524/tree/master/paper1/S17-IO-3008/report.pdf}}

\begin{abstract}
  A portable lightweight packing and run time tool a platform for developers and administrators to build, ship, and run distributed application with Docker Engine. Docker can get code tested and deployed faster. 
\end{abstract}

\setboolean{displaycopyright}{true}

\begin{document}

\maketitle

\section{What is Docker}
Dockers package the application into filesystem, these filesystem contains software need to run, runtime environment, system tools and system libraries anything that can be installed on server. These units are known as Docker containers which are also known as lightweight VMs. Docker container is application image hosted on single machine, but it differs from VM with its architecture. It runs on single machine and shares the same Kernel. The image is constructed from layered filesystem and shares common files, making disk usage and image download much more efficient. Virtual Machine also include application, binaries, libraries and over head of entire guest operating system. "The ability to create multiple lightweight, self-contained execution environments on the same Linux host simplifies application deployment and management. By improving collaboration between developers and system administrators, container technology encourages a DevOps culture of continuous deployment and hyperscale, which is essential to meet current user demands for mobility, application availability and performance."\cite{DockerandtheLinuxcontainerecosystem}.

Docker however differs from conventional container based echo system. Docker enhances the container technology with its open source platform it makes it more accessible by creating simpler and more powerful tool. With the help of Docker the lifecycle of tens of thousands of containers can easily be managed.

\section{Docker Application}
Docker-Machine controls the remote Docker-Engines as if they were locally installed. With Docker-Swarm, installation of thousands of Docker machine becomes as easy as only running job with single command. In practice many cloud platform providers are adapting to Docker containers, Docker Machine and Docker Swarm. Such Amazon AWS or Microsoft Azure working in collaborative form. If the Docker host is setup with help of these, user can access host remotely and one don't have to establish SSH connection to work with specific Docker Engine.

The highly scalable Amazon EC2 Container Service (ECS), supports Docker containers that allows to run applications easily on cluster of Amazon EC2 instances with high performance container management service. "Amazon ECS eliminates the need for you to install, operate, and scale your own cluster management infrastructure. With simple API calls, you can launch and stop Docker-enabled applications, query the complete state of your cluster, and access many familiar features like security groups, Elastic Load Balancing, EBS volumes, and IAM roles. You can use Amazon ECS to schedule the placement of containers across your cluster based on your resource needs and availability requirements. You can also integrate your own scheduler or third-party schedulers to meet business or application specific requirements."\cite{ECS} 

\section{Infrastructure}
Docker don't require any specific infrastructure to setup, unlike VM there is no requirement for guest OS or hypervisor require for container. Docker container contains only those what is necessary to build and run the application. With this organization can manage to reduce the cost over infrastructure, as extra cost over storage and licensing of hypervisor gets reduced.

Docker container are light weight and portable which allows easy movement of the images across any infrastructure and it allows enterprises to leverage these features and help IT operation teams to move workloads across different cloud service, physical servers or virtual Machines which don't demand for specific infrastructure. This way enterprises can optimize their infrastructure and reduce the maintenance cost.

\section{Docker and Big data}
For faster and easily available compute instance Docker plays vital role, by collaborating with OpenStack or Apache Spark etc. Docker containers helps getting these environment ready running the instances in no time and don't consumer much resources.

Project Manila is one of the example of which is community driven project represents the management of shared files as core service to OpenStack. With use of Docker container Project Manila get flexibility to share the compute instances among the clusters which is as easy as copying the image. With Manila shared files user can create new instances (with configuration) which are not in use and publish with VM and Docker container. Use case here is deploying Manila services into OpenStack container, deploying Big Data clusters/services using HEAT into containers. As the Big data processing requires shared file system which Manila provides.\cite{Manila}

Talend Big Data Sandbox is another case which allows user to experiment and test real word Big Data scenarios. Which provides facility to users to work with Apache Spark and Hadoop distribution. With help of Docker containers SandBox provider free preconfigured, easy to use virtual environment. With SandBox user can perform real time analytics of data from multiple streaming sources. Personal recommendation, Visualization with heat map, monitoring of IT operations using Apache weblogs.\cite{SandBox}

Docker Inc is key player in provide Docker container, and are in collaboration providing services along with Amazon EC2 Container Service, IBM Bluemix Container Service, Joyent Smart Data Center and MicroSoft Azure to integrate their offerings with Docker Swarm.

By providing online learning material in terms of tutorials, white papers, Admin guide, video tutorials and blogs and forums to increase the awareness and its usage to enterprises. By leveraging these learning and support enterprises are also adapting Docker and provide their services packaged with Docker container. 

Popularity of Docker container is increasing now a days and many enterprises are adapting use of Docker container, Docker Machine and Docker Swarm. Few are case studies shows the adaptation Docker in running their business day to day. Such as SA Home Loan Adopts Microservices and deploys 20-30 times a day with Docker enabled datacenter.

As SA home loan started with using support and services as RabitMQ and nginx, and move all of their main application services over to container. Which facilitate immutable, transferable development platform and deployment pipline. Also production ready Orchestration service that gives single point from which to manage and distribute containers on nodes.

"SA Home Loans now uses Docker Datacenter, the on-premises solution that brings container management and deployment services to the enterprise via a supported Container-as-a-Service platform that is hosted locally."
\cite{SA}

\section{Similar Technology}
As Docker gains it popularity and dominating the Container-as-a-Service market there are other similar container based technologies and providers are emerging over the period of time such as

1.	Open Container Initiative (OCI)
It is Open System Interconnection initiative similar to OSI layer network communication and integration model.\cite{Alternate-Container}

2.	Kubernetes
"One of the primary advantages of the system is that it provides a consistent object model and API for many underlying resources that vary between cloud providers, and has modules allowing it to run on most of the major ones including Amazon Web Services and, of course, Google Cloud."\cite{Alternate-Container}

3.	CoreOS and rkt
Simple command-line tool "supports multiple container formats, including Docker, and 'pluggable' levels of runtime container isolation, which is useful for certain kinds of system and server applications"\cite{Alternate-Container}

4.	Apache Mesos and Mesosphere
"Mesos is a cluster management system and control plane for efficient allocation of computing resources between application delivery platforms, called frameworks that are layered above it"\cite{Alternate-Container}

"Mesosphere is an enterprise software OEM that sells a 'data center operating system' also built on Mesos and providing cluster management, container orchestration, service discovery and build automation for elastic computing."\cite{Alternate-Container}

5.	Canonical and LXD
"LXD builds on the capabilities of LXC by adding to it a systemwide daemon with an API for LXC container management and an OpenStack Nova plug-in for managing virtual LXD hosts in the cloud"
\cite{Alternate-Container}

Though Docker has gain it popularity these are few technologies provides similar services, with companies like Amazon, Google, IBM in collaborating with few of these to provide cloud base solution to enterprise the service and support backed up by these big players.

\section{Take Away}
Docker is container based ecosystem growing rapidly is most promising disruptive solution on cloud platform. Community is leveraging multiple IaaS service provider for portability, scale on demand, fault tolerance and performance with continuous application delivery.

Docker containers has added the ease of installation to cloud platform over the heavy virtual machines because of its portability and one stop installation module many enterprises are moving towards container based services (Container-as-a-Service).

As increase use of Docker container in business enterprises are really moving more and more towards use of Docker container to help reduce the cost and ease of IT infrastructure maintenance. It is more likely evident the enterprise moving towards and adapting container driven data centers.


\bibliography{references}
 
\section*{Author Biography}
\begingroup
\setlength\intextsep{0pt}
\begin{minipage}[t][3.2cm][t]{1.0\columnwidth} % Adjust height [3.2cm] as required for separation of bio photos.
%  \begin{wrapfigure}{L}{0.25\columnwidth}
%   \includegraphics[width=0.25\columnwidth]{images/john_smith.eps}
%  \end{wrapfigure}
  \noindent
  {\bfseries Vishwanath Kodre} received his Masters Degree in Computer Science from Pune University.  He is currently studying Data Science at Indiana University Bloomington.
\end{minipage}
\endgroup

\end{document}
