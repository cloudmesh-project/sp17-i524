\documentclass[9pt,twocolumn,twoside]{styles/osajnl}
\usepackage{fancyvrb}
\usepackage[colorinlistoftodos,prependcaption,textsize=normal]{todonotes}
\newcommand{\TODO}[2][]{\todo[color=red!10,inline,#1]{#2}}
\newcommand{\GE}{\TODO{Grammar}}
\newcommand{\SE}{\TODO{Spelling}}
\newcommand{\TE}{\TODO{Term}}
\newcommand{\CE}{\TODO{Citation}}
\newcommand{\DE}{\TODO{Corrected}}
\journal{i524}

\title{Couchbase Server: A Usable Overview}

\author[1]{Matthew Lawson}

\affil[1]{School of Informatics and Computing, Bloomington, IN 47408, U.S.A.}

\affil[*]{Corresponding authors: laszewski@gmail.com}

\dates{Paper1, \today}

\ociscodes{Couchbbase, Memcache, CouchDB, Cloud, I524}

% replace this with your url in github/gitlab
\doi{\url{https://github.com/eunosm3/classes/blob/master/docs/source/format/report/report.pdf}}


\begin{abstract}
Couchbase, Inc. develops Couchbase Server (CBS), an open-source, document-oriented, NoSQL database. Couchbase targets situations requiring high availability and high throughput of large amounts of data, i.e., big data.  CBS integrates Couchstore, Memcache and ForestDB, as well as a host of maintenance, administration \SE and querying tools, in order to attempt to meet its promises to its users.  The vast majority of the information comes from Couchbase itself via its developers website.  CBS appears to be a viable option in the burgeoning NoSQL database industry.
\newline
\end{abstract}

\setboolean{displaycopyright}{true}

\begin{document}

\maketitle

\TODO{This review document is provided for you to achieve your
  best. We have listed a number of obvious opportunities for
  improvement. When improving it, please keep this copy untouched and
  instead focus on improving report.tex. The review does not include
  all possible improvement suggestions and if you sea comment you may
  want to check if this comment applies elsewhere in the document.}
\TODO{As a technology review, the abstract should be the abstract of your
whole paper. Treat your abstract as mini-version of your paper.
 The first part looks fine, but the last two
sentences of the abstract are totally inappropriate with the abstract.}

\section{Introduction}

Couchbase, Inc. offers Couchbase Server (CBS) to the marketplace as its entry in the NoSQL, \textit{big data} database field \CE. \TODO{When you mention a technology for the first time in a paper, that's where you should cite.}  Salient features include a) an integrated cache tier which is essental \SE to the product's operation; b) persistent storage in JSON document format, i.e. document-based storage, or simple key-value pairs; c) relatively uncomplicated scalability across clusters of commodity servers; d) sub-millisecond response times; e) a SQL-like query language; and, f) built-in cluster replication, failover and disaster recovery features. In addition, Coucbhase markets a mobile product, Couchbase Mobile, which uses a Couchbase-designed syncing system to extend CBS to mobile devices and offline use cases.

Couchbase, headquartered in Mountain View, CA, began existance \SE in February 2011 when Membase, Inc. and CouchOne, Inc. merged into a single company.
\TODO {I don't see the point putting company history in a technology review.}
\section{Architecture, even in the introduction.}

A Couchbase Server (CBS) system consists of at
least one cluster of interconnected server computers \TE
\TODO{weird, server computers?}
running a copy of CBS. By
default, the CBS system computers \TE \TODO{same here, CBS system computers?},
referred to as nodes, work together in a
master-master setup, which Couchbase calls a peer-to-peer topology. \CE  In a
master-master distributed cluster, the nodes co-exist in flat heirarchy \SE, i.e.,
no node acts as the central authority.  This paradigm contrasts with the master-
slave paradigm utilized by distributed NoSQL database competitor MongoDB \CE.
Despite the egalitarian nature of the cluster, the nodes still need to
coordinate activities.  Therefore, the nodes \textit{elect} a node to coordinate
cluster functions. If the node fails or is removed from the cluster, the
remaining  nodes elect a new \textit{orchestrator}.

In addition, the database adminstrator \SE can override the default peer-to-peer
topology by taking advantage of CBS' \textit{multi-dimensional scaling}.  This
functionality allows the adminstrator \SE to customize nodes to perform tasks for
which the node is best suited, e.g., memory-intensive processes or
I/O-intensive processes, etc.

\begin{figure}[htbp]
\centering
\fbox{\includegraphics[width=\linewidth]{images/multiDimScaling}}
\caption{Multidimensional Scaling with Couchbase \cite{www-components-cbsinc}}
\label{fig:multidim scaling}
\end{figure}
\TODO{Font size within the figure is too small to read.}
A complete CBS system physically consists of a) one or more server clusters
running the couchbase daemon; b) high-speed connections between the servers and
between the clusters; and, c) client computer applications utilizing memcached-
compatible SDKs. \CE

The main components of a Couchbase Server node consist of the following: a) the
cluster manager; b) the data service; c) the query service; and, d) the index
service, as well as the underlying managed cache and storage components.\cite
{www-components-cbsinc}

\paragraph{Cluster Manager} The Cluster Manager, which runs on every node,
manages each respective node's interaction and involvement with the other nodes
in the cluster.  The Cluster Manager configures and monitors the node,
determines the layout for CBS' primary services, e.g., Data, Query and Index
Services, controls data rebalancing amongst the cluster's nodes, gathers
operational statistics, determines the nodes's membership in a cluster,
authenticates connections to the cluster, responds to heartbeat requests and
repairs itself if possible.
\cite{www-clustermanager-cbsinc,www-components-cbsinc}

\paragraph{Data Service} The Data Service provides the core functionality of any type of database management service - data access.  CBS organizes documents, or items, into \textit{buckets} and \textit{vBuckets}.  A bucket conceptually maps to a RDBMS database object \CE.  Unlike a database object, CBS distributes a bucket evenly across the cluster's nodes.  CBS refers to the portions of a bucket on a single node as a vBucket, which conceptually resembles a RDBMS shard.  Buckets typically have 1,024 vBuckets, so a three-node cluster with one bucket would have 341 vBuckets on two nodes and 342 vBuckets one the last node.

The Data Service provides an API for creating,
retrieving, updating and deleting (CRUD) items in CBS.  It operates on items
with keys in buckets.\cite{www-architecture-cbsinc}

\paragraph{Indexes and Index Services} The index services create, maintain
and destroy primary and secondary indexes of a bucket's keys for three index
services.  Couchbase refers to its index services as \textit{Map-Reduce Views},
\textit{Spatial Views} and \textit{Global Secondary Indexes} or GSIs \CE.

Views, which represent CBS's first generation index service, exist within CBS'
Data Service.  Map-Reduce Views return, or emit, document attributes as View
keys after applying user-defined map-reduce functions to JSON documents. Spatial
Views act in a similar fashion, except they process geographic information and
emit geographic coordinates as View keys.\cite{www-queries-cbsinc}  Spatial
Views for geospatial data equate to Map-Reduce Views non-geospatial data.

In contrast, CBS' Index Service represents the software's next-
generation index, the GSIs.  Couchbase developed GSIs in conjunction with, and
in service of, its SQL-like query service, called N1QL (pronounced
\textit{nickel}). \CE

As a result of their respective historical development paths, choosing to use a
View or a GSI depends on the use case. For instance, Map-Reduce View indexes
allow users to create arbitrarily complex indexes for later use.  "[Map-Reduce]
[v]iews are typically useful for interactive reporting type queries where
complex data processing and custom data reshaping is necessary."  \CE Spatial View indexes allow users to create "multidimensional bounding box queries for location aware applications." \cite{www-viewsindexing-cbsinc}

Since Views, Map-Reduce or Spatial, exist as part of the Data Service, they are
partition-aligned with the core data distribution.  That is, CBS spreads Views
across the cluster roughly proportionate to the underlying data.  Therefore,
performance slows as the number of nodes contacted increases due to network
processing needs.

In contrast, Couchbase constrains a GSI to residence on a single node, i.e.,
\textit{not} partition-aligned.  This design allows GSIs to return results
faster than Views.  However, GSIs can handle only relatively simpler queries.
In addition, users must manually create identical GSIs in order to use
the index on multiple nodes for concurrent searches, or as a backup option.

Finally, a CBS system's \textit{primary index} holds information for all of the data
in a bucket, while its \textit{secondary index} holds data for a pre-specified
subset of the data. Couchbase encourages the use of secondary indexes since they
avoid scanning the contents of an entire bucket index.

\paragraph{Query Service} Couchbase Server provides four methods of querying
the data.  First, users can take advantage of the Data Service's key-value API.
This method returns results faster the other methods, but it requires the user
to know the item's key.  The second and third methods complete query execution
by accessing the Views API.  Such queries operate on the map-reduce or spatial
Views keys.  These two methods provide the greatest query flexibility, including
data reshaping, at the cost of increased elapsed query time.  The fourth method
provides query flexibility and speed between the key- value \SE API and the Views
API.  Couchbase calls its newest method N1QL (pronounced \textit{nickel})
\TODO{mentioned in previous paragraph}in
homage to its SQL inspiration. Although the company designed GSIs for use by
N1QL, it can complete ad-hoc queries, i.e., queries without a pre-defined index.
It can also exploit View indexes in a limited fashion.

\paragraph{Managed Cache} "Since Couchbase built Couchbase Server on a
memory-first architecture, achieving high performance and scalability requires
effective memory management."\cite{www-cachinglayer-cbsinc}  CBS stores
frequently accessed data items, such as documents and indexes, in its integrated
cache tier.  Couchbase opted for this setup as a method to provide high-
performance, i.e., as fast as volatile memory allows, reads, writes and queries.
CBS monitors the frequency with which users access items in order to determine
which items to retain in cache and which items to write to disk.  The various
CBS services, e.g., Data Service, manage their respective cache usage to
optimize their respective tasks. In addition, CBS administrators can allocate
certain amounts of cache space by changing the system's Ram Quotas.
\cite{www-cachinglayer-cbsinc}

\paragraph{Storage Components} CBS utilizes two distinct storage engines,
namely, Couchstore and ForestDB.  Couchstore supports the Data Service, and, by
extension, the View index service.  It uses a B+tree structure for key-based
access.  It also captures changes to items via an append-only write model. In
contrast, ForestDB uses a B+trie structure for key-based access.  "B+trie
provides a more efficient tree structure compared to B+trees and ensures a
shallower tree hierarchy to better scale large item counts and very large index
keys." \cite {www-bplustrie-cbsinc}.
\TODO{This is very confusing since the difference of B+trie and B+tree is not explain at all.}
ForestDB defaults to using an append-only
write model, but can also utilize a "circular-reuse" model.  The latter takes
advantage of orphaned space the former ignores, thus reducing the frequency of
compaction.
\TODO{After this paragraph, I still don't understand the difference between
two engines rather than two different tree-structure and models they use.
It would be much better if you can elaborate on the models and perhaps B+trie
since B+trie is something very new and unique. Don't just stuff a collection of
jargons without explaining.}

\paragraph{Cross Data Center Replication Service [XDCR]}
Couchbase created a service for CBS, \textit{Cross Data Center Replication} or XDCR, \CE to enhance data availability and disaster recovery.  XDCR syncs data between separate CBS clusters, which can co-exist within a single data center or can reside in entirely separate geographies.  Besides data replication for disaster recovery, XDCR can be configured to immediately take over for a failed primary cluster.  In addition, XDCR can reduce latency by moving the data closer to the end user.  Companies using CBS can target "external applications (e.g. Elastic, Spark, Storm, etc.)."\cite{www-xdcr-cbsinc}

\section{User Interfaces}
\paragraph{API}
Client applications interact with CBS through memcached-compatible SDKs,
\TODO{What is a SDK?} which
support numerous programming languages.  As of version 4.6, developers could
choose from an SDK for Node.js, Java, PHP, .NET, Python, Go and C.
\cite{www-sdklist-cbsinc}.
\TODO{I know these are languages, but do others also know about them? What if
the reader is not from Computer Science at all?}  Couchbase also provides a client library for JDBC/ODBC \CE.\cite{www-downloads-cbsinc}

\paragraph{Shell Access} Couchbase offers a variety of command line tools.
The \textit{cbc} tool operates on a node, a bucket or a vBucket (shard).
\TODO{what is cbc? If you don't want to explain it, you shouldn't mention it.}
  It includes commands to create, retrieve or remove documents in a
CBS system, list the buckets in a cluster, manage users, etc.  In
addition, each CBS installation includes the \textit{cbq} tool to issue N1QL
queries. \cite{www-cli-cbsinc}.
\TODO{what is cbq?} Other CLI interfaces \TE
\TODO{CLI, command line interface. No one says command line interface interfaces.} listed as tools versus
commands of cbc include a) cbbackupmgr; b) cbanalyze-core, used to parse and
analyze a core dump; c) cbbackup; d) cbbackupwrapper and cbrestorewrapper,
which improve throughput for the backup process; e) cbcollect\_info, detailed
node-specific statistics; f) cbdocloader for loading a group of JSON document
into a bucket; g) cbepctl for managing vBuckets; h) cbft-bleve-dump, a
troubleshooting tool that prints all the rows in a specified index; i) cbft-
bleve-query, a troubleshooting tool to run queries on a bleve index; j)
cbreset\_password; k) cbrestore; l) cbstats, lower-level statistics from within
a cluster; m) cbworkloadgen, a tool for assessing read / write performance; and, n) mctimings, which provides "end-to-end timing information for all operations." \cite{www-cli-intro-cbsinc}

\TODO{I don't see the point of making a long list of interfaces.
What are you trying to do here? Are you trying to show the richness of
command line tools/interfaces? You don't have to make a list of it if that's
the case.}
\paragraph{Graphical Interface} Couchbase implements CBS' gui \TE
\TODO{GUI, graphical user interface, needs to be in capitals.} via a web
browser.  Users access the web gui by navigating to a cluster's url appended
with port number 8091, the admin port.  The browser interface acts as the
primary management tool for CBS.
\TODO{Why are you put the actual port number in your paper? Are you trying to
create a security breach?} It offers access to node management, queries,
indexes, etc.  At login, the console displays the Overview page.
\cite{www-webconsole-cbsinc}

\begin{figure}[htbp]
\centering
\fbox{\includegraphics[width=\linewidth]{images/webGuiOverview}}
\caption{Couchbase Web Console Overview Page \cite{www-webconsole-cbsinc}}
\label{fig:web gui overview}
\end{figure}
\TODO{The figure is confusing here. I thought you were at least going to
elaborate the overview page right after the figure, since it's part of the GUI.
But you showed it in the paper and that's it? So within your GUI paragraph,
you only describe that CBS's GUI is a web page, period. Nothing else.}
\section{Licensing}

Couchbase, Inc. offers a community edition of Couchbase
Server as well as an enterprise edition.  Couchbase Server Enterprise Edition
includes more features and better quality assurances, e.g., testing and bug
fixes, versus Couchbase Server Community Edition.  Couchbase targets
"enterprise customers with large production deployments running in data centers
and/or public clouds" \CE with the Server Edition. The remaining, primary differentiating factor of the Enterprise Edition over the Community Edition consists of Couchbase's 24x7 technical support. Community Edition users must rely on published material and the online CBS community forum instead of dedicated technical support.  \cite{www-downloads-cbsinc}

\section{Ecosystem} CBS does not have a large ecosystem built around it,
but Couchbase has developed a number of interfaces to software often used in conjunction with large data sets.  The
company offers the aforementioned client librariers, e.g., .NET, node.js, et al, as well as connectors and plugins for a) Spring Framework (connector); b) Spark (connector); c) Kafka (connector); d) Hadoop Sqoop (plugin); e) ElasticSearch (plugin); and, f) Solr LucidWorks Fusion (unspecified).  Couchbase also maintains Moxi Server, a proxy for memcached traffic.\cite{www-downloads-cbsinc}

\section{Use Cases}
\paragraph{General}
Use cases include a) supporting / enabling real-time analytics; b) building mobile apps with offline support via Couchbase Lite; c) digital communication by enabling low-latency read / write access to messages; and, d) purportedly holistic views of client data via aggregation from multiple sources even when the sources have different data models.

\paragraph{Use Cases for Big Data} Couchbase markets CBS to customers who
desire high throughput / low latency response times from a so-called schema-less
database managing data at scale, i.e., \textit{big data}. In the context of
NoSQL, big data databases, low latency translates to sub-millisecond response
times.  Other aspects of competitive products in this space include scalability,
a flexible data model (as implied by the NoSQL tag), a SQL-like query language
and simple administration.\cite{www-cbsintro-cbsinc}

The company highlights a number of real-world business wins to support its
assertions that CBS meets these criteria.

\paragraph{Equifax, Inc.}
For instance, Equifax chose Couchbase
Server Enterprise Edition when it needed to meet a new customer need in a short
amount of time.  In October 2015, the Federal National Mortgage Association, a
government-sponsored entity (GSE) more commonly referred to as Fannie Mae,
announced it would begin providing 24 months of trended credit history on its
industry-standard \textit{Desktop Underwriter} software instead of a point-in-time snapshot.  Fannie Mae promised this change by the end of the second quarter of 2016.  Therefore, Equifax had less than three calendar quarters to scale up its trended data product for a customer that underwrote nearly 46\% of all US residential mortgages at the time, when combined with its GSE-twin, Freddie Mac.
\cite{www-trendeddata-equifax,www-gsemktshare-valuewalk}

Equifax needed a solution to handle the five petabytes (5Pb) of data plus the
necessary throughput associated with trended data.  In addition, it needed a)
its new software to work with systems the company already used, like Hadoop and
Spark; b) it needed the solution to facilitate application development; and, c)
it needed five millisecond (5ms) response times.  CBS met those requirements for
Fannie Mae.  The mortgage underwriting GSE also found the ease of data
replication offered by CBS' XDCR attractive, as well as the minimal Java coding
needed to make CBS' Views useful to its operations teams.
\cite{www-fanniemae-equifax-couchbaseconnect}

\paragraph{LinkedIn Corp.}
LinkedIn also opted for Couchbase as its data management needs grew.  More
specifically, the challenges of moving data across its hosts / clusters with its prior Memcache-only design prompted it to consider other solutions.  The company currently utilizes CBS as a) a simple read-through cache; b) an ephemeral counter store, i.e., storage for temporary IDs; c) a temporary de-duplication store; and, d) a \textit{source of truth} for internal tooling.  LinkedIn's data expands across 148 buckets and 2,821 hosts.  The largest cluster by nodes consists of 72 hosts, while the largest cluster by documents holds 1.4 billion items.  Overall, its CBS system handles 10 million-plus queries per second (QPS). \cite{www-linkedin-couchbaseconnect}

\section{Educational material}

If you would like to learn more about Couchbase Server, visting developer.couchbase.com or the Couchbase Connect section of Couchbase's youtube.com channel should prove beneficial.  In addition, perusing the works cited in the reference section may also prove beneficial.

\section{Conclusion}

Couchbase Server appears to offer the necessary features to succeed commercially as a \textit{big data} database.  That is, it scales well, it handles extremely large datasets well, it handles high-throughput transactions well and it has a SQL-like query interface.  Whether or not CBS will succeed due to technical superiority, administrative ease or becasue \SE Couchbase simply marketed better than the competition exceeds the scope of this write-up, though.  Based on the feature set and the business wins, it appears to be a legitimate option for orgnaizations \SE interested in this type of general product.

\section{Acknowledgement}

I would like to thank Dr. Gregor von Laszewski, the TAs for I524, Big Data Software and Projects in the Cloud and the other students in the class for their insights and assistance related to this paper.

I would also like to thank my employer, Indiana Farm Bureau, which funded this research, in part, via its employee education assistance program.

% Bibliography

\bibliography{references}

\section*{Author Biographies}
\begingroup
\setlength\intextsep{0pt}
\begin{minipage}[t][3.2cm][t]{1.0\columnwidth} % Adjust height [3.2cm] as required for separation of bio photos.
  \noindent
  {\bfseries Matthew Lawson} received his BSBA, Finance in 1999 from
  the University of Tennessee, Knoxville. His research interests include
  data analysis, visualization and behavioral finance.
\end{minipage}
\endgroup

% \newpage

\appendix

\section{Work Breakdown}

The work on this project was distributed as follows between the
authors:

\begin{description}
\item[Matthew Lawson.] Researched Couchbase Server and related topics, wrote the paper and edited the paper.
\end{description}


\end{document}
