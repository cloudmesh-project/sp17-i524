\documentclass[9pt,twocolumn,twoside]{../../styles/osajnl}
\usepackage{fancyvrb}
\journal{i524} 

\title{CUBRID RDBMS}

\author[1]{Abhijit Thakre}

\affil[1]{School of Informatics and Computing, Bloomington, IN 47408, U.S.A.}
\affil[2]{Mechanical Engineer,Nagpur University, 2003}

\affil[*]{Corresponding authors: abhijit.thakre@gmail.com}

\dates{project-000, \today}

\ociscodes{Cloud, I524}

% replace this with your url in github/gitlab
\doi{\url{https://github.com/cloudmesh/sp17-i524/blob/master/paper1/S17-IO-3021/report.pdf}}

\begin{abstract}

With advanced techniques of data mining and analysis, bigdata
processing has become a key in today’s world.  Many of the bigdata
processing uses NOSQL data for storing. However in order to avail the
ACID behavior of database, the focus is again back to the RDBMS
databases. This paper focuses on one the similar ORDBMS CUBRID. It
also highlight the architecture of CUBRID with it key component and
features provided.
  
\end{abstract}


\setboolean{displaycopyright}{true}

\begin{document}

\maketitle

\section{Introduction}

CUBRID is open source RDBMS with object support developed by Navel
corporation. Developed in C language CUBRID provides key features like
scalability, high availability, higher performance, online and
incremental backups. CUBRID is distributed under GNU general public
license for the database server engine and BCD license for API and
client tool.

\section{Architecture}

CUBRID has distinguished 3 layer architecture.  It consists of
Database server, connection broker and application layer.

\subsection{Database Server}

It is the core component of the CUBRID Database Management
System. The main function for the database server are as below
\begin{itemize}
\item Saving and managing the data.
\item Processing of the queries from user.
\item Providing smooth functioning for multiple users.
\end{itemize}
\subsection{Broker}

It acts as middleware between the Database Server and GUI application
to provide seamless experience.
It provides functions
\begin{itemize}
\item Connection pooling.
\item Monitoring.
\item Log tracing and analysis.
\end{itemize}
\subsection{CUBRID Manager}

 It is a GUI tool that manages database and broker. It also provides
the Query Editor for executing queries on database for users.

\section{Architecture Diagram}

\begin{figure}[htbp]
\centering
\fbox{\includegraphics[width=\linewidth]{images/cubrid_architecture}}
\caption{\cite{www-cubrid.org}}
\label{Reference:false-color}
\end{figure}

\newpage

\section{Description}

Database server and the broker work in co-ordination with each other
as server and client respectively. The can be deployed on the
different or same machine. The broker takes care of the queries from
the users, it process it to the optimum level. On optimization it
creates a query plan tree and sends the request to the server. The
response from the server is via cursor navigation which is further
returned to the user.

The client caches object instances from the
database to its memory to provide fast access to data by using the
query execution results or directly by users/applications. In
addition, it caches locks as well as objects from the server for
concurrency control. The execution of triggers or methods specified by
users or applications is also performed in the client module.

References: \cite{www-cubrid.org}


\section{Module Configuration}

The CUBRID client and server modules consist of the following components:

\begin{itemize}
\item Transaction Management Component.
\item Server Storage Management Component.
\item Client Storage Management Component.
\item Object Management Component.
\item Client-Server Communications.
\item Thread Management.
\item Query Processing.
\end{itemize}

\section{Authentication in CUBRID}


CUBRID provides two levels of authentication.

User needs to enter credentials to login to the Host Server. On first
login user need to set the admin credentials.

User needs to login to each database in the host server to access the
individual database.

References: \cite{www-authentication}


\section{Key Features}

\subsection{High Availability and Scalability}

CUBRID uses heartbeat technology to
provide automated accurate fail-over and fail-back features which
makes the database continuously available. The server is available
during the upgrades, replacement or even during the maintenance phase.

\subsection{Database Sharding}

CUBRID 8.4.3 provides free sharding feature where data can be divided
on multiple instance.  In addition to unlimited sharding it also
provides the features like connection pooling and load balancing to
all the shards.

\subsection{Performance}

CUBRID provides high performance to the users with feature like query
caching, optimized algorithm for indexing, fast object access.

CUBRID performance for hotspot read and special character for web service using 
logging technique have applied to disk have resulted in increase in the overall 
performance. 

Function based indexing, filtering indexing and index skip scan
provides various features to user for increasing the performance

\subsection{Reliability}

CUBRID is highly reliable with features like online incremental backup
and restore.  It provides the access restriction based on userip and
databaseid. Naver is one of the top web portal in Korea rely on 
cubrid database.

\subsection{Language Support}
It provides 90 percent support to sql language support. 


\section{CUBRID SHARD}

CUBRID shard is sharding is RDBMS specially targeted to address the
problem on processing bigdata. CUBRID shard distributes the user data
on multiple server to store it. So for fetching the data specific to
user it needs to pass key information about the shard in the
query. Parsing the query and finding the shard both things are taken
care by the broker and does not needs the additional layer. This helps
in increasing the performance for big data. 


\section{Performance Benchmarking}

\subsection{CUBRID Vs MySQL}

Test was performed by one of the social networking site on new 8.4.0 
version of cubrid database to understand the time taken by cubrid database 
for managing the queries of 6 million active users. The queries mostly involved 
the IN and UNION operator.  

For executing the test both Mysql and CUBRID database was made to store 3 user
group table with 1520000 rows each. Each of the db had given 40 thread load for 
10 min. The systems used for both the database are of same configuration.The 
test was conducted for five different scenarios.


\subsection{Analysis Results}

In most realistic scenario CUBRID IN query have performed twice as fast as Mysql
UNION query and 8 times as fast as the Mysql IN query. 

It also came out from the test that the effect of Key limit have considerably decreased
the IO operations.


\subsection{NBD BenchMark}
The NBD (NHN Internet Bulletin Board Application Database) Benchmark is important benchmark to 
measure of performance of a DBMS used for a Bulletin Board System (BBS) type of web services.

This benchmark test was conducted to compare the performance of CUBRID R1.1 DBMS with  OSS DBMS 
D1, Commercial DBMS D1, and Commercial DBMS D2. Factors considered for the benchmark includes 
database size, workload increase and cache functionality.


\subsection{Conclusion of NBD Benchmark}

In the small and medium-sized database benchmarking, CUBRID is at second place after commercial DBMS D1. 
In the small database benchmarking, CUBRID performance can be improved if provided more work load. For the 
given workload it showed the usage of 30\% CPU.

For medium size however it showed 80\% of CPU usage.
DBMS D2 shows the highest CPU usage of 100\% in both small and medium-sized databases. References: \cite{www-benchmark}


\newpage

% Bibliography

\bibliography{references}
 
\section*{Author Biographies}
\begingroup
\setlength\intextsep{0pt}
\begin{minipage}[t][3.2cm][t]{1.0\columnwidth} % Adjust height [3.2cm] as required for separation of bio photos.
  \begin{wrapfigure}{L}{0.25\columnwidth}
    \includegraphics[width=0.25\columnwidth]{images/john_smith.eps}
  \end{wrapfigure}
  \noindent
  {\bfseries Abhijit Thakre} received his BE (Mechanical) in 2003 from
  The University of Nagpur.
\end{minipage}
\endgroup


\appendix

\end{document}
