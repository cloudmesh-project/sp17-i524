\documentclass[9pt,twocolumn,twoside]{../../styles/osajnl}
\usepackage{fancyvrb}
\journal{i524} 

\title{Flight Data Analysis Using Big Data Tools}

\author[1,*]{Anvesh Nayan Lingampalli}

\affil[1]{School of Informatics and Computing, Bloomington, IN 47408, U.S.A.}

\affil[*]{Corresponding authors:anveling@indiana.edu}

\dates{project-S17-IR-2016, \today}

\ociscodes{Apache, Hive, Ansible, Pig}

% replace this with your url in github/gitlab
\doi{\url{https://github.com/cloudmesh/classes/blob/master/project/S17-IR-2016/report/report.pdf}}


\begin{abstract}
Analysis of flight data provides insights on the United States of
America's Airline data by using Hadoop in the cloud environment. The
On-time performance of flights operated by large air carriers are
tracked and made as a report, Air Travel Consumer Report, which is a
big data set. Hive component of Hadoop ecosystem, is utilized to
process the big data in distributed environment. Efficient accessing
and processing of the user queries is acheived by this analysis on
flight data.
\newline
\end{abstract}

\setboolean{displaycopyright}{true}

\begin{document}

\maketitle

\section{Introduction}

Real world data is large and growing exponentially from several
years. Data with structured and unstructured format is Big
Data. Aviation industry manages enormous amount of data, which
consists of the information regarding the delayed, cancelled, diverted
or on-time flights by large air-carriers\cite{aviationanalysis}. This
is essentially a big-data set, where statistics are publicly available
in the Air Travel Consumer Report. Big Data analysis of this data will
provide a consistent understanding and importance of the given
data. With 35 million flight departures per year, data is critically
important for any planning decision made by airlines and airports. The
results of analysis has benefits which can help airline operations to
predict and reduce redundance\cite{bigdatainaviation}.

Hive is one of the ecosystems in Hadoop framework which is built to
analyze the data on hadoop cluster. Syntax of Hive is based on SQL,
which is also known as HiveQL. MySQL or PostGreSQL can be used for
implementation of the queries. Hive provides tools which enable easy
data extraction, transformation and data loading. Files can be stored
in Hadoop Distributed File System(HDFS) and accessed by Hive
efficiently.

Schema of the Hive tables is stored in Hive Metastore. Metastore holds
the information about tabes and partitions which are present in the
data warehouse. In Hive the default Metastore is DerBy Database, which
is a relational database management system provided by Apache
Software.  There are two components of Hive, HCatalog and
WebHCat. HCatalog is a storage management layer for Hadoop which
provides data processing tools such as Pig and MapReduce. WebHCat
provides a service that is used to run Hadoop MapReduce, Pig, Hive
jobs using REST interface.

Hive's SQL provides the basic SQL operations, such as,
\begin{itemize}
\item Filtering the rows from the table using WHERE clause.
\item Selecting columns from table using SELECT clause.
\item Joining two tables
\item Aggregations of the data using 'group by' clause.
\item Storing the results in a hdfs directory.
\end{itemize}
  
\section{Implementation}

Implementation of Hive to perform data analysis consists of the
following steps
\section{Performing Analysis on local Virtual Machine}
\subsection{Install Hadoop environment}

\subsection{Loading Data into HDFS}
   Data is stored into HDFS by, 
\subsection{This data into Hive tables}
\subsection{Queries in Hive Query Language}
  The following are the queries
  in Hive QL, which are similar to SQL statements.
 
  What are the total number of flights which are cancelled?
  What are the total number of flights which are diverted?
  What is effect of flight distance on cancellations?
  What is the effect of flight distance on average departure delay?
  What is the monthly average departure delay?
  What is the yearly average departure delay?
  
\item Benchmarking

\end{itemize}


\section{Technologies}

\begin{itemize}
\item Distributed Computation and Storage:- HDFS, Hive and Pig
\item Development:- Python and Java
\item Deployment:- Ansible
\end{itemize}

\section{Deployment}
Ansible Playbook is used as the application and configuration
deployment tool. Deploying the Hive and Pig framework into the
cluster environment. 

\section{Benchmarking}
TBD

\bibliography{references}

\end{document}
