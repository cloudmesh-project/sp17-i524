\documentclass[9pt,twocolumn,twoside]{styles/osajnl}
\usepackage{fancyvrb}
\journal{i524} 


\title{Project Report: Detecting Street Signs in Videos in a Robot Swarm}

\author[1,*]{Rahul Raghatate}
\author[1,*]{Snehal Chemburkar}

\affil[1]{School of Informatics and Computing, Bloomington, IN 47408, U.S.A.}

\affil[*]{Corresponding authors: rraghata@iu.edu, snehchem@iu.edu}

\dates{project-001, \today}

\ociscodes{Street Signs, Video Streams,  OpenCV, Spark, Cloud, I524}


\doi{\url{https://github.com/rahulraghatate/sp17-i524/project/S17-IR-P003/report/report.pdf}}

\begin{abstract}
The aim of this project is to deploy a software package to detect different street signs in a video stream. This will be a scalable system over hadoop based cloud ecosystem to incorporate multiple video feeds and parallel real-time processing of the feeds. A comparative benchmark will be developed based on the performance of package on multiple cloud systems.\newline
\end{abstract}

\setboolean{displaycopyright}{true}

\begin{document}

\maketitle

\section{Introduction}
Detecting objects in images has always been keen area of interest in the field of computer vision.There are many applications developed based on this simple idea like auto tagging pictures (e.g. Facebook, Phototime), counting the number of people in a street(e.g. Placemeter), classifying pictures, detecting vehicles,etc.On the similar grounds,we are building a software package which can be deployed easily on cloud  infrastructure and establish a platform to  detect different street signs in a video stream. A benchmark will be developed based on performance of this software on different cloud systems. The database of street signs will be restricted to US street signs. The video streams used for this project are simulated or captured using mobile camera. 

\section{Requirement Analysis}
We are using following technologies  for complete project development and deployment:
\begin{enumerate}
\item Cloudmesh - For connecting to different cloud environments. 
\item Ansible -For deploying software from master node on virtual slave nodes.
\item Python - writing script for Data Analysis and Data Processing in Spark engine over hadoop.

\item Hadoop -Required for uploading our dataset of images on Distributed Data Storage Platform as well as for Video Streams and its processing in Spark Stream. Using prei-build Hadoop and Spark in cloudmesh so as to focus on video data distribution and analysis and perform optimization testing on cluster.

\item OpenCV -Perform Video Analysis for Street Signs Detection using open source computer vision Libraries of Video/Image Analysis algorithms.The OpenCV library provides several features to manipulate images(apply filters, transformation), detect and recognize objects in images.
\end{enumerate}

\section{Methodology}
\begin{enumerate}
\item Data Gathering for street signs and video streams
\item Deploy Hadoop clusters on cloud using Cloudmesh
\item Develop Ansible script to install OpenCV on cloud
\item Build a model to detect or track street signs using OpenCV.We are planning on implementing two programs-
    \begin{itemize}
    \item Read an image and run the Haar cascade classifier to detect the signs in the image and 
    \item use the video stream and detect signs in real time.
    \end{itemize}
\item To detect street signs, we will be using the Haar Feature-based Cascade Classifier which detect objects in an image. As detecting signs and recognizing them(categories?) are two different problems and use two different approaches. Hence, we will benchmark detection first and will work on recognition as future development.
\item Test the performance of softwarepackage on 3 different clouds or on the same cluster with multiple nodes.
\item Create benchmarks based on the above results

\end{enumerate}


\section{Execution Summary}
This section specifies the week by week timeline for project completion.
\begin{enumerate}
\item {Mar 6 - Mar 12, 2017}
 Create virtual machines on Chameleon cloud using Cloudmesh and submit Project Proposal.
\item {Mar 13-Mar 19, 2017}
 Deploy Hadoop cluster to Chameleon cloud using Cloudmesh and develop Ansible playbook to install the required software packages to the clusters (OpenCV, etc)
\item {Mar 20-Mar 26, 2017}
Train data to detect or track street signs using OpenCV.Develop Ansible playbook to setup database and connectivity among multiple nodes.
\item {Mar 27-Apr 02, 2017}
Capture the results of street sign tracking in video streams.
\item {Apr 03-Apr 09, 2017}
Test on different cloud systems and define benchmarks.
\item {Apr 10 - Apr 16, 2017}
Create deployable software package in Python.
\item {Apr 17-Apr 23, 2017}
Write Project Report  
\end{enumerate}

\section{Use Cases}
\begin{enumerate}
\item Street Sign Detection for autonomous vehicles.
\item Analysis of traffic signs in google street view to estimate all signs ahead hence,useful in ambulance , fire brigade services, simplest path finder etc.
\end{enumerate}

\section{Benchmark}
Benchmarks will be created based on the performance of the software in different cloud environments. The initial analysis will be done on a single short video stream and then on video streams distributed across 2 or 3 nodes. The different cloud systems used for the purpose of benchmarking are Chameleon, FutureSystems and JetStream.

\section*{Acknowledgements}

% Bibliography
\noindent Add citations with the cite command. See
\cite{las14cloudmeshmultiple} for an example on how to use multiple
clouds. In \cite{www-i524} we list the class content.

Here a test of a citation with an underscore in the url \cite{www-underscore}.

\bibliography{references}
 

\section*{Author Biographies}
\begingroup
\setlength\intextsep{0pt}
\begin{minipage}[t][3.2cm][t]{1.0\columnwidth}
% Adjust height [3.2cm] as required for separation of bio photos.
  \begin{wrapfigure}{L}{0.25\columnwidth}
    \includegraphics[width=0.25\columnwidth]{images/john_smith.eps}
  \end{wrapfigure}
  \noindent
  {\bfseries Rahul Raghatate} will receive his Masters (Data Science) in 2018 from
  The Indiana Univeristy Bloomington. His research interests include Big Data and Machine Learning. 
\end{minipage}
\begin{minipage}[t][3.2cm][t]{1.0\columnwidth} % Adjust height [3.2cm] as required for separation of bio photos.
  \begin{wrapfigure}{L}{0.25\columnwidth}
    \includegraphics[width=0.25\columnwidth]{images/alice_smith.eps}
  \end{wrapfigure}
  \noindent
  {\bfseries Snehal Chemburkar} will receive her Masters (Data Science) in 2018 from
  Indiana University Bloomington. Her research interests also include
  Big Data and Machine Learning. 
\end{minipage}
\endgroup

\appendix

\section{Work Breakdown}
Will be updated in later phases of project.
\section{following topics are yet to be included}

\paragraph{2.2. Shell Access}

If applicable comment on how the tool can be used from the command line

\paragraph{3. Licensing}

Often tools may have different versions, some fre, some for
pay. Comment on this. For example while a tool may offer a comercial
version this version may be too costly for others. Identify especially
the difference between features for free vs commercial tools.

Sometimes you may need to introduce this also in the introduction as
there may be a big difference and without the knowledge you do not
provide the user an adequate introduction.

\paragraph{4. Ecosystem}

Some technologies have a large ecosystem developed around them with
extensions plugins and other useful tools. Identify if they exists and
comment on what they can achieve

provide potentially a mindmap or a figure illustrating how the
technology fits in with other technologies if applicable.


\paragraph{5. Educational material}

Put information here how someone would find out more about the
technology. Use important material and do not list hundrets of web
pages, be selective.

\end{document}
