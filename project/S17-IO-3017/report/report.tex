\documentclass[9pt,twocolumn,twoside]{../../styles/osajnl}
\usepackage{fancyvrb}
\journal{i524} 

\title{Analysis of USGS Earthquake Data}

\author[1,*]{Nandita Sathe}

\affil[1]{School of Informatics and Computing, Bloomington, IN 47408, U.S.A.}

\affil[*]{Corresponding authors: nsathe@iu.edu}

\dates{project-001, \today}

\ociscodes{I524, geospatial, MongoDB, D3.js, Apache Spark, Python, USGS, Ansible}

% replace this with your url in github/gitlab
\doi{\url{https://github.com/nsathe/sp17-i524/blob/master/project/S17-IO-3017/report/report.pdf}}


\begin{abstract}
Geo-spatial data fits into definition of Big Data as it has all three 'V's viz. high-velocity, high-volume and high-variety. Big Data Analytics tools now allow us to analyze the huge volumes of geo-spatial data. Data of earthquakes that take place globally is a major part of crucial geo-spatial data. This application analyzes data related to earthquake which can be utilised in further research. US Geological Survey's (USGS) Earthquake Hazards Program monitor and report earthquakes, assess earthquake impacts and hazards, and research the causes and effects of earthquakes \cite{www-usgs1}.
\newline
\end{abstract}

\setboolean{displaycopyright}{true}

\begin{document}

\maketitle

\section{Introduction}

The USGS estimates that several million earthquakes occur in the world each year, although many go undetected because they occur in remote areas or have very small magnitudes \cite{www-usgs2}. Thus earthquakes pose significant risk globally to the mankind. USGS collects volumes of geospatial data pertaining to earthquakes and makes it available for analysis. This project intends to analyze this data. The application will be deployed on cloud. Deployment will be automated using Ansible. 

\section{Technologies Used}

Technologies used for development and deployment of this project are listed below.
\begin{enumerate}

\item Cloudmesh - For connecting to different cloud environments.
\item Ansible -For deploying software and associated packages.
\item Python - Writing script for data analysis and data processing 
\item Apache Spark - For data processing
\item Mongo-DB - For storing Geo-spatial data
\item D3.js - As a visualization tool
\end{enumerate}

\section{Execution Plan}

This is how I intend to execute the project on week-by-week basis. Although my intention is to follow the plan deligently, it is possible that because of technical and other un-foreseen challenges deadlines may be pushed ahead.

\begin{enumerate}

\item {\bfseries 6 Mar 2017 - 12 Mar 2017} Create virtual machines on Chameleon
cloud using Cloudmesh and submit the project proposal.

\item {\bfseries 13 Mar 2017 - 19 Mar 2017} Deploy Mongo DB to Chameleon
cloud using Cloudmesh and develop initial Ansible playbook to install the required software packages.

\item {\bfseries 20 Mar 2017 - 26 Mar 2017} Write script in Python for downloading USGS data at run-time. Write Python and Spark scripts for data analysis and processing.  

\item {\bfseries 27 Mar 2017 - 02 Apr 2017} Implement visualization using D3.js. Update Ansible playbook to install D3js package. 

\item {\bfseries 03 Apr 2017 - 09 Apr 2017} Test on different cloud systems. Define quantitative benchmarks. Tentatively benchmarks will be for data insertion time and data processing time.

\item {\bfseries 10 Apr 2017 - 16 Apr 2017} Create deployable software package
in Python.

\item {\bfseries 17 Apr 2017 - 23 Apr 2017} Update and finalize the Project Report
\end{enumerate}

\section{Benchmark}

As mentioned in Section 3, benchmarks are tentatively for data insertion time and data processing time on different clouds.  

\section {Acknowledgements}

This project is undertaken as part of the I524: Big Data and Open Source Software Projects course at Indiana University. The author would like to thank Prof. Gregor von Laszewski and his associates from the School of Informatics and Computing for providing all the technical support and assistance.

\section {Licensing}

TBD

\section {Conclusion}

TBD

% Bibliography

\bibliography{references}
 
\section*{Author Biographies}
\begingroup
\setlength\intextsep{0pt}
\begin{minipage}[t][3.2cm][t]{1.0\columnwidth} % Adjust height [3.2cm] as required for separation of bio photos.
  \begin{wrapfigure}{L}{0.25\columnwidth}
    \includegraphics[width=0.25\columnwidth]{images/john_smith.eps}
  \end{wrapfigure}
  \noindent
  {\bfseries Nandita Sathe} is PMP certified project manager by profession. She will obtain MS in Data Sciences from Indiana University in May 2018. Her interests are in data analytics and machine learning.
\end{minipage}
\endgroup

\newpage

\section{Work Breakdown}

The work on this project was distributed as follows between the
authors:

\begin{description}

\item[Nandita Sathe.] She completed all the work related to development of this   application including research, testing and writing the project report. 

\end{description}

\end{document}
