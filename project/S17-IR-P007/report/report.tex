\documentclass[9pt,twocolumn,twoside]{../../styles/osajnl}
\usepackage{fancyvrb}
\journal{i524} 

\title{Deploying a spam message detection application using R over Docker and Kubernetes}

\author[1,*]{Sagar Vora}
\author[1]{Rahul Singh}

\affil[1]{School of Informatics and Computing, Bloomington, IN 47408, U.S.A.}

\affil[*]{Corresponding authors: vorasagar7@gmail.com, rahul\textunderscore singh919@yahoo.com}

\dates{project-P007, \today}

\ociscodes{Docker, Ansible, Kubernetes, R, Pandas, Spam, <add spam detection algorithms>}

% replace this with your url in github/gitlab
\doi{\url{https://github.com/cloudmesh/sp17-i524/raw/master/project/S17-IR-P007/report/report.pdf}}

\begin{abstract}
 
In the last few decades, online spam has become one of the major problem for the sustainability of the Internet. Due to the excessive amount of spams, the quality of information available on the Internet has reduce drastically. Moreover spam messages are also creating problems among the various search engines available and the web users. This report aims at developing an application which would detect spam messages from actual meaningful messages using Pandas and R. For the purpose for parallelizing the process, we would deploy the application using Docker containers on the Kubernetes cluster using ansible scripts which would automate the deployment.
 

\end{abstract}

\setboolean{displaycopyright}{true}

\begin{document}

\maketitle

\section{Introduction}

Today, the Internet \cite{www-internet} has been adopted rapidly in
the day to day life of people. It has provided a platform for
information generation and consumption. Moreover, it is used on a
daily basis to search for information and acquire knowledge. The
online encyclopedia Wikipedia™ \cite{www-wikipedia} provides a good
example of a more socialized Internet because the content within
Wikipedia™ is collectively generated by its users, rather than
webmasters or designated editors. The ease with which content can be
generated and published has also made it easier to create spam. Spam
can be stated as any information which does not add value to a user of
the web. Messages which are inappropriate, unsolicited, repeated and
irrelevant can be all classified as spam.

\noindent
So in this report, we are provinding an application that would
identify valid messages and spam messages from a given dataset. For
spam detection, we are using various techniques like Bayesian, <text
here> ...... Moreover, deploying our application using Docker
\cite{www-docker-about} containers on the Kubernetes
\cite{www-kubernetes} cluster will give a distributed approach.This
would also speed up the process of identifying the spam messages. We
have also deployed it on different cloud environments like Chameleon,
JetStream, FutureSystem and have performed benchmark analysis of the
application. This would let us the time taken by the algorithm on
these cloud solutions.

\section{Software Stack}

\begin{figure}[ht]
\begin{center}
 \begin{tabular}{|c | c|} 
 \hline
Name of the Technology & Purpose in the Project \\ [0.5ex] 
 \hline\hline
    
R & data analytics \\
\hline

Docker & container for the application \\
\hline

Kubernetes & cluster creation and management \\[1ex]
\hline

Cloudmesh Client & An client application used to ssh in various cloud
environments \\[1ex]

\hline
Ansible & Automation language to deploy
application on the target machines \\[1ex]
\hline

\end{tabular}
\end{center}
  \caption{Technologies used in the Project}
\end{figure}

\section{Why Kubernetes}
<write why we choose Kubernetes>

\section{What is Kubernetes}
<what is Kubernetes>

\section{Architecture}
<Architecture of Kubernetes>

\section{Ansible}
No one likes repetitive tasks, so with Ansible \cite{www-ansible}, IT
admins can begin automating away the drudgery from their daily routine
tasks. Ansible is a simple automation language that can perfectly
describe an IT application infrastructure. Ansible is an open source
automation engine which can be used to automate cloud provisioning,
configuration management, and application deployment. It can also
perform more advanced IT tasks such as continuous deployment or
rolling out updates with zero downtime.

A major difference in Ansible and many other tools in the
space is its architecture.

\subsection{Architecture}
Ansible is an agentless tool,it doesn't requires any software to be
installed on the remote machines to make them manageable. By default
is manages remote machines over SSH or WinRM, which are natively
present on those platforms \cite{www-ansible-architecture}.

Like the other configuration management software, Ansible
distinguishes between two types of servers: one being the controlling
machines and other being the nodes. Ansible uses a single controlling
machine where the orchestration begins. Nodes are then controlled by a
controlling machine over SSH \cite{www-ssh}. The location of the nodes
are described by the inventory of the controlling machine.

Ansible modules are deployed by Ansible over SSH. These modules are
temporarily stored in the nodes and communicate with the controlling
machine through a JSON protocol over the standard output

\subsection{Playbooks}

Playbooks \cite{www-ansible-playbook} are Ansible’s configuration,
deployment, and orchestration language. They let us control the remote
systems with a policy which we might want them to enforce. If Ansible
modules act as tools in your workshop, then playbooks are your
instruction manuals, and your inventory of hosts are your raw
material. Playbooks can be used to manage configurations of and
deployments to remote machines. They can sequence multi-tier rollouts
involving rolling updates, and can delegate actions to other hosts,
interacting with monitoring servers and load balancers.

\subsection{Ansible Galaxy}





\subsection{Pandas}
Pandas is an opensource Python library that provides data analysis
functionality with Python. Python initially lacked data analysis and
modeling capability. Pandas filled out this gap by providing essential
analytic functions thus saving the need to switch to a more domain
specific language for data analysis.

\subsection{R}
R is a language and environment for statistical computing and graphics
\cite{www-about-rproject}. Pandas does not provide a significant
statistical modeling environment as it is still a work in progress. R
provides a variety of statistical model analysis, classification,
clustering and graphical techniques to provide this
environment. Integrating Python's efficiency with R's capability
allows us to build a highly a desirable analysis model for our
application.

\subsection{Docker}
Docker allows application developers to package their applications
into isolated containers. A container comprises of only the libraries
and settings that are required to make the software work.  Docker
automates the repetitive tasks of setting up and configuring
development environments thus allowing developers to focus only on
building software. A dockerized application
can simply ship between platforms as the complexity of software
depencies is handled by the container.

\subsection{Kubernetes}
Kubernetes is an open-source platform which helps
in automating deployment, scaling, and operations of application
containers across clusters of hosts. Kubernetes helps in faster
deployment of application and scaling them on the fly. Moreover it
optimizes the use of hardware by using the resources which are
needed. A Kubernetes cluster can be deployed on either physical or
virtual machines. We will be using Minikube which is a lightweight
Kubernetes implementation which creates a VM on the local machine and
deploys a simple cluster containing only one node. The Minikube CLI
provides basic bootstrapping operations for working with the cluster,
including start, stop, status, and delete commands.

\section{Design}
\subsection{Building the Classification model}

\subsubsection{CrossValidation for the training data}

To address the problem of incoming spam messages, a model shall be
developed using the Bayesian Classification technique to correctly
classify each incoming email/text message as a spam or a legitimate
one. The model aims at developing a message filter that shall
correctly classify messages based on word probabilities that are
extracted from the training dataset.The training dataset to build the
model consists of 5574 message records. Dataset taken from
\cite{www-sms-spam-dataset}. The training process shall use the
cross-validation feature provided by R to build the classification
model and use Bayes theorem of conditional probability to predict the
class of each incoming message.
< I just copy pasted the above paragrah here> Donno what will come here.

To develop an efficient training model, we shall partition the data
into 2 subsets - training data and classification data. We shall
choose one of the subsets for training and other for testing. In the
next iteration the roles of the subsets shall be reversed, i.e the
training data becomes the classification one and vice versa. This
operation shall be carried out until each individual record is used
both as a classification and training record. We shall use the cross
validation feature provided by R for this subsampling. This
subsampling technique handles the underfitting problem and guarantees
an effective classification model.


\subsubsection{Training process}
Content of each of the spam marked messages shall be processed through
Naive Bayes Classifier.  The classifier shall maintain a bag of words
along with the count of each word occuring in the spam messages. This
word count shall be used to calculate and store the word probability
in a table that shall be cross-referenced to determine the class of
the record on classification
data \cite{paper-classification-of-email}.

A selected few words have more probability of occuring in a spam
messages than in the legitimate ones. Eg: The word "Lottery" shall be
encountered more often in a spam message.  The classifier shall
correlate the bag of words with spam and non-spam messages and then
use Bayes Theorem to calculate a probability score that shall indicate
whether a message is a spam or not. The results shall be verified with
the results available on the training dataset and the classifier
accuracy shall be calculated.  The classifier shall use the Bayesian
theorem over the training dataset to calculate probabilities of such
words that occur more often in spam messages and later use a summation
of scores of the occurence of these word probabilities to estimate
whether a message shall be classified as spam or not. After working on
several samples of the training dataset, the classifier shall have
learned a high probability for spam based words whereas, words in
legitimate message like family member or friends names shall have a
very low probability of occurence.

\subsection{Classifying new data}

Once the training process has been completed, the posterior
probability for all the words in the new input email is computed using
Bayes theorem. A threshold value shall be defined to classify a
message into either class. A message's spam probability is computed
over all words in its body and if the sum total of the probabilities
exceeds the predefined threshold, the filter shall mark the message as
a spam \cite{www-wiki-naivebayes}.

A higher filtering accuracy shall be achieved through filtering by
looking at the message header i.e the sender's number/name. Thereby if
a message from a particular sender is repeatedly marked as spam by the
user, the classifier need not evaluate the message body if it is from
the same sender.


\section{Discussion}
TBD


\section{Deployment}
Our application will be deployed using Ansible \cite{www-ansible}
playbook. Automated deployment should happen on two or more nodes
clouds or on multiple clusters of a single cloud. Deployment script
should install all necessary software along with the project code to
Kubernetes cluster nodes using the Docker image.

\section{Conclusion}

TBD

\section{Acknowledgement}

We acknowledge our professor Gregor von Laszewski and all associate
instructors for helping us and guiding us throughout this project.

\section{Appendices}
TBD

% Bibliography

\bibliography{references} 

\end{document}
