\documentclass[9pt,twocolumn,twoside]{../../styles/osajnl}
\usepackage{fancyvrb}
\usepackage{comment}
\hypersetup{
	colorlinks=flase,
	linkcolor=blue,
	filecolor=magenta,      
	urlcolor=blue,
}
\fvset{fontsize=\footnotesize}

\journal{i524} 

\title{Amazon Web Services Cloudmesh Extension}

\author[1]{Gregor von Laszewski}
\author[1]{Milind Suryawanshi}
\author[1]{Piyush Rai}

\affil[1]{School of Informatics and Computing, Bloomington, IN 47408, U.S.A.}


\dates{project-006, \today}

\ociscodes{Cloud, I524}

% replace this with your url in github/gitlab
\doi{Report:
	\url{https://github.com/cloudmesh/sp17-i524/blob/master/project/S17-IR-P006/report/report.pdf}\\
	Code: \url{https://github.com/cloudmesh/cloudmesh.aws}}


\begin{abstract}
	
Cloudmesh client provides and easy abstraction to gain access to hybrid
clouds from the commandline. In contrast to other systems it provides
a mechanism of adding plugins to the command. The command can not only
be used as a commandline tool, but as a shell. Due to the swift
developments in the area of cyber infrastructure such a convenient
plugin allows us to integrate easily new services easily. This work
uses our next generation cloudmesh shell and demonstrates the ease of
integrating Amazon Web Services services into it. Hence Cloudmesh
client can via other plugins achieve interoperability for virtual
machine management by providing plugins for OpenStack, Azure, Comet
Cloud Services and other frameworks including containers. Here we
focus on extensions to bring its capability to work with Amazon EC2
services. As our integration demonstrates how to leverage libcloud as a
backend, more than 30 other providers could easily be targeted.
 
\end{abstract}

\setboolean{displaycopyright}{true}

\begin{document}

\maketitle

\section{Introduction}

Amazon provides a web service in a cloud (AWS) to deploy
virtual machines (VM) with numerous operating systems environment
available in different flavors. The images are called Amazon Machine
Image (AMI) and are available as pre-configured for different
applications. Once can also create his own custom environment
\cite{www-amazon-ec2}.

The Amazon EC2 drivers provides numerous functionalities to
authenticate into the AWS, list available configurations and create
and boot VMs. Many interfaces exist providing access to many different
frameworks and platforms. Being able to utilize access through Web
services allows developers to integrate AWS into their service
offerings. Within Indiana University we have provided for the last
seven years cloud services based on a variety of cloud services.
While observing the evolution and practical use of these services it
has become clear that in order to switch easily from one cloud to
another we need a uniform interfaces that provides the most elementary
service offerings of managing virtual machines. Our design offers tha
ability to switch easily between different cloud providers. Figure
\ref{F:switch} demonstrates this convenient feature. Here we define in
our cloudmesh client a cloud and set it to AWS, we than start a vm, we
can do the same for OpenStack as demonstrated. Important to note is
that all details of name assignment, flavor and image management, are
abstracted out, but can be controlled and are provided as part of an
easy to manage configuration framework. This has advantages for
university settings as we can distribute such configurations for
particular classes and customize the use for many hundrets of non
cloud experts. In fact using our abstractions allows non cloud experts
easily to use clouds. 

\begin{figure}[htb]
  \centering
  \begin{verbatim}
  set default cloud=aws
  vm boot
  set default cloud=openstack
  vm boot 
  \end{verbatim}
  \vspace{-2.0\baselineskip}
  \caption{cms aws image refresh}
  \label{F:switch}
\end{figure}
		

\section{Getting Started}

One needs to create an aws account first to be able to access its cloud. The instructions for it are available at \cite{www-amazon-aws}. A pair of access key and secret keys are required to be generated to authenticate into the cloud \cite{www-amazon-key}. These keys are required to be kept confidential. They are to be specified in a yaml configuration file. Default vm image and flavor can also be specified in the configuration file. Following are some of the configuration entries:

\begin{verbatim} 
cloudmesh:
clouds:
aws:
credentials:
EC2_ACCESS_KEY: 'ACCESS KEY'
EC2_SECRET_KEY: 'SECRET KEY'
.
.

default:
image: 'IMAGE ID'
size: 'SIZE ID'
location: 'LOCATION'

\end{verbatim}

The user need to ensure that the combination of image, size and location is valid and that it's account has the required privileges for it. The instructions for installation of AWS client can be found at \cite{www-cloudmesh-aws}. Once, the client has been installed and configurations settings enabled, the mongodb and the rest services to access the database can be started by following command from the root directory of the code:

\begin{verbatim}
make rest
\end{verbatim}

The above services will be required by AWS client to store cloud related information locally. The user can now execute the client commands e.g.:

\begin{verbatim}
cms aws flavor refresh
\end{verbatim}

This will fetch the list of image sizes available on Amazon EC2 cloud.

\section{Cloudmesh Design}
\label{S:cmclient}

The cloudmesh client toolkit \cite{www-cloudmesh-client} is a
lightweight client interface of accessing heterogeneous clouds,
clusters, and workstations right from the users computer. The user can
manage her own set of resources she would like to utilize. Thus the
user has the freedom to customize the integration of a variety of
cyber infrastructure resources into Cloudmesh. Cloudmesh client
includes an API, a command-line client, and a command-line shell. It
strives to abstract backends to databases that are used to manage the
workflow utilizing different infrastructures and services. As
explained earlier, switching to activate virtual machines from
OpenStack clouds to Amazon is as simple as specifying the name of the
cloud. Moreover, cloudmesh client can be installed on Linux, MacOSX,
and in future Windows.

\subsection{General Requirements and Goals}

Cloudmesh has from its inception followed the following general design
goals and requirements.

\begin{description}
\item [Technology agnostic.] An important aspect of cloudmesh is to
  offer access to useful services APIs and interfaces in a technology
  agnostic fashion. Thus it should be possible for example to switch
  easily between different IaaS providers. CM has excellently
  protected us during the changes of the OpenStack interfaces and
  libraries from the very first version of OpenStack and other IaaS
  frameworks \cite{paper-eucalyptus}.

\item[Easy to use.] Cloudmesh simplifies access of 
  complex workflow to integrate with IaaS and deploy on them new
  platforms and software stacks. This advanced feature is not only to
  be performed by expert IT personal or programmers, but in fact by
  data scientists which we found in practice have less experience in
  such areas. As we deal often with many compute nodes it is often
  insufficient to just provide a graphical user interface or portal, but
  we need to provide APIs, REST interfaces and especially command-line
  and shell access in an easy comprehensible fashion. 

\item[Expandable.] While working over the last years in the area it is
  obvious that the technology is rapidly evolving and new features
  need to be integrated, Thus it is important that cloudmesh is easy
  to expand and new features can be added while leveraging a core set
  of functionality and services.

\item[Documented.] It is important that we provide from the start
  documentation to existing and new features and make it easy for the
  developers to contribute documented add ons, but also allow users to
  have access to documentation including examples. This will include
  easy to follow documented installation and configuration steps to
  guarantee successful deployment and use

\item[Repeatable and automated deployment.] When we install software
  stacks an a variety of platforms it is expected that that can easily
  be replicated and repeated automatically.

\item[Portable.] It is important to provide services in cross
  platforms compatible fashion. It ensures working, executable
  software deployment on multiple platforms. This includes not only
  the installation of the software, but the integration of external
  services and tools such as DevvOps or workflow frameworks that could
  support the general mission of a data scientist.

\item[Abstractions.] To address some of the design issues our
  requirements implicitly asks for the existence of a number of
  abstractions and interfaces that can be used to enable portable and
  crossplatform services and tools.

\end{description}

\subsection{Architecture Requirements and Goals}

In addition to the general requirements we set some specific
architectural requirements and goals that have been growing from
previous versions of cloudmesh and our earlier work in this area.

\begin{description}

\item{\bf Client based.} Cloudmesh client as the name indicates is a
  client based toolkit that is installed and run on the users
  computers. An add on component to use the client within a portal is
  available. Thus we distinguish the client that contains most of the
  functionality and the portal that can access the functionality
  through a locally maintaine Web portal. Important to note is that
  the user manages its own credentials and thus security and
  credential management is done directly on the users machine instead
  through a hosted Web portal. This increases the security as access
  to any credential is managed by the user and is not part of a
  credential management system.

\item{\bf REST.} Although Cloudmesh provides a client interface, it
will provide a REST interface to many of its services in order to
support service based deployments. The basic APIs developed for the
client can easily be reused to implement such interfaces.\footnote{we
  have demonstrated that cloudmesh APIs can be used to implement REST
  interfaces in a variety of frameworkks such as Flask, Django, and Cherypy}

\item{\bf Layered Architecture.} Cloudmesh client has a layered
architecture that allows easy development of new features. This also
allows contribution by the community while developing integrated and
smaller sub components. Figure \ref{??} depicts the various layers. A
resource abstraction layer allows the integration of a multitude of
resources spanning HPC, Containers, and Cloud resources. (At this time
we focus on OpenStack and Slurm resources. We are working on
reintegrating resources such as Azure, AWS, Maui, Moab, and others
which we previously supported, as well as new resources such as
docker).

\item{\bf Management Framework.} Cloudmesh client contains a
management framework, and its components are depicted in Figure
B. cloudmesh allows easy management of virtual machines, containers,
and the data associated with them. We are currently developing a
choreography framework that leverages Ansible, chef, and heat. All of
the functionality is easily usable through a comand-shell that also
can be used from the command-line, and a Python API. IN future we will
be providing a REST API.

\item{\bf Database Agnostic.} Cloudmesh contains some state about the
resource and environment that a user may want to use. The information
is managed in an database abstraction that would allow storing the
data in a variety of databases such as SQL and MongoDB. At this time
we have chosen SQLite to be the default database as it does not
require any additional setup and is universally available on all
operating systems without change.

\item{\bf Command-shell and Command-line.} Cloudmesh contains a
  comand-shell allowing scripts to be developed and run. However we
  designed the comand-shell in such a way that each command can also
  be called from the command-line. Through the cloudmesh state machine
  the state between comand-shell, command-client, and the portal is
  shared.

\item{\bf Cloudmesh Client Portal.} Previously, we distributed
cloudmesh with client, server, and a portal components in one
package. This however turned out to be to complex to be installed for
some of our less technically skilled user community. Thus we split up
the install into two independent packages. The cloudmesh client and
the cloudmesh portal. The portal provides some elementary features to
manage virtual machines and HPC jobs. At this time the portal is
considered to be alpha technology. Just as the client the portal is to
be run on the local user machine in oredr to allow increased security
by managing the credentials locally rather than on a server.

\item{\bf Cloudmesh Two Factor Authentication.} We have an
exploratory project in place that looks at the use of Yubikeys for
cloudmesh, client and cloudmesh portal.

\item{\bf Cloudmesh Comet.} We are actively developing the client
interface for SDSC’s comet supercomputer allowing bare metal
provisioning. The interface reuses cloudmesh components and
technologies while interfacing with the comet cloud REST
interface. The goal here is to manage virtual clusters.

\end{description}

\subsection{Architecture}


\begin{figure}[htb]
  \centering
     \includegraphics[width=1.0\columnwidth]{images/cloudmesh-arch-1.pdf}
  \caption{Cloudmesh layered architecture.} 
  \label{F:NIST-arch}
\end{figure}

\begin{figure}[htb]
  \centering
     \includegraphics[width=1.0\columnwidth]{images/aws-rest.pdf}
  \caption{Cloudmesh AWS REST Service.} 
  \label{F:NIST-arch}
\end{figure}



\section{Implementation considerations}

The AWS client is based on cloudmesh.cmd5 \cite{www-cloudmesh-cmd5}
and cloudmesh.common \cite{www-cloudmesh-common}. It uses mongodb in
the back-end to store the cloud information such as list of available
images or instances running on the cloud. Requests library
\cite{www-python-requests} is used to connect to the back-end through
rest services. The rest service is deployed using cloudmesh REST
framework \cite{www-cloudmesh-rest}.

\begin{verbatim}
client -> rest (eve) -> database (nosql)
           ^ object definitions
\end{verbatim}

\section{Architecture}

The commands are implemented as methods of a class AwsCommand which is
based on PluginCommand from cloudmesh shell. Depending on the
arguments passed, corresponding routine is called from Aws client API
which acts as a wrapper around the libcloud EC2 drivers
\cite{www-libcloud-ec2} and is also responsible to connect with
back-end database apart from reading the configuration from yaml
file. The entire code is written in Python.

\subsection{Technologies Used}

To adhere to the principal of code reuse we are using a number of core
technologies that allow us to keep the overall developed new code
small. This includes leveraging our previous {\it cloudmesh} efforts, the
reuse of {\it mongodb} and the automatic creation of a REST service based
on example objects with {\it evengine}. Particularly the following
software components have been leveraged:

\begin{description}

\item [Libcloud EC2 driver \cite{www-libcloud-ec2}:] The driver
  provides a number of functions for various functionalities such as
  listing the available nodes, generating a key pair and deploying a
  VM.

\item [MongoDb \cite{www-mongodb}:] MongoDB is an open source document store
  database. It's used by AWS client to store information about various
  VM configuration options available on the Amazon EC2 cloud. It's
  also used to store information regarding the VMs that are running on
  the cloud. The information in the database is refreshed whenever
  it's fetched from the cloud.

\item [Cloudmesh.common \cite{www-cloudmesh-common}:] A library of
  many commonly useful functions and classes shared among all
  cloudmesh projects.

\item [Cloudmesh.cmd5:] The cmd5 is a dynamically extensible CMD based
  command shell \cite{www-cloudmesh-cmd5}.

\item [Cloudmesh.evegenie \cite{www-cloudmesh-evengine}:] The schema
  for the collections are specified in json files which are then
  converted to Eve syntax using evegenie. It creates the configuration
  file for starting up the rest services using cloudmesh rest
  framework.
	 	
\item [Cloudmesh.rest \cite{www-cloudmesh-rest}:] The cloudmesh rest
  framework is used to deploy and start the mongodb and rest
  services. The schema for the objects to be stored in the database is
  collectively specified in json file called {\it all.json}. The
  schema defined in the file is closely associated with the code in
  awsclient.py which is responsible for fetching the information from
  cloud and passing it to mongodb through rest services. From our
  experience during the development of this project, we observed that
  rest services required the schema to be precise and didn't handle
  null values for collection fields. There's another file
  all.settings.py which contains the configuration information for
  rest services such as port mongodb is running on along with the
  database name. It contains the schema for the collections and the
  list of methods to be provided by the rest services. The database
  connectivity was initially developed using pymongo library. However,
  during the review it was suggested that the code based on pymongo is
  quite low level and does not have adequate security features. This
  led to the use of Pyhthon Requests library.
	
\end{description}

\section{Usage}

One needs to create an AWS account first to be able to access its
cloud. The instructions for it are available at
\cite{www-amazon-aws}. A pair of access key and secret keys are
required to be generated to authenticate into the cloud
\cite{www-amazon-key}. These keys are required to be kept
confidential. They are to be specified in a yaml configuration
file. Default vm image and flavor can also be specified in the
configuration file (see FIgure \ref{F:conf}). 

\begin{figure}[htb]
\begin{verbatim} 
cloudmesh:
  clouds:
    aws:
      credentials:
        EC2_ACCESS_KEY: 'ACCESS KEY'
        EC2_SECRET_KEY: 'SECRET KEY'
        . . .
    default:
      image: 'IMAGE ID'
      size: 'SIZE ID'
      location: 'LOCATION'
\end{verbatim}
\vspace{-1.0\baselineskip}
\caption{Configuration}\label{F:conf}
\end{figure}

The user need to ensure that the combination of image, size and
location is valid and that it's account has the required privileges
for it. The instructions for installation of AWS client can be found
at \cite{www-cloudmesh-aws}. Once, the client has been installed and
configurations settings enabled, the mongodb and the rest services to
access the database can be started by following command from the root
directory of the code:

\begin{verbatim}
    make rest
\end{verbatim}

The above services will be required by AWS client to store cloud
related information locally. The user can now execute the client
commands e.g.:

\begin{verbatim}
    cms aws flavor refresh
\end{verbatim}

This will fetch the list of image sizes available on Amazon EC2 cloud.

\section{Cloudmesh REST Service}

\begin{figure}[htb]
\VerbatimInput{specification/flavor.json}
\vspace{-1.0\baselineskip}
\caption{Specification for flavor.}\label{F:spec-flavor}
\end{figure}

\begin{figure}[htb]
\VerbatimInput{specification/image.json}
\vspace{-1.0\baselineskip}
\caption{Specification for image.}\label{F:spec-image}
\end{figure}

\begin{figure}[htb]
\VerbatimInput{specification/keypair.json}
\vspace{-1.0\baselineskip}
\caption{Specification for keypair.}\label{F:spec-keypair}
\end{figure}

\begin{figure}[htb]
\VerbatimInput{specification/location.json}
\vspace{-1.0\baselineskip}
\caption{Specification for location.}\label{F:spec-location}
\end{figure}

\begin{figure}[htb]
\VerbatimInput{specification/node.json}
\vspace{-1.0\baselineskip}
\caption{Specification for node.}\label{F:spec-node}
\end{figure}

\begin{figure}[htb]
\VerbatimInput{specification/volume.json}
\vspace{-1.0\baselineskip}
\caption{Specification for volume.}\label{F:spec-volume}
\end{figure}

\section{AWS Commands}


\begin{table*}[p]
\caption{Cloudmesh AWS Commands}\label{T:aws-commands} 
\begin{center}
\begin{tabular}{p{2cm}p{10cm}p{5cm}}
  Function & Description & Command \\
  \hline
  Refresh &
            Whether to always fetch the information from the cloud
            over the network or display it from the local database when asked
            can be configured by setting the configuration variable {\it refresh} to
            either {\it on} or {\it off}. When the value is set to on, the information
            will always be fetched from the cloud.This functionality is
            implemented for only some of the commands as of now.
                         & \verb+aws refresh on+ \\
  \hline
  Image refresh&
                 The images available on the cloud could be listed
                 using this command. The database will also be updated with the newly
                 fetched list. The command may take up to minutes to run as the list
                 contains more than 24,000 images. 
    & \verb+aws image refresh+ \\
\hline
Image list & Depending on whether the value of {\it refresh} is
      set to {\it on} or {\it off}, the list is either fetched from the
      cloud or from the local database. &
    \verb+aws image list+\\
\hline	
Flavor refresh & This will list the different sizes of VMs
                   that are available on the cloud. The information is fetched from
                   the cloud and stored locally. & 
                                                   \verb+aws flavor refresh+ \\ 
\hline
Flavor list & The flavor list is either fetched from the
      cloud or from the local database depending on whether the
      value of {\it refresh} is set to {\it on} or {\it off}.
	
    \verb+aws flavor list+ \\
	
  \hline
  Start/Boot vm & It creates a new node instance and start that
                  node automatically. The required parameter is
                  \textit{IMAGE\_ID}. This command assumes user has created keypair
                  with name \textit{AWS1}. The default values; flavor and location are
                  taken from cloudmesh.yaml. Those values are required to be set
                  before executing this command. Once it has been created, it can be
                  seen using vm list.
                         & \verb+aws vm boot IMAGE_ID+ \\
  \hline
  Reboot vm & This command is used to reboot a running vm
              instance. The state of the node gets changed from RUNNING to
              REBOOTING while its restart. The rebooting time of node depends on
              the configuration the configuration that was selected while its
              creation. \textit{vm reboot} requires \textit{NODE\_UUID} of the
              node to be rebooted which can be retrieved using either \textit{vm
              list} or \textit{vm refresh}.
                         & \verb+aws vm reboot NODE_UUID+ \\
  \hline
  VM delete & It destroys the instance of a node and all the data
              associated with it including backup. It takes few seconds to
              terminated the instance. At first, the state of the node changes
              to \textit{TERMINATED} which can be observed using \textit{vm
              refresh} . The node gets removed from the list subsequently.
                         & \verb+aws vm delete UUID+ \\
  \hline
  VM list  & It lists out all the nodes along with their status as stored in the database.
                         & \verb+aws vm list+ \\
  \hline
  Keypair create & To create the node, one of the essential component is \textit{key pair}. Amazon EC2 uses public key of the user to encrypt the password and then recipient uses private key to decrypt it. These public and private keys are know as a \textit{key pair}. They are required to be given a name.
                         & \verb+aws keypair create NAME+ \\
  \hline
  Keypair delete & It allows user to delete a created \textit{key pair} which is no longer in use.
                         & \verb+aws keypair delete NAME+ \\
  \hline
  Keypair refresh & It refreshes the list of created \textit{key pairs} in the database and also displays it on the screen.
                         & \verb+aws keypair refresh+ \\
  \hline
  Keypairs list & It lists out all the \textit{key pairs} information stored locally in the database.
                         & \verb+aws keypair list+ \\
  \hline
  Keypair get & It returns the key pair object, which has the name of \textit{key pairs}, driver and hash key.
                         & \verb+aws keypair get NAME+ \\
  \hline
\end{tabular}
\end{center}
\end{table*}


\begin{table*}[htb]
\caption{Cloudmesh AWS Commands}\label{T:aws-commands} 
\begin{center}
\begin{tabular}{p{2cm}p{10cm}p{5cm}}
  Locations refresh & It will show all the available locations associated with Amazon EC2 account. For free tier, user will get two locations. More locations will be available in paid service. The database records are also updated with this command.
                         & \verb+aws location refresh+ \\
  \hline
  Locations list & It lists the locations information stored in the database.
                         & \verb+aws location list+ \\
  \hline
  Volume create & It creates the volume for vm. The size of volume is specified in GB, the default value is set to 1 GB. The maximum number of volumes that can be attached to a vm will depend on its operating system.
                         & \verb+aws volume create VOLUME_NAME+ \\
  \hline
  Volume refresh & This command shows the created volumes with the id, size and the driver name. It also updates the database.
                         & \verb+aws volume refresh+ \\
  \hline
  Volume list & This command displays the volume information stored in the database.
                         & \verb+aws volume list+ \\
  \hline
  Volume delete & User can delete the unwanted volumes using the \textit{VOLUME\_NAME}.
                         & \verb+aws volume delete VOLUME_NAME+ \\
\end{tabular}
\end{center}
\end{table*}

	
\section{Benchmarking Cloudmesh AWS}

We benchmarked the aws commands that fetch information either from the cloud over the network or from the local database. The information is then displayed to the user. The informations that is fetched over the network is also updated in the database. The performance statistics observed give us the overview of the difference in execution time of the two kind of commands. Each command was run 50 times with its execution time reocrded.  The mean and standard deviation of the timings were calculated using statistical methods in R. Table \ref{tab:awsbanchmark}  shows the observed values in seconds.

\begin{table}[h!]
	\caption{\bf Banchmarking of AWS commands}
	\begin{tabular}{ | m{3cm} | m{1.7cm} | m{2.7cm} |} 
		\hline
		Commands & Mean & Standard Deviation\\
		\hline
		flavor refresh & 0.612 & 0.059 \\
		flavor list & 0.207 & 0.024\\
		keypair refresh & 1.967 & 0.964 \\
		keypair list & 0.240 & 0.062 \\
		location refresh & 1.933 & 1.246 \\
		location list & 0.172 & 0.039 \\
		vm refresh & 2.342 & 1.876 \\
		vm list & 0.160 & 0.023 \\
		volume refresh & 2.184 & 1.878 \\
		volume list & 0.170 & 0.020 \\
		
		\hline
	\end{tabular}
	\label{tab:awsbanchmark}
\end{table}

The commands that fetch the information from local database are on an average 85\% faster than the commands that fetch information from the cloud and also have lesser variation in their timings. The difference in performance tends to decrease when the information to be fetched is more. This may be due to the limitation of the machine on which it was tested. 


\section{Use Case: Create an AWS Node}
	The following are the pre-requisites to be able to create an EC2 node.

\subsection{Prerequisite}

\begin{enumerate}

\item User should have a valid AWS account.\item
  users Create a IAM
  \cite{www-aws-iam} user for which the access and secret key will
  have to be generated \cite{www-attach-policy}. The keys will ahve to
  added to the yaml cloudmesh configuration file.
	
\item The required permissions are needed to be granted to the user.
  
  \begin{itemize}			
  \item visit IAM home \cite{www-aws-iam-home}
  \item select \textbf{policie}s in left hand menu
  \item create \textbf{administrator policy} from amazons existing
    policies
  \item select \textbf{administrator checkbox} and \textbf{attach} to
    your user
  \end{itemize}
		
\item Open the ../cloudmesh.yaml configuration file and update with
  EC2\_ACCESS\_KEY, EC2\_SECRET\_KEY, you just copied. Now set default
  flavor, image and location, as t2.micro, ami-0183d861 and us-west-1
  respectively in same file.

\item Mongodb server should be up and running (please refer section
  2. Getting started)


\end{enumerate}

\subsection{Create Node}

\begin{enumerate}

\item Create keypair name using command
	
	\begin{verbatim}
	cms aws keypair create  AWS1
	\end{verbatim}
	
  It will create the \textit{keypair name}, it is essential to create
  the \textit{node}. To verify it, \textit{cms aws keypair list}
  command will list down all created \textit{keypairs} so far.
	
\item Now create the node
	
	\begin{verbatim}
	cms aws vm boot ami-0183d861
	\end{verbatim}
	
  Above command will create a node instance with the image of
  ami-0183d8661.

\end{enumerate}	

\begin{table}[h!]
	\caption{\bf cms aws image refresh }
	\begin{tabular}{ | m{1.5cm} | m{3.8cm} | m{2.2cm} |} 
		\hline
		id & name & driver\\
		\hline
		aki-02b79b47  & pv-grub-hd00\_1.03-i386 & 627 \\
		aki-033c6d46 & & <libcloud.compu te.drivers.ec2. EC2NodeDriver object at 0x105898a90>  \\
		aki-037a5e46 & ubuntu/kernels-testing/ubuntu-lucid-i386-linux-image-2.6.32-347-ec2-v-2.6.32-347.52-kernel &  <libcloud.comp ute.drivers.ec2 .EC2NodeDriver object at 0x105898a90>\\
		. & . & .   \\
		\hline
	\end{tabular}
	\label{tab:imagerefresh}
\end{table}

\begin{table}[h!]
	\caption{\bf aws flavor refresh }
	\begin{tabular}{ | m{1.2cm} | m{2.5cm} | m{1.1cm} | m{1.1cm} |m{0.8cm} |} 
		\hline
		id & name & ram & disk & price..\\
		\hline
		t1.micro & Micro Instance & 627 & 15 & 0.025 \\
		m1.small & Small Instance & 1740 & 160 & 0.047 \\
		m1.large & Large Instance & 3840 & 410 & 0.095 \\
		. & . & . & . & . \\
		\hline
	\end{tabular}
	\label{tab:flavorlist}
\end{table}

	\begin{table}[h!]
	\caption{\bf cms aws vm boot ami-0183d861 }
	\begin{tabular}{ | m{5.2cm} | m{1.1cm} | m{1.2cm} |} 
		\hline
		uuid & name & states ..\\
		\hline
		eec0cb8ddb3a17764c6658278162496 81cec1f67 & test1 & running \\
		
		\hline
	\end{tabular}
	\label{tab:vmboot}
\end{table}

\begin{table}[h!]
	\caption{\bf cms aws vm list }
	\begin{tabular}{ | m{5.1cm} | m{1.2cm} | m{1.2cm} |} 
		\hline
		uuid & name & states ..\\
		\hline
		eec0cb8ddb3a17764c6658278162496 81cec1f67 & test1 & running \\
		
		\hline
	\end{tabular}
	\label{tab:vmlist}
\end{table}


\begin{table}[h!]
	\caption{\bf cms aws keypair create AWSCLI }
	\begin{tabular}{ | m{0.8cm} | m{7.3cm} |} 
		\hline
		name & fingureprint\\
		\hline
		AWS1 & 17:20:41:93:1d:b1:e1:a6:84:44:99:0d:87:d2:ab:72:3d:a1 :f0:a7 \\
		AWS2 & 72:31:87:fc:2b:0f:93:1a:24:f7:9a:01:91:29:f7:eb:27:6a :2d:fa \\
		\hline
	\end{tabular}
	\label{tab:keypaircreate}
\end{table}

	\begin{table}[h!]
	\caption{\bf cms aws location refresh }
	\begin{tabular}{ | m{0.7em} | m{1.7cm}| m{1cm} | m{2.2cm} |m{1.7cm} |} 
		\hline
		id & name & country & availability\_zone & zone\_state ..\\
		\hline
		0 & us-west-1a & USA & us-west-1a & available \\
		1 & us-west-1b & USA & us-west-1b & available \\
		\hline
	\end{tabular}
	\label{tab:locationlist}
\end{table}

\begin{table}[h!]
	\caption{\bf cms aws volume create VOL\_TEST\_1 }
	\begin{tabular}{ | m{3.8cm} | m{1.8cm} | m{2cm} |}
		\hline
		id & size & driver \\
		\hline
		vol-009775889a2d50154 & 1 & Amazon EC2\\
		\hline
	\end{tabular}
	\label{tab:createvolume}
\end{table}

	\begin{table}[h!]
	\caption{\bf cms aws volume refresh }
	\begin{tabular}{ | m{4.2cm} | m{1cm} | m{2.4cm} |}
		\hline
		Id & Size & Driver \\
		\hline
		vol-009775889a2d50154 & 1 & Amazon EC2\\
		vol-0c30e77f6a2a263a & 1 & Amazon EC2\\
		\hline
	\end{tabular}
	\label{tab:tab8}
\end{table}

	

\begin{comment}
\section*{Acknowledgements}

Prof. Gregor von Laszewski originally suggested this project and
provided the objectives in simplistic form. He reviewd the code as it
was developed during the course of this project. His inputs helped us
to make the code more secure and efficient. He provided us with
different resources to look for help and overcome the challenges faced
during the course.
\end{comment}

% Bibliography

\bibliography{references}
 
\section*{Author Biographies}
\begingroup
\setlength\intextsep{0pt}
\begin{minipage}{1.0\columnwidth}
  \noindent
  {\bfseries Milind Suryawanshi} received his BE (Electronics and
  Telecommunication) in 2010 from The University of Pune. His research
  interests include Big Data analytics for intelligence and research.
\end{minipage}
\begin{minipage}{1.0\columnwidth} 
  \noindent
  {\bfseries Piyush Rai} received his BE (Computer) in 2011 from The
  University of Pune. His research interests also include Big Data
  analytics for military intelligence and frinancial markets.
\end{minipage}
\begin{minipage}{1.0\columnwidth} 
  \noindent
  {\bfseries Gregor von Laszewski} has written more than 100
  publications in the are of Grid and Cloud computing. He obtained a
  Ph.D. from Syracuse University and has worked at Argonne National
  Laboratory. His team is providing with cloudmesh a client interface
  to XSEDE SDSC comet virtual clusters.
\end{minipage}

\endgroup


\end{document}
