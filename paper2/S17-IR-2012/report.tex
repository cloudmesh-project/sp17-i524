\documentclass[9pt,twocolumn,twoside]{../../styles/osajnl}
\usepackage{fancyvrb}
\journal{i524} 

\title{Retainable Evaluator Execution Framework}

\author{Pratik Jain}
\affil{School of Informatics and Computing, Bloomington, IN 47408, U.S.A.}
\affil{Corresponding authors: jainps@iu.edu}

\dates{Paper-2, \today}

\ociscodes{REEF, Tang, Wake}

% replace this with your url in github/gitlab
\doi{\url{https://github.com/cloudmesh/sp17-i524/blob/master/paper2/S17-IR-2012/report.pdf}}


\begin{abstract}
Apache REEF is a Big Data system that makes it easy to implement
scalable, fault-tolerant runtime environments for a range of data
processing models on top of resource managers such as Apache YARN and
Mesos. The key features and abstractions of REEF are discussed. Two
libraries of independent value are introduced. Wake is an
event-based-programming framework and Tang is a dependency injection
framework designed specifically for configuring distributed
systems. \newline
\end{abstract}

\setboolean{displaycopyright}{true}

\begin{document}

\maketitle

\section{Introduction}

With the continuous growth of Hadoop the range of computational
primitives expected by its users has also broadened. A number of
performance and workload studies have shown that Hadoop MapReduce is a
poor fit for iterative computations, such as machine learning and
graph processing. Also, for extremely small computations like ad-hoc
queries that compose the vast majority of jobs on production clusters,
Hadoop MapReduce is not a good fit. Hadoop 2 addresses this problem by
factoring MapReduce into two components: an application master that
schedules computations for a single job at a time, and YARN, a cluster
resource manager that coordinates between multiple jobs and
tenants. In spite of the fact that this resource manager, YARN, allows
a wide range of computational frameworks to coexist in one cluster,
many challenges remain \cite{reefpaper}.

From the perspective of the scheduler, a number of issues arise that
must be appropriately handled in order to scale-out to massive
datasets. First, each map task should be scheduled close to where the
input block resides, ideally on the same machine or rack. Second,
failures can occur at the task level at any step, requiring backup
tasks to be scheduled or the job being aborted. Third, performance
bottlenecks can cause an imbalance in the task-level progress. The
scheduler must react to these stragglers by scheduling clones and
incorporating the logical task that crosses the finish line first.

Apache REEF (Retainable Evaluator Execution Framework), a library for
developing portable applications for cluster resource managers such as
Apache Hadoop YARN or Apache Mesos, addresses these challenges. It
provides a reusable control-plane for scheduling and coordinating
task-level work on cluster resource managers. The REEF design enables
sophisticated optimizations, such as container re-use and data
caching, and facilitates workflows that span multiple
frameworks. Examples include pipelining data between different
operators in a relational system, retaining state across iterations in
iterative or recursive data flow, and passing the result of a
MapReduce job to a Machine Learning computation.

\section{Features}

Due to its following critical features, Apache REEF drastically
simplifies development of resource managers \cite{reefhome}.

\subsection{Centralized Control Flow} 
Apache REEF turns the chaos of a distributed application into various
events in a single machine. These events include container allocation,
Task launch, completion, and failure.
\subsection{Task runtime}
Apache REEF provides a Task runtime which is instantiated in every
container of a REEF application and can keep data in memory in between
Tasks. This enables efficient pipelines on REEF.
\subsection{Support for multiple resource managers}
Apache REEF applications are portable to any supported resource
manager with minimal effort. In addition to this, new resource
managers are easy to support in REEF.
\subsection{.NET and Java API}
Apache REEF is the only API to write YARN or Mesos applications in
.NET. Additionally, a single REEF application is free to mix and match
tasks written for .NET or Java.
\subsection{Plugins} 
Apache REEF allows for plugins to augment its feature set without
hindering the core. REEF includes many plugins, such as a name-based
communications between Tasks, MPI-inspired group communications, and
data ingress. \newline

As a result of such features Apache REEF shows properties like
retainability of hardware resources across tasks and jobs,
composability of operators written for multiple computational
frameworks and storage backends, cost modeling for data movement and
single machine parallelism, fault handling \cite{techopedia} and
elasticity.

\section{Key Abstractions}

REEF is structured around the following key abstractions
\cite{reefarticle}:

Driver: This is a user-supplied control logic that implements the
resource allocation and Task scheduling logic. There is exactly one
Driver for each Job. The duration and characteristics of the Job are
determined by this module.

Task: This encapsulates the task-level client code to be executed in
an Evaluator.

Evaluator: This is a runtime environment on a container that can
retain state within Contexts and execute Tasks (one at a time). A
single evaluator may run many activities throughout its lifetime. This
enables sharing among Activities and reduces scheduling costs.

Context: It is a state management environment within an Evaluator that
is accessible to any Task hosted on that Evaluator.

Services: Objects and daemon threads that are retained across Tasks
that run within an Evaluator \cite{reefpaper}. Examples include caches
of parsed data, intermediate state, and network connection pools.

\section{Wake and Tang}

The lower levels of REEF can be decoupled from the data models and
semantics of systems built atop it. This results in two standalone
systems, Tang and Wake which are both language independent and allow
REEF to bridge the JVM and .NET.\newline

Tang is a configuration management and checking framework
\cite{reeftang}. It emphasizes explicit documentation and automatic
checkability of configurations and applications instead of ad-hoc,
application-specific configuration and bootstrapping logic. It not
only supports distributed, multi-language applications but also
gracefully handles simpler use cases. It makes use of dependency
injection to automatically instantiate applications. Given a request
for some type of object, and information that explains how
dependencies between objects should be resolved, dependency injectors
automatically instantiate the requested object and all of the objects
it depends upon. Tang makes use of a few simple wire formats to
support remote and even cross-language dependency injection.\newline

Wake is an event-driven framework based on ideas from SEDA, Click,
Akka and Rx \cite{reefwake}. It is general purpose in the sense that
it is designed to support computationally intensive applications as
well as high-performance networking, storage, and legacy I/O
systems. Wake is implemented to support high-performance, scalable
analytical processing systems i.e. big data applications. It can be
used to achieve high fanout and low latency as well as high-throughput
processing and it can thus aid to implement control plane logic and
the data plane.

Wake is designed to work with Tang. This makes it extremely easy to
wire up complicated graphs of event handling logic. In addition to
making it easy to build up event-driven applications, Tang provides a
range of static analysis tools and provides a simple aspect-style
programming facility that supports Wake’s latency and throughput
profilers.

\section{Relationships with Other Apache Products}

Given REEF's position in the big data stack, there are three
relationships to consider: Projects that fit below, on top of, or
alongside REEF in the stack \cite{reefproposal}.

\subsection{Below REEF}
REEF is designed to facilitate application development on top of
resource managers like Mesos and YARN. Hence, its relationship with
the resource managers is symbiotic by design.

\subsection{On Top of REEF}
Apache Spark, Giraph, MapReduce and Flink are some of the projects
that logically belong at a higher layer of the big data stack than
REEF. Each of these had to individually solve some of the issues REEF
addresses.

\subsection{Alongside REEF}

Apache builds library layers on top of a resource management
platform. Twill, Slider, and Tez are notable examples in the
incubator. These projects share many objectives with REEF. Twill
simplifies programming by exposing a programming model. Apache Slider
is a framework to make it easy to deploy and manage long-running
static applications in a YARN cluster. Apache Tez is a project to
develop a generic Directed Acyclic Graph (DAG) processing framework
with a reusable set of data processing primitives. Apache Helix
automates application-wide management operations which require global
knowledge and coordination, such as repartitioning of resources and
scheduling of maintenance tasks. Helix separates global coordination
concerns from the functional tasks of the application with a state
machine abstraction.

\section{Conclusion}

REEF is a flexible framework for developing distributed applications
on resource manager services. It is a standard library of reusable
system components that can be easily composed into application
logic. It possesses properties of retainability, composability, cost
modeling, fault handling and elasticity. Its key components are
driver, task, evaluator, context and services. Its relationship with
projects above it, below it and alongside it in the big data stack is
discussed. Thus, a brief overview of REEF is shown here.

\section*{Acknowledgements}
The author thanks Prof. Gregor von Laszewski and all the course AIs
for their continuous technical support.

% Bibliography

\bibliography{references}

\end{document}
