\documentclass[9pt,twocolumn,twoside]{../../styles/osajnl}
\usepackage{fancyvrb}
\journal{i524} 

\title{InCommon}

\author[1]{Michael Smith}


\affil[1]{School of Informatics and Computing, Bloomington, IN 47408, U.S.A.}

\affil[*]{Corresponding authors: mls35@iu.edu}

\dates{paper2, \today}

\ociscodes{InCommon, User authentication, identity management, I524}

% replace this with your url in github/gitlab
\doi{\url{ttps://github.com/cloudmesh/sp17-i524/blob/master/paper2/S17-IO-3019/report.pdf}}


\begin{abstract}

InCommon is a federated security service that is responsible for the
management of identity verification solutions serving U.S. education
and research\cite{www-incommonppt}.  All users within this federation
allow partners to share identity information in order easily recognize
the user. This federation provides numerous benefits for users and
service providers through the convenience of single sign on
capabilities for the user.  Privacy is enhanced by limiting the
distribution of personal information amongst numerous service
providers.  Scalability is easily facilitated due to the unified
policies and management procedures.  InCommon utilizes the single sign
on software Shibboleth which is an open source project that enables
federated organizations to connect users to various applications
through a convenient secure method.  The backbone of this software is
built on the language SAML or security assertion markup language that
creates the basis for its application.  Programs such as InCommon
assurance and university case studies are examined.
\end{abstract}

\setboolean{displaycopyright}{true}

\begin{document}

\maketitle

\TODO{This review document is provided for you to achieve your
  best. We have listed a number of obvious opportunities for
  improvement. When improving it, please keep this copy untouched and
  instead focus on improving report.tex. The review does not include
  all possible improvement suggestions and for each comment you may
  want to check if it applies elsewhere in the document.}

\TODO{The reference in the abstract should be moved to the main
  text. References in the abstract are ok in general, but should be
  reserved for specific claims that critical to the abstract and
  paper, e.g. "In this paper, we reproduce the findings that
  ... [1]..." In your case, the reference is general and should be
  included in the main text.}

\TODO{Some parts of the abstract, such as the sentences on using
  Shibboleth and SAML don't seem that critical and can be omitted. The
  abstract should tell the reader what is most important about the
  technology and entice them to read the entire paper, and going into
  these two details in this case might not be necessary. Leaving up to
  you whether to omit. Good abstract overall.}

\TODO{Overall, great job. You can improve the flow of the paper by
  providing more transition between some of the sections. In addition,
  a couple sections are borderline out of scope for a neutral paper
  like this. Finally, there are some cases where you can use more
  neutral language. See below for details.}

\TODO{Assessment: Some revisions suggested.}

%MCEPASTEBIN%
\section{Introduction}

Electronic credentialing of individuals requires an effective
implementation of a set of policies and procedures.  In order to be
successful, identity management requires an organization to keep user
information up to date, providing the trust needed for secure
transactions, and determine user access of online applications.  The
major issues with identity management is the increasing number of
cloud services or applications that are web hosted \TODO{,} all of
which have different policies for implementing identity verification.
The solution is to establish a federation which is defined \TODO{as}
''an association of organizations that come together to exchange
information, as appropriate, about their users and resources in order
to enable collaborations and transaction'' \cite{www-incommonppt}.
Within this federation the parties come into an accordance on the
policies associated with identity management.  A great example of a
federation that encompasses this definition is InCommon\TODO{.}

\section{InCommon}

InCommon was founded by the advanced technology organization
Internet2.  Their mission is to create an environment that facilitates
the ability for educators and researchers to collaborate regardless of
their location.  Their network encompasses over 90,000 institutions,
305 universities, 70 government agencies with network operations
center powered by Indiana University \cite{www-internet2}

Through the InCommon service, users will no longer have to \TODO{Why
  "will no longer?" Isn't this already the case?}  remember a plethora
of usernames and passwords for each web service.  Instead\TODO{,} they
will be able to have single sign on (SSO) conveniences.  Giving time
back to faculty, staff and students for education, research and other
contributions to the University.  Any service provider within this
federation no longer needs to manage databases of username and
passwords, the users are verified and then administered security
tokens to then engage with service providers within the federation.
By limiting the amount of identity information required of the service
provider, the users privacy is safe \TODO{Probably more correct to say
  "safer" since there is still (always) risk. But storing login
  information in a single place is definitely an improvement.} in the
event of a security breach of the service provider.

\section{SAML}

The language used by InCommon is referred to as security assertion
markup language or SAML.  This language is based in XML which allows
for the exchange of authentication information between a user and a
provider \cite{www-wiki}.  It is the industry accepted standard
language for identity verification by government, businesses and
service providers. \TODO{Could you provide a source or more detail
  about it being "the industry accepted?"} The general user
verification is done by an identity provider(IdP) which is responsible
for user authentication through the use of security tokens with SAML
2.0 \cite{www-empower}. Service providers (SP) are defined as entities
that provide web services, internet, web storage etc.  They rely on
the IDPs for the verification process.  A significant amount of the
major web service providers such as Google, Facebook, Yahoo,
Microsoft, and Paypal play a dual role and exist also as identity
providers.

\section{Shibboleth}

Shibboleth is the service that has a suite of products that assist the
InCommon federation through utilization of SAML in programming
languages such as C ++ and Java\cite{www-shibboleth}.  The normal
authentication process for Shibboleth is to intercept access to a
service, determine who is the identity provider for the user.  Once
the identity provider has been discovered a SAML authentication
request is sent to the identity provider.  Identity providers SAML
response will have the relevant user information for verification.
The extracted user information will then be passed to the service
provider or resource determining user accessibility.  While the
process sounds complex it will occur instantaneously, after the user
has entered its single sign on.


\section{Certificate Service}

The types of certificates that InCommon have available for issue are
SSL/TLS, extended validation, client, code signing, IGTF server, and
elliptical curve cryptography certificates (ECC).  SSL (secure sockets
layer) is “the standard security technology for establishing an
encrypted link between a web server and a browser.  The link ensures
all data is passed between the web server and browers remain private
and integral”\cite{www-ssl}.  The details of an SSL certificate
issued by InCommon will contain user information and the
expiration date. For all educational institutions, InCommon offers
unlimited server and client certificates for the annual fee.

\section{Duo}

In collaboration with the trusted access company Duo, Incommon offers
two factor authentication through the utilization of the users smart
phone\cite{www-duo}.  A duo mobile app supports the following
platforms: Apple iOS, Google Android, Windows mobile, Palm WeboS,
Symbian OS, RIM blackberry, Java J2ME.  The application will generate
a randomly generated one time password that the user will type into
the web application for a more secure identity verification.  Two
factor authentication does not require smartphone, other methods such
as automate voice calls or SMS messages.  In addition to Duo mobile, a
service called Duo push is available which does not require the user
to type in the password, authentication occurs directly from the
mobile app.  It is up to the university to determine how Duo is
deployed, whether it will occur with the identity provider or the
service provider.  If it is deployed at the service provider
destination, Duo web supports the following client libraries: python,
ruby, classic ASP, ASP.net, Java, PHP, Node.js, ColdFusion, and Perl.

\TODO{The content for these sections (SAML, Shibboleth, Certs, Duo) is
  good and at the right level of detail. However, there is not much
  flow between the sections. You can improve by putting these four as
  subsections to an Architecture section and provide some motivation
  for why each component is necessary and how they fit
  together. E.g. "Architecture: InCommon's main functionality is
  implemented by SAML. <SAML subsection>. Then: To facilitate
  InCommon's use in programming environments, <Shibolleth section>"
  and so on. Just provide a little more transition and motivation for
  why each part of the system exists.}

\section{Assurance Program}

Incommon offers an assurance program that will examine and the
practices of an organization and will rank them based on a number of
criteria.  Areas of examination include ''...identity proofing(such as
checking government issued ID before accepting that people are who
they say they are), password handling (including making sure that
passwords are not sent or stored in the clear), and
authentication(such as ensuring the resistance of an authentication
method to session hijacking)'' \cite{www-harvard}.  There are two
levels of assurance in the InCommon program, bronze and silver.
Bronze is comparable to NIST level of Assurance 1 \TODO{What is that? You haven't mentioned it in the paper before.}, which is for common
usage of internet identity management.  Silver is comparable to NIST
level of Assurance 2, which defines the institute as having sufficent
requirements provide a security at the level for a financial
transaction \cite{www-levels}.  Compliance with a bronze level only
requires a level of self certification of the requirements, where as a
silver level is more difficult to achieve.  A third party or evaluator
that has been verified by InCommon is required to peform an audit
ensuring the identity provider is meeting all the rules and
requirements.  Many organizations and government agencies such as a
national institute of health and public universities are requiring
identity providers to become certified in this program.

\TODO{You may want to start this section with this last statement
  about large agencies requiring this kind of certification. Otherwise
  it sounds like you're describing a service that is outside the scope
  of the paper. }

\section{Cost}

The current rate for Universities to subscribe to the InCommon service
varies based on a couple of factors such as highest level of degree
offered (Doctoral, master’s etc.) and Carnegie classification.  The
range in price will vary from 20,000 to 2,000 dollars annually
\cite{www-price}.  If the organization is an internet2 member, a 25
percent discount applied to the annual fee.

\TODO{This is out of scope for the paper. Focus on the service being
  provided and how it is implemented, and how it compares to similar
  service. Interested readers can figure out how cost themselves.}

\section{University of Minnesota}

\TODO{Please, give more context and a transition from the previous
  sections of the paper. E.g. "One example where InCommon was
  successfully deployed is..."}

The University of Minnesota adopted into the InCommon federation on
September 2010 \cite{www-casestudy1}.  The university contains 51,000
students, and over 300 institutes.  Their previous identity management
vendor charged on a per certificate basis.  This differs from InCommon
which offers an annual fee with unlimited certificates.  This
simplifies the ability for IT departments within Universities to
properly budget.  Additionally the university saw a tremendous \TODO{"tremendous" is subjective, please avoid} cost
savings of 38,000 dollars.  This model also encourages enhancing
security because cost does not influence which servers to secure.

\section{Students Only}

The bottom line of a university is not the only one who sees the cost
benefits of InCommon \cite{www-casestudy2}.  Student verification
provider known as students only is a way for students to enroll to
verify their status as a student.  This verification is then passed to
businesses that would like to offer discounts to students.  To prevent
nonstudents from taking advantage of generous \TODO{"generous" is
  subjective and more appropriate for an advertisment; please avoid.}
offerings of companies it can be cumbersome for a student to properly
verify their status.  With the help of InCommon Students Only helped
streamline the process for students to verify their identity in a
single sign on.  This reassured the companies and students were able
to save money without the difficulties of personally handling identity
verification.

\TODO{Like the Cost section, this is bordeline out of scope, and
  sounds a bit like a marketing pitch. Consider skipping unless you
  can provide some source that more neutrally establishes how
  something like this is beneficial to students.}

\section{Conclusion}

As the number of services on the web continue to grow it can be quite
challenging for both universities and service providers to properly
manage accessibility manage identities.  InCommon hopes to address
this issue by bringing U.S. educational institutes into the same
federation.  This will create a common groundwork of policies and
procedures related to identity management.  Through this unity, users
such as faculty, staff, and students alike can benefit from the
obvious conveniences of single sign on.  However, they will also benefit
from enhanced security and privacy.  Institutions that have entered
into InCommon have seen benefits such as cost savings over competitors
in this market as well as simplification of the billing process for
University IT.  The unlimited certificate model as well as the diverse
types of certificates allows IT flexibility to issue the appropriate
certificate without the worry of budgeting constraints.  Partners such
as Duo further improve security through two factor authentication
dramatically improving the protection of the user.


% Bibliography

\bibliography{references}
 


\end{document}
