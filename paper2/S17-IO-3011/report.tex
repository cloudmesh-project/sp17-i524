\documentclass[9pt,twocolumn,twoside]{../../styles/osajnl}
\usepackage{fancyvrb}
\journal{i524} 

\title{Apache Crunch}

\author[1,*]{Scott McClary}

\affil[1]{School of Informatics and Computing, Bloomington, IN 47408, U.S.A.}

\affil[*]{Corresponding authors: scmcclar@indiana.edu}

\dates{paper-002, \today}

\ociscodes{Big-Data, Cloud, Hadoop, MapReduce}
\doi{\url{https://github.com/cloudmesh/sp17-i524/blob/master/paper2/S17-IO-3011/report.pdf}}

\begin{abstract}
Apache Crunch is an Application Programming Interface (i.e. API)
designed for the Java programming language. This software is built on
top of Apache Hadoop as well as Apache Spark and simplifies the
process of developing MapReduce pipelines. Apache Crunch abstracts
away the explicit need to manage MapReduce jobs. This defining
characteristic alleviates much of the steep learning curve inherently
within developing scalable applications that utilize a MapReduce type
approach. Therefore, developers using Apache Crunch are able to
streamline the process of converting Big Data solutions into runnable
code. As a result, this Java API is leveraged in industry and academia
to develop efficient, scalable and maintainable codebases for Big Data
solutions.
\newline
\end{abstract}

\setboolean{displaycopyright}{true}

\begin{document}

\maketitle

\section{Introduction} \label{introduction} 
Apache Crunch is an open source Java API that is ``built for
pipelining MapReduce programs which are simple and efficient;'' more
specifically, Crunch allows developers to write, test and run
MapReduce pipelines with minimal upfront investment
\cite{www-hadoop-ecosystem, www-mapreduce}. The minimal upfront
investment of this API lowers the barrier for entry for Big Data
developers.

Apache Crunch's purpose is to make ``writing, testing, and running
MapReduce pipelines easy, efficient, and even fun''
\cite{www-wills-crunch, www-mapreduce}. This open source Java API
provides a ``small set of simple primitive operations and lightweight
user-defined functions that can be combined to create complex,
multi-stage pipelines'' \cite{www-wills-crunch}. Apache Crunch
abstracts away much of the complexity from the user by compiling ``the
pipeline into a sequence of MapReduce jobs and manages their
execution'' \cite{www-wills-crunch, www-mapreduce}.

\subsection{History} \label{about}
Josh Wills at Cloudera was the major contributor/developer to the
Crunch project in 2011 \cite{www-crunch-about,
  www-wills-cloudera}. The original version of this software was based
on Google's FlumeJava library \cite{FlumeJava-paper-2012,
  www-crunch-about, www-wills-cloudera}. From 2011 until May 2012
(i.e. version 0.2.4), the Apache Crunch project was open sourced at
GitHub \cite{www-crunch-about, www-github}. After May 2012, the
original Crunch source code was donated to Apache by Cloudera and
shortly after ``the Apache Board of Directors established the Apache
Crunch project in February 2013 as a new top level project''
\cite{www-crunch-about}. Since February 2013, the Apache Crunch
project continues to be used, maintained and improved in an open
source fashion by the software's user and developer community. The
user community has grown to include large reputable companies such as
Spotify and Cerner \cite{www-crunch-spotify, www-crunch-cerner}.

\subsection{Advantages}
As Hadoop continues to grow in popularity, the variation of data
(i.e. satellite images, time series data, audio files, and
seismograms) that is stored in HDFS grows as well
\cite{www-wills-crunch, www-apache-hadoop}. Many of these data
``formats are not a natural fit for the data schemas imposed by Pig
and Hive;'' therefore, ``large, custom libraries of user-defined
functions in Pig or Hive'' or ``MapReduces in Java'' have to be
written, which significantly ``drain on developer productivity''
\cite{www-wills-crunch, www-apache-pig, www-apache-hive,
  www-mapreduce}. The Crunch API provides an alternative solution,
which does not inhibit developer productivity. Apache crunch
integrates seamlessly into Java and therefore, allows developers full
access to Java to write functions. Thus, Apache Crunch is ``especially
useful when processing data that does not fit naturally into
relational model, such as time series, serialized object formats like
protocol buffers or Avro records, and HBase rows and columns''
\cite{www-crunch-api, www-apache-avro, www-apache-hbase}.

\subsection{API} \label{api}
Apache Crunch is a Java API that is used ``for tasks like joining and
data aggregation that are tedious to implement on plain MapReduce''
\cite{www-crunch-api}. The Apache Software Foundation provides
thorough documentation of the API for Apache Crunch and even provides
useful examples of how to explicitly leverage this API from a Java
application \cite{www-crunch-api}.

\subsubsection{Shell Access} \label{shell}
For users of the Scala programming language, there is the ``Scrunch
API, which is built on top of the Java APIs and includes a REPL
(read-eval-print loop) for creating MapReduce pipelines''
\cite{www-crunch-api}.

\section{Licensing} \label{licensing}
The Apache Software Foundation, which includes the software tool named
Apache Crunch, is licensed under the Apache License, Version 2.0
\cite{www-apache-lic}.

\subsection{Source Code}\label{source}
Apache Crunch leverages Git for version control, which allows the user
and developer communities to contribute freely to this open source
project \cite{www-crunch-git, www-github}.

\section{Architecture \& Ecosystem} \label{ecosystem}
In the simplest of terms, Apache Crunch runs on top of Hadoop
MapReduce and Apache Spark \cite{www-crunch-api}. Therefore, Apache
Crunch abstracts away the need for the programmer to explicitly manage
the MapReduce jobs through a Java API. However, the Apache Crunch's
place within the software stack (i.e. on top of Hadoop MapReduce and
Apache Spark) indicates its reliance on the MapReduce software
subsystem. Given Apache Crunch's dependence on Hadoop MapReduce and
Apache Spark, this API provides the ability for developers use the
Java programming language to efficiently and effectively leverage
MapReduce style processing to solve their complicated and complex Big
Data problems.

\section{Use Cases} \label{use}
Apache Crunch has its applicability in the Cloud Computing and Big
Data industry, as shown in the following section. The widespread usage
of Java, Apache Hadoop and Apache Spark in Cloud Computing help
promote Apache Crunch in industry and academia alike.

\subsection{Use Cases for Big Data} \label{big}
The Apache Hadoop ecosystem indicates that Apache Crunch is built on
top of Hadoop MapReduce and Apache Spark, which both go hand in hand
in solving many complicated and challenging Big Data problems. The
following sections demonstrate Apache Crunch's applicability in Big
Data problems. Furthermore, these use cases explains that the software
facilitates the rapid and clean development of the respective Big Data
solutions. The benefits realized at companies such as Cerner and
Spotify are due in part to Apache Crunch's well-defined applicability
in the Big Data space.

\subsection{Cerner} \label{cerner}
Cerner, ``an American supplier of health information technology (HIT)
solutions, services, devices and hardware'' \cite{www-cerner}, employs
Apache Crunch to solve many of their Big Data problems
\cite{www-crunch-cerner}. Cerner decided to use Apache Crunch since it
interestingly solves what they refer to as ``a people problem''
\cite{www-crunch-cerner}. As a company, they have noticed that Apache
Crunch diminishes a potential steep learning curve for new employees
and/or teams to leverage Big Data technologies in their projects.

Cerner definitively believes that Apache Crunch stands above the other
``options available for processing pipelines including Hive, Pig, and
Cascading'' since the Apache Crunch API allows their employees to
straightforwardly code solutions to Big Data problems
\cite{www-crunch-cerner, www-apache-hive, www-apache-pig,
  www-cascading}. The diminished learning curve as a result of using
Apache Crunch allows Cerner to focus their time, effort and money on
performance tuning and/or algorithm adjustments rather than wasting a
significant amount the developers time simply translating a Big Data
problem into runnable and efficient MapReduce code
\cite{www-crunch-cerner}.

\subsection{Spotify} \label{spotify}
Spotify, the popular ``music, podcast, and video streaming service''
\cite{www-spotify}, leverages Apache Crunch to process the many
terabytes of data generated every day by their large user community
\cite{www-crunch-spotify}. Spotify has been using Apache Hadoop since
2009 and have spent significant effort since then to develop tools
that make it simple for the Spotify ``developers and analysts to write
data processing jobs using the MapReduce approach in Python''
\cite{www-crunch-spotify, www-mapreduce}.

However, in 2013 Spotify came to the realization that this approach
wasn't performing well enough so they decided to start using Java and
Apache Crunch to solve their Big Data problems
\cite{www-crunch-spotify}. This transition to Apache Crunch resulted
in higher performance, higher-level abstractions (e.g.  filters, joins
and aggregations), pluggable execution engines (e.g. MapReduce and
Apache Spark) and added simple powerful testing (e.g. fast in-memory
unit tests) \cite{www-crunch-spotify}. Apache Crunch has given Spotify
a significant enhancement for both their ``developer productivity and
execution performance on Hadoop'' \cite{www-crunch-spotify}.

\section{Educational Material} \label{educational}
Apache Crunch makes the process of developing applications that
leverage MapReduce and Apache Spark easier; therefore, the learning
curve is much less significant in relation to developing applications
that directly interact with MapReduce and Apache Spark. The Apache
Software Foundation provides a lot of useful documentation. For
instance, there is API documentation \cite{www-apache-docs} as well as
getting started information \cite{www-crunch-started}, a user guide
\cite{www-crunch-user-guide} and even source code installation
information \cite{www-crunch-git}. If this is not enough, complete and
extensive third-party code examples explain how to develop ``hello
world'' applications that use Apache Crunch
\cite{www-crunch-tutorial}.

\section{Conclusion} \label{conclusion}
In general, Apache Crunch simplifies the process of writing and
maintaining large-scale parallel codes by abstracting away the need to
manage MapReduce jobs. This abstraction diminishes the inherent
learning curve in solving Big Data problems and therefore allows
developers to focus their time and effort in developing the general
concept of their solution rather than in the detailed process of
writing their code. The aforementioned benefits of Apache Crunch are
proven by its widespread use in industry (e.g. Spotify and Cerner) and
in academia. This software tool helps diminish the gap between domain
scientists solving Big Data problems and the potentially complicated
Computer Science tools/mechanisms provided to the Cloud Computing/Big
Data community.

\section*{Acknowledgements}
The authors would like to thank the School of Informatics and
Computing for providing the Big Data Software and Projects (INFO-I524)
course \cite{www-i524}. This paper would not have been possible
without the technical support \& edification from Gregor von Laszewski
and his distinguished colleagues.

 
\section*{Author Biographies}
\begingroup
\setlength\intextsep{0pt}
\begin{minipage}[t][3.2cm][t]{1.0\columnwidth} 
  \begin{wrapfigure}{L}{0.25\columnwidth}
    \includegraphics[width=0.25\columnwidth]{images/scott_mcclary}
  \end{wrapfigure}
  \noindent
  {\bfseries Scott McClary} received his BSc (Computer Science) and
  Minor (Mathematics) in May 2016 from Indiana University and will
  receive his MSc (Computer Science) in May 2017 from Indiana
  University. His research interests are within scientific application
  performance analysis on large-scale HPC systems. He will begin
  working as a Software Engineer with General Electric Digital in San
  Ramon, CA in July 2017.
\end{minipage}
\endgroup

\section*{} %used to create a little more spacing..
\section*{Work Breakdown}
The work on this project was distributed as follows between the
authors:
\begin{description}
\item[Scott McClary.] He completed all of the work for this paper
  including researching and testing Apache Airavata as well as
  composing this technology paper.
\end{description}

% Bibliography
\bibliography{references}
\end{document}

