\documentclass[9pt,twocolumn,twoside]{../../styles/osajnl}
\usepackage{fancyvrb}
\journal{i524} 

\title{Jelastic}

\author[1]{Ajit Balaga, S17-IR-2004}

\affil[1]{School of Informatics and Computing, Bloomington, IN 47408, U.S.A.}

\affil[*]{Corresponding authors: abalaga@iu.edu, ajit.balaga@gmail.com}

\dates{project-000, \today}

\ociscodes{Cloud, I524}

% replace this with your url in github/gitlab
\doi{\url{https://github.com/argetlam115/sp17-i524/blob/master/paper2/S17-IR-2004/report.pdf}}

\begin{abstract}
Jelastic (acronym for Java Elastic) is an unlimited PaaS and Container based 
IaaS within a single platform that provides high availability of applications, 
automatic vertical and horizontal scaling via containerization to software 
development clients, enterprise businesses, DevOps, System Admins, Developers, 
OEMs and web hosting providers.\cite{www-jelastic2}\newline
\end{abstract}

\setboolean{displaycopyright}{true}

\begin{document}

\maketitle

\section{Introduction}
Enterprises are moving away from traditional virtualization solutions and 
transitioning into the cloud. As software development in the enterprise becomes 
more agile, there is an equivalent demand on IT to provide an infrastructure 
that is responsive, scalable, highly available and secure. Enterprise IT 
departments are responding with private cloud and hybrid cloud solutions that 
provide IT-as-a-Service, a utility approach that delivers an agile 
infrastructure to the user community, with ultimate control for administrators 
but self-management for developers and users where appropriate. The promise of 
the cloud was: simplicity, scalability, availability and reduced operating cost. 
However, enterprises are quickly finding that current large-scale cloud 
implementations are often complicated and expensive, often requiring the help of 
third party integrators. Jelastic is a cloud service that solves the above 
problems the above promises and allows enterprises to speed up the development 
process.This paper introduces the architecture behind Jelastic.\cite{www-jelastic6}

\section{Jelastic}
Jelastic is a cloud platform solution that combines benefits of both Platform as
a Service (PaaS) and Container as a Service cloud models, using an approach
unleashes the full potential of a cloud for enterprises, ISVs, hosting service
providers and developers. Jelastic software encompasses PaaS functionality, a
complete infrastructure, smart orchestration, and containers support - all
together.\cite{www-jelastic1}

Jelastic solutions benefits for all kinds of clients:
\begin{itemize}
\renewcommand{\labelitemi}{\scriptsize$\square$}
\item enterprises
\item hosting providers
\item developers
\end{itemize}\cite{www-jelastic1}

Some of the Jelastic key benefits:
\begin{itemize}
\renewcommand{\labelitemi}{\scriptsize$\square$}
\item a turnkey Platform for Public, Private, Hybrid and Multi-Cloud deployments
with automated continuous integration, delivery and upgrade processes
\item support of numerous software stacks, extended with cartridges packaging
model and custom Docker containers\cite{www-jelastic2}
\item automated replication and true automated scaling, both vertical and
horizontal - all applications scale up and down on demand
\item various development environments for the most comfortable work experience
- intuitive UI, open API and SSH access to containers
\item intelligent workloads distribution with multi-cloud and multi-region
management\cite{paper-jelastic2}
\item smart pricing integration - alongside multiple billing systems support, 
Jelastic provides comprehensive billing engine, quotas and access control
policies
\item embedded troubleshooting tools for metering, monitoring, logging, etc.
\end{itemize}\cite{www-jelastic1}

\section{Architecture}
A consistent outline of the underlying Jelastic components with pointers to the
corresponding documentation, namely:\cite{www-jelastic3}
%\begin{itemize}
%\item Сloudlet
%\item Сontainer
%\item Layer
%\item Environment
%\item Application
%\item Hardware Node
%\item Environment Region
%\item Jelastic Platform
%\end{itemize}
%

\subsection{Cloudlet}
Cloudlet is a special infrastructure component that equals to 128 MiB of RAM and
400 MHz of CPU power simultaneously. Such high granularity of resources allows
the system to allocate the exactly required capacity for each instance in the
environment. There are two types of cloudlets:
\begin{itemize}
\renewcommand{\labelitemi}{\scriptsize$\square$}
\item Reserved Cloudlets are fixed amount of resources reserved in advance.
Reserved cloudlets are used when the application load is permanent.
\item Dynamic Cloudlets are added and removed automatically according to the
amount of resources required. Dynamic cloudlets are used for applications with
variable load or when it cannot be predicted in advance.
\end{itemize}\cite{www-jelastic1}

\subsection{Container}
Container (node) is an isolated virtualized instance, provisioned for software
stack handling and placed on a particular hardware node. Each container can be
automatically scaled, both vertically and horizontally, making hosting of
applications truly flexible. The platform provides certified containers for a
lot of commonly used languages and the ability to deploy custom Docker
containers. Each container has its own private IP and unique DNS record.\cite{www-jelastic4}

\subsection{Layer}
Layer (node group) is a set of similar containers in a single environment. There
is a set of predefined layers within Jelastic topology wizard for certified
containers, such as:\cite{paper-jelastic1}
\begin{itemize}
\renewcommand{\labelitemi}{\scriptsize$\square$}
\item load balancer (LB)
\item compute (CP)
\item database (DB)
\item data storage (DS)
\item cacheVPS
\item build node
\item extra (custom layer)
\end{itemize}\cite{www-jelastic1}

The layers are designed to perform different actions with the same type of
containers at once. The nodes can be simultaneously restarted or redeployed, as
well as horizontally scaled manually or automatically based on the load triggers,
checked for errors in the common logs and stats and make the required
configurations via file manager for all containers in a layer. The containers of
one layer are distributed across different hardware servers.\cite{www-jelastic5}

\subsection{Environment}
Environment is a collection of isolated containers for running particular
application services. Jelastic provides built-in tools for convenient
environment management. There is a number of actions that can be performed for
the whole environment, such as stop, start, clone, migrate to another region,
share with team members for collaborative work, track resource consumption, etc.
Each environment has its own internal 3rd level domain name by default. A custom
external domain can be easily bound or even further swapped with another
environment for traffic redirection.\cite{www-jelastic2}

\subsection{Application}
Application is a combination of environments for running one project. A simple
application with one or two stacks can be run inside a single environment.
Applications with more complex topology usually require more flexibility during
deploy or update processes They may be distributed across different types of
servers and several environments, to be maintained independently. Application
source code can be deployed from:\cite{paper-jelastic2}
\begin{itemize}
\renewcommand{\labelitemi}{\scriptsize$\square$}
\item GIT/SVN repository
\item local archive
\item custom Docker template
\end{itemize}\cite{www-jelastic1}

\subsection{Hardware Node}
Hardware node is a physical server or a big virtual machine that is virtualized
via KVM, ESXi, Hyper-V, etc. Hardware nodes are sliced into small isolated
containers that are used to build environments. Such partition provides the
industry-leading multitenancy, as well as high density and smart resource
utilization with the help of containers distribution according to the load
across hardware nodes.\cite{www-jelastic3}\cite{paper-jelastic1}

\subsection{Environment Region}
Environment region is a set of hardware nodes orchestrated within a single
isolated network. Each environment region has its own capacity in a specific
data centre, predefined pool of private and public IP addresses and \cite{paper-jelastic2}
corresponding resource pricing. Moreover, the initially chosen location can be
effortlessly changed by migrating the project between available regions.\cite{www-jelastic4}


\subsection{Jelastic Platform}
Jelastic Platform is a group of environment regions and cluster orchestrator to
control and act like a single system. This provides versatile possibilities to
develop, deploy, test, run, debug and maintain applications due to the multiple
options while selecting hardware - different capacity, pricing, location, etc.
The platform provides a multi-data center or even multi-cloud solution for
running your applications within a single panel, where each Platform is
maintained by a separate hosting service provider with its local support team.\cite{www-jelastic1}

\section{Conclusion}
For enterprises, moving from traditional virtualization to the cloud using PaaS
and IaaS can be a daunting proposition. However, the cloud market is growing 
rapidly and enterprises are recognizing that PaaS allows them to develop and 
deploy scalable, highly available cloud-based applications in a rapid and agile 
fashion. Enterprises can capitalize on this new and sticky revenue stream by 
quickly implementing PaaS and establishing a brand-defining presence in the 
market. Jelastic provides the only integrated private cloud solution that 
integrates PaaS/IaaS and is specifically built for enterprises.\cite{www-jelastic6}

\bibliography{references}

%\subsection{Sample Table}
%Table \ref{tab:shape-functions} shows an example table.
%\begin{table}[htbp]
%\centering
%\caption{\bf Shape Functions for Quadratic Line Elements}
%\begin{tabular}{ccc}
%\hline
%local node & $\{N\}_m$ & $\{\Phi_i\}_m$ $(i=x,y,z)$ \\
%\hline
%$m = 1$ & $L_1(2L_1-1)$ & $\Phi_{i1}$ \\
%$m = 2$ & $L_2(2L_2-1)$ & $\Phi_{i2}$ \\
%$m = 3$ & $L_3=4L_1L_2$ & $\Phi_{i3}$ \\
%\hline
%\end{tabular}
%  \label{tab:shape-functions}
%\end{table}

\end{document}
